%
%\documentclass[bj,numbers]{imsart}% uncomment this for numbers citation
\documentclass[bj,authoryear,noshowframe]{imsart}

%% Packages
% \usepackage{}

\startlocaldefs
%%%%%%%%%%%%%%%%%%%%%%%%%%%%%%%%%%%%%%%%%%%%%%
%%                                          %%
%% Uncomment next line to change            %%
%% the type of equation numbering           %%
%%                                          %%
%%%%%%%%%%%%%%%%%%%%%%%%%%%%%%%%%%%%%%%%%%%%%%
%\numberwithin{equation}{section}
%%%%%%%%%%%%%%%%%%%%%%%%%%%%%%%%%%%%%%%%%%%%%%
%%                                          %%
%% For Axiom, Claim, Corollary, Hypothesis, %%
%% Lemma, Theorem, Proposition              %%
%% use \theoremstyle{plain}                 %%
%%                                          %%
%%%%%%%%%%%%%%%%%%%%%%%%%%%%%%%%%%%%%%%%%%%%%%
\theoremstyle{plain}
\newtheorem{axiom}{Axiom}
\newtheorem{claim}[axiom]{Claim}
\newtheorem{theorem}{Theorem}[section]
\newtheorem{lemma}[theorem]{Lemma}
%%%%%%%%%%%%%%%%%%%%%%%%%%%%%%%%%%%%%%%%%%%%%%
%%                                          %%
%% For Assumption, Definition, Example,     %%
%% Notation, Property, Remark, Fact         %%
%% use \theoremstyle{remark}                %%
%%                                          %%
%%%%%%%%%%%%%%%%%%%%%%%%%%%%%%%%%%%%%%%%%%%%%%
\theoremstyle{remark}
\newtheorem{definition}[theorem]{Definition}
\newtheorem*{example}{Example}
\newtheorem*{fact}{Fact}
%%%%%%%%%%%%%%%%%%%%%%%%%%%%%%%%%%%%%%%%%%%%%%
%% Please put your definitions here:        %%
%%%%%%%%%%%%%%%%%%%%%%%%%%%%%%%%%%%%%%%%%%%%%%
\usepackage{caption}
% \usepackage{enumerate}
\usepackage{subcaption}
\usepackage{float}

\usepackage{graphicx}
% \usepackage{color}

\usepackage{tikz}
\usetikzlibrary{automata,topaths}
\usetikzlibrary{shapes}
\usetikzlibrary{plotmarks}

\usepackage{accents} 

\let\Horig\H
\DeclareMathAlphabet{\eufrak}{U}{}{}{} 
\SetMathAlphabet\eufrak{normal}{U}{euf}{m}{n}
\SetMathAlphabet\eufrak{bold}{U}{euf}{b}{n}

\newcommand{\IP}{\mathbb{P}}
\newcommand{\E}{\mathbb{E}}
\newcommand{\N}{\mathbb{N}}
\newcommand{\R}{\mathbb{R}}
\newcommand{\bone}{{\bf 1}}

\def\P{\mathbb{P}}
\def\real{{\mathord{\mathbb R}}}

\newtheorem{prop}{Proposition}[section]
\newtheorem{corollary}[prop]{Corollary}
\newtheorem{thm}[prop]{Theorem}
\newtheorem{remark}[prop]{Remark}
\newtheorem{assumption}{Assumption}[section]

\usetikzlibrary{arrows,automata,shapes,positioning,decorations.pathmorphing,snakes}

\tikzset{snake it/.style={-stealth,
decoration={snake, 
    amplitude = .4mm,
    segment length = 2mm,
    post length=0.9mm},decorate}}

\usetikzlibrary{matrix,calc}

\newcommand{\convexpath}[2]{
[   
    create hullnodes/.code={
        \global\edef\namelist{#1}
        \foreach [count=\counter] \nodename in \namelist {
            \global\edef\numberofnodes{\counter}
            \node at (\nodename) [draw=none,name=hullnode\counter] {};
        }
        \node at (hullnode\numberofnodes) [name=hullnode0,draw=none] {};
        \pgfmathtruncatemacro\lastnumber{\numberofnodes+1}
        \node at (hullnode1) [name=hullnode\lastnumber,draw=none] {};
    },
    create hullnodes
]
($(hullnode1)!#2!-90:(hullnode0)$)
\foreach [
    evaluate=\currentnode as \previousnode using \currentnode-1,
    evaluate=\currentnode as \nextnode using \currentnode+1
    ] \currentnode in {1,...,\numberofnodes} {
-- ($(hullnode\currentnode)!#2!-90:(hullnode\previousnode)$)
  let \p1 = ($(hullnode\currentnode)!#2!-90:(hullnode\previousnode) - (hullnode\currentnode)$),
    \n1 = {atan2(\y1,\x1)},
    \p2 = ($(hullnode\currentnode)!#2!90:(hullnode\nextnode) - (hullnode\currentnode)$),
    \n2 = {atan2(\y2,\x2)},
    \n{delta} = {-Mod(\n1-\n2,360)}
  in 
    {arc [start angle=\n1, delta angle=\n{delta}, radius=#2]}
}
-- cycle
}

\tikzset{hide labels/.style={every label/.append style={text opacity=0}}}
\endlocaldefs

\begin{document}

\begin{frontmatter}
\title{Normal approximation of subgraph counts in the random-connection model}
%\title{A sample article title with some additional note\thanksref{t1}}
\runtitle{Normal approximation of subgraph counts in the random-connection model}
%\thankstext{T1}{A sample additional note to the title.}

\begin{aug}
%%%%%%%%%%%%%%%%%%%%%%%%%%%%%%%%%%%%%%%%%%%%%%%
%% ORCID can be inserted by command:         %%
%% \orcid{0000-0000-0000-0000}               %%
%%%%%%%%%%%%%%%%%%%%%%%%%%%%%%%%%%%%%%%%%%%%%%%
\author[A]{\inits{F.}\fnms{Qingwei}~\snm{Liu}\ead[label=e1]{qingwei.liu@ntu.edu.sg}\orcid{0000-0001-9485-950X}}
\author[B]{\inits{S.}\fnms{Nicolas}~\snm{Privault}\ead[label=e2]{nprivault@ntu.edu.sg}\orcid{0000-0003-4148-8543}} 
%%%%%%%%%%%%%%%%%%%%%%%%%%%%%%%%%%%%%%%%%%%%%%
%% Addresses                                %%
%%%%%%%%%%%%%%%%%%%%%%%%%%%%%%%%%%%%%%%%%%%%%%
\address[A]{Division of Mathematical Sciences, School of Physical and Mathematical Sciences, Nanyang Technological University, 21 Nanyang Link, Singapore 637371\printead[presep={,\ }]{e1}}

\address[B]{Division of Mathematical Sciences, School of Physical and Mathematical Sciences, Nanyang Technological University, 21 Nanyang Link, Singapore 637371\printead[presep={,\ }]{e2}}
\end{aug}

\begin{abstract}
  This paper derives normal approximation results for subgraph counts written as multiparameter stochastic integrals in a random-connection model based on a Poisson point process. By combinatorial arguments we express the cumulants of general subgraph counts using sums over connected partition diagrams, after cancellation of terms obtained by M\"obius inversion. Using the Statulevi\v{c}ius condition, we deduce convergence rates in the Kolmogorov distance by studying the growth of subgraph count cumulants as the intensity of the underlying Poisson point process tends to infinity. Our analysis covers general subgraphs in the dilute and full random graph regimes, and tree-like subgraphs in the sparse random graph regime.
\end{abstract}

\begin{keyword}
  \kwd{cumulant method}
  \kwd{Kolmogorov distance}
  \kwd{normal approximation}
  \kwd{Poisson point process}
  \kwd{Random-connection model} 
  \kwd{random graphs}
  \kwd{subgraph count}
\end{keyword}

\end{frontmatter}

\vspace{-0.1cm}
\section{Introduction}
\vspace{-0.1cm}
This paper treats the asymptotic behavior of
 random subgraph counts in the random-connection model (RCM), 
 which is used to model physical systems in e.g. wireless networks, % \cite{georgiou2013,georgiou2015,georgiou2016,kartungiles2016},
 complex networks, % \cite{mulder2018,boguna2020},
 and statistical mechanics. % \cite{bianconi2001,betz2008,biskup2015}, 
% or cosmology \cite{cunningham2017,fountoulakis2020}. 
 Our approach relies on the study of cumulant growth rates
 as the intensity of the underlying Poisson point process tends to infinity. 

The distributional approximation of subgraph counts has attracted
significant interest in the random graph literature. 
 In \cite{rucinski}, conditions for the asymptotic normality
 of renormalized subgraph counts have been obtained
 in the Erd{\H o}s-R\'enyi random graph model \cite{ER,G}. 
% $\mathbb{G}_n(p_n)$
% that are isomorphic to a fixed graph $G$ obtained
 In \cite[Chapter~3]{penrosebk}, non-quantitative central limit theorems have been obtained for subgraph and component counts
 on random geometric graphs.
   Rates of normal convergence with respect to the Kolmogorov distance
   have been obtained for certain random
   functionals on random geometric graphs in
   \cite{schulte} using Poisson U-statistics,
   see also \cite{lachiezerey4} for the use of stabilizing functionals.

 In the Erd{\H o}s-R\'enyi setting,
 those results have been made more precise in \cite{BKR} by
 the derivation of convergence rates in the Wasserstein distance via the
 Stein method. 
 They have also been strengthened in \cite{reichenbachsAoP} 
 using the Kolmogorov distance in the case of triangle counts,
 and in \cite{PS2} in the case of general subgraphs $G$.
 The case of triangles has also been treated in \cite{roellin2}
 by the {S}tein-{T}ikhomirov method, which has been extended 
 to general subgraphs in \cite{rednos}.    
 In \cite{khorunzhiy}, the counts of line ($X$-model) and cycles ($Y$-model) 
 in discrete Erd{\H o}s-R\'enyi models 
 have been analyzed via the asymptotic behavior of their cumulants.

 The random connection-model is a natural generalization of the % random geometric graph and
 Erd\H os-R\'enyi random graph in which vertices are randomly located
 and can be connected
 with location-dependent probabilities
 $H(x,y)\in [0,1]$.
%\textcolor{purple}{ In the (deterministic) random-connection model with $\{0,1\}$-valued connection function $H(x,y)$, a Central Limit Theorem has been obtained in \cite{can2022} under a weakly stabilizing condition, cf. \cite{penrose01}, plus some moment conditions. As mentioned in \cite[Remark~2.5]{can2022}, the stabilizing condition appeared in \cite{penrose05,lachiezerey4} does not apply to subgraph counts with general $[0,1]$-valued random connection function. }
 Recently, a Central Limit Theorem has been derived in \cite{can2022} for the counts of induced subgraphs in the random-connection model under certain stabilization and moment conditions. 
 Obtaining normal approximation error bounds 
 in the random-connection model
 with a general $[0,1]$-valued random connection function 
 is more difficult due to the additional layer of complexity coming from
 the randomness of vertex locations.
% RCM turns out to be much harder, despite the fruitful studies of random geometric graph and Erd\H os-R\'enyi model.  
 Regarding convergence rates, in \cite{LNS21},
 a central limit theorem and Berry-Esseen convergence rates have
 been presented and applied to the number of components isomorphic
 to a given finite connected graph in the random-connection model,
 together with a study of first moments and covariances.
% However, those results were not applied to subgraph counting. 
 In \cite{zhangzs}, Berry-Esseen convergence rates have
 been obtained for subgraph counts
 in the binomial RCM random-connection (graphon) model. 
 However, those results do not cover the case of
 general subgraph counting in the Poisson RCM. 
% considered in this paper. 
 % In particular, in \cite[Theorem~5.1]{LNS21}, a variance representation for functionals of an edge-marked Poisson process is established.
 % in terms of some ``subtract-one cost''.
 % A brief comparison between the methods of deriving central limit theorems for functionals of spatial point process\\
   % To be more specific, recent work \cite{lachiezerey4} shows that the Malliavin-Stein method, combined with stabilization method, yields rates of normal approximation for general stabilizing functionals of an underlying Poisson or binomial point process under the Kolmogorov distance.
% The rates are presumably optimal.
   The Malliavin-Stein machinery on Poisson space \cite{lastpeccatipenrose} has been applied to in \cite{LNS21} the numbers of components isomorphic to a given graph in the RCM, using edge marked Poisson processes.
   In \cite{can2022} a CLT has been derived for subgraph counts in the RCM
   using edge-marking structure of \cite{LNS21} 
   under a weak stabilizing condition originating from \cite{penrose01}. 
   However, as pointed out in Remark~2.5-$(i)$ of \cite{can2022},
   no convergence rates are derived by this method, as 
   the strong stabilization condition of \cite{penrose05,lachiezerey4}
   is not satisfied by general functionals
   when the connection function $H(x,y)$ is $(0,1)$-valued. 
 On the other hand, the cumulant method, which uses partition diagrams, enables us to establish a quantitative CLT for functionals on the RCM. % Find more discussions in Remark~\ref{rem-1}.

 In this paper, we derive normal approximation rates under a mild condition
 on the connection function $H(x,y)$ of the random-connection model,
 by deriving growth rates of cumulants written as sums over
 connected partitions, see Propositions~\ref{t1} and \ref{th6.4}.  \textcolor{red}{Related cumulant bounds have been obtained in the {E}rd{\Horig{o}}s-{R}\'enyi model, cf. Proposition~10.1.2 in \cite{feray}. However,} to the best of our knowledge, this is the first time that the normal approximation
 of subgraph counts with convergence rates is established
 in the random-connection model.

 In comparison with \cite{khorunzhiy}, which also uses the cumulant method, 
 we obtain convergence rates in the Kolmogorov distance and our results are 
 not restricted to line and cycle graphs, as they cover more general subgraphs,
 see Corollaries~\ref{c01}-\ref{c01-2}. 
% \textcolor{purple}{See more details in Remark~\ref{rem-1}.}
 Furthermore, various random graph regimes are discussed.
 In addition,
   we show in Section~\ref{rgg} that our approach can be
   specialized to derive Kolmogorov rates for
   subgraph counting in the setting of random geometric graphs,
   see Corollary~\ref{jdkj10}. 

 A number of probabilistic conclusions can be derived
 from the behavior of cumulants of random variables using the 
Statulevi\v{c}ius condition, 
 including convergence rates in the Kolmogorov distance
 and moderate deviation principles, see
 \cite{saulis},
 \cite{doring},
 \cite{doering}. \textcolor{red}{In stochastic geometry, the method of cumulant has also been applied to Poisson cylinder processes \cite{heinrich09}, the volumes of simplices in Poisson-Delaunay tessellations \cite{thale21}, the Boolean model \cite{heinrich} and the random $m$-dependent fields \cite{gotze95}.} In \cite{grotethale18,thale18}, 
 this method has been used to derive concentration inequalities, normal approximation with error bounds, and moderate deviation principles for random polytopes. \textcolor{red}{For a more detailed overview, we refer the survey \cite{doering}.}
 
Given $\mu$ a % finite
diffuse \textcolor{red}{$\sigma$-finite} measure on $\R^d$,
 we consider a random-connection model
 based on an underlying Poisson point process $\Xi$ on $\R^d$
 with intensity of the form $\lambda\mu(\mathrm{d}x)$, in which 
 any two vertices $x,y$ in $\Xi$ are connected
 with the probability $H_\lambda(x,y):= c_\lambda H(x,y) \in [0,1]$,
 where $H_\lambda$ is the connection function of the model. 
 Here, we investigate the limiting behavior
 of the count $N_G$ of a given subgraph $G$
 as the intensity $\lambda$ of the underlying Poisson point process on $\R^d$
 tends to infinity. 
 To this end, we use the combinatorics of the cumulants $\kappa_n(N_G)$ 
 based on moment expressions obtained in \cite{prkhp} for
 multiparameter stochastic integrals in the random-connection model.
% In comparison, our approach uses representations of all moments and cumulants of multiparameter Poisson stochastic integrals, while \cite{LNS21} focuses on variance expressions for general functionals of an edge-marked Poisson process. 

Using partition diagrams and dependency graph arguments,
we start by showing in Proposition~\ref{mainthm-1}
that the (virtual) cumulants of a random functional admitting
a certain connectedness factorization property \eqref{dia-factoriz}  
can be expressed as sums over connected partition diagrams, 
 generalizing Lemma~2 in \cite{MalyshevMinlos91}.  
 A related result has been obtained in \cite{jansen}
 in the particular case of two-parameter Poisson stochastic
 integrals, in relation to 
 cluster expansions for Gibbs point processes in statistical
 mechanics. 
 In Proposition~\ref{p01-1}, we apply Proposition~\ref{mainthm-1} 
 to express the cumulants of multiparameter stochastic integrals,
 for which this factorization property can be checked from 
 the moment formulas for 
 multiparameter stochastic integrals computed in Proposition~\ref{p01-1-0}. 
 
 Such expressions allow us to determine the dominant terms in the growth of
 cumulants as the intensity $\lambda$ of the underlying point process tends to infinity,
 by estimating the counts of vertices and edges in connected partition diagrams
 as in \cite{khorunzhiy}. 
 We work under a mild condition \eqref{integ-connecting3} 
 which is satisfied by e.g. any translation-invariant
 continuous connection function $H : \real^d\times \real^d \to [0,1]$ non vanishing at $0$, such as the {Rayleigh} connection function
  given by $H(x,y) = e^{ - \beta \Vert x - y\Vert^2}$, $x,y\in \real^d$, for some $\beta > 0$. 
 
 For our analysis of cumulant behavior
 we identify the leading terms in the sum \eqref{cumulant-diagram1}
 over connected partition diagrams. 
 When $G$ is a connected graph with $|V(G)|=r$ vertices,
 satisfying
 Assumption~\ref{a61} in the dilute regime \eqref{fjnldsf}
 with $\lambda^{-1/\zeta } \ll c_\lambda \leq K$,
 where $\zeta \geq 1$ is defined in \eqref{fjkldf}, 
 the dominant terms are given by connected partition diagrams with the
 highest number of blocks, 
 see also \cite{privaultkhops}
 in the case of $k$-hop counting on the line.
 % in the one-dimensional random-connection model. 
 In Proposition~\ref{t1} this yields the cumulant bounds 
$$
 (n-1)! c_\lambda^{n |E(G)| } ( K_1 \lambda )^{1+(r-1)n} 
 \leq 
  \kappa_n(N_G)
\leq 
n!^r c_\lambda^{n |E(G) |} ( K_2 \lambda )^{1+(r-1)n},
\quad \lambda \geq 1, 
$$
for some constants $K_1$, $K_2>0$ independent of $\lambda, n\geq 1$,
where $E(G)$ denotes the set of edges of $G$.
 From the {Statulevi\v{c}ius condition}
 \eqref{Statuleviciuscond2} below, see \cite{rudzkis,doering},
 letting $\Phi$ denote the cumulative distribution function of the standard normal distribution, 
 we deduce the Kolmogorov distance bound 
$$
\sup_{x\in \real}
\big| \P \big( \widetilde{N}_G \leq x \big) - \Phi (x) \big| \leq
\frac{C }{\lambda^{1/(4r - 2)}},
\qquad \lambda \to \infty, 
$$ 
% as $\lambda$ tends to infinity,
 for the normalized subgraph count $\widetilde{N}_G$, 
see Corollary~\ref{c01}, and a moderate deviation principle, 
% by Theorem~1.1 of \cite{doring}.
 see Corollary~\ref{c01-2-0}. 
  
 In the sparse regime \eqref{fjnldsf-2} where
 $ c_\lambda \leq \lambda^{-\alpha}$ for some $\alpha \geq 1$,
 the maximal rate $\lambda^{
 \alpha         -(\alpha - 1)r 
          }$
 is attained for $G$ a tree-like graph, and
 in Proposition~\ref{th6.4} we obtain the cumulant bounds % equivalence 
$$ 
    \nonumber % \label{cumulant-rhop2}
            (K_1)^{(r-1)n}
    \lambda^{
 \alpha     -(\alpha - 1)r 
      }
     \leq 
% \lambda  \sum_{k=r}^{1+(r-1)n}          (K_1)^k {\cal N}^{(n,r)}_k      \leq 
  \kappa_n(N_G)
  %  \leq  \lambda  \sum_{k=r}^{1+(r-1)n} (K_2)^k {\cal N}^{(n,r)}_k
  \leq
  n!^r
  % r!^n
  (K_2)^{(r-1)n}
  \lambda^{
   \alpha -(\alpha - 1)r 
    } 
  , \quad \lambda > 0, 
$$ 
  if $G$ is a tree, and
$$ 
  \nonumber % \label{cumulant-rhop2}
    (K_1)^r 
  \lambda^{r-\alpha |E(G)|}
  \leq 
  \kappa_n(N_G)
  \leq
    n!^r
  % r!^n
    (K_2)^{(r-1)n}
    \lambda^{r-\alpha |E(G)|}, \quad \lambda >0, 
$$ 
  if $G$ is a not a tree, such as e.g. a cycle graph.
  As a consequence of the {Statulevi\v{c}ius condition}
  \eqref{Statuleviciuscond2}, 
 when $G$ is a tree
 we find the Kolmogorov distance bound % Berry-Esseen
$$  
\sup_{x\in \real}
\big| \P \big( \widetilde{N}_G \leq x \big) - \Phi (x) \big| \leq
 C \lambda^{
    - (
\alpha    -(\alpha - 1)r 
        ) / ( 4r - 2) }
, \qquad \lambda \to \infty, 
$$ 
% as $\lambda$ tends to infinity, 
 provided that $1 \leq \alpha < r/(r-1)$, 
 see Corollary~\ref{c01-2}. 
 
 Convergence rates in the Kolmogorov distances may be improved
 into classical Berry-Esseen rates when 
 the connection function $H(x,y)$ is $\{0,1\}$-valued, e.g. in disk models
 as in \cite{privaultkhops}, 
 by representing subgraph counts as multiple Poisson stochastic integrals
 and using the fourth moment theorem for $U$-statistics and sums of
 multiple stochastic integrals Corollary~4.10 in \cite{eichelsbacher}, 
 see also Theorem~3 in \cite{lachieze-rey} 
 or Theorem~6.3 in \cite{PS4} for Hoeffding decompositions.
 On the other hand, the study of stabilizing functionals \cite{penrose05,lachiezerey4} yields normal approximation with rates for random functionals written represented as sums of stabilizing score functions on random geometric graphs. In the general case where $H(x,y)$ is $[0,1]$-valued, both methods no longer apply, which is why we rely on the {Statulevi\v{c}ius condition} which in turn may yield suboptimal convergence rates. 
 
 This paper is organized as follows.
 Sections~\ref{s2} and \ref{s3} introduce the preliminary
 framework and notations on connected partition diagrams
 and combinatorics of virtual cumulants that will be used for 
 the expression of cumulants of multiparameter stochastic integrals
 in Section~\ref{s4} and for subgraph counts in Section~\ref{s5}. 
 Those expressions are applied in Section~\ref{s6}
 to derive cumulant growth rates in the random-connection model, with application to Kolmogorov rates in subgraph counting via the {Statulevi\v{c}ius condition} in Section~\ref{s6-1}.
 In Section~\ref{rgg}, normal approximation for subgraph counts on the random geometric graph is discussed under different limiting regimes. 

 \section{Set partitions and diagram connectivity} % , connectivity and virtual cumulants} 
\label{s2}
\noindent
Given $\eta$ a finite set, we denote by $\Pi ( \eta )$ the collection
of its set partitions, and we let $|\sigma|$ denote the number of blocks in any partition $\sigma \in \Pi ( \eta )$. 
Given $\rho,\sigma$ two set partitions, we say that $\sigma$ is coarser than $\rho$,
or that $\rho$ is finer than $\sigma$,
and we write $\rho\preceq\sigma$,
if every block in $\sigma$ is a combination of blocks in $\rho$. 
We also denote by $\rho\vee\sigma$ the finest partition which is coarser than $\rho$ and $\sigma$, and by $\rho\wedge\sigma$ the coarsest partition that is finer than $\rho$ and $\sigma$.
We let $\widehat{0}$ be the finest partition, which is made of a single element in each block, and we let $\widehat{1}$ be the coarsest (one-block) partition. 
In general, given any graph $G$ we denote by $V(G)$ the set of its vertices, and by $E(G)$ the set of its edges.

\medskip
 
Our study of cumulants and moments of functionals of random fields
relies on partition diagrams, see \cite{MalyshevMinlos91,khorunzhiy,peccatitaqqu}
and references therein for additional background. 
In what follows we let $[n]:=\{1,2,\dots,n\}$ for $n\geq 1$.
\begin{definition}
  Let $n,r\geq 1$. 
\begin{enumerate}%[\rm 1.] 
\item Given $\eta \subset [n]$ we let
$\Pi ( \eta \times [r])$ denote the set of all partitions of the set 
 $$
 \eta \times [r] := 
 \big\{ (k,l) \ : \
 k\in \eta, \ l = 1,\ldots , r \big\}. 
$$
 % of cardinality $n \times [r]$,
 % identified to $\{1,\ldots , nr\}$,
\item
  We also let 
$\pi_\eta : = (\pi_i)_{i\in \eta} \in \Pi ( \eta \times [r])$ denote the
 partition made of the $|\eta|$ blocks
 of size $r$ given by 
 $$\pi_k := \{ (k,1), \ldots , (k,r) \}, \qquad k\in \eta. 
$$
\end{enumerate} 
\end{definition}
Next, we introduce the definition of partition diagrams.
\begin{definition}
  Let $n,r \geq 1$.
  Given $\eta\subset [n]$ and $\rho\in\Pi(\eta\times[r])$
 a partition of $\eta \times [r]$, we denote by $\Gamma (\rho,\pi_\eta )$
 the diagram, or graphical representation of the partition $\rho$,
 constructed by: 
\begin{enumerate}%[\rm 1.] 
\item arranging the elements of $\eta \times [r]$
 into an % $|\eta |\times r$
 array of $|\eta |$ rows and $r$ columns, and
\item
 connecting all elements within a same block of $\rho$ 
 by a tree graph. 
\end{enumerate} 
% \medskip
In addition, we say that the partition diagram $\Gamma(\rho,\pi )$
 is connected when $\rho\vee\pi_\eta=\widehat{1}$. 
\end{definition}
\noindent 
 For shortness of notation, in the sequel  
 we say that a partition $\rho$ is connected 
 when its diagram $\Gamma ( \rho , \pi )$ is connected. 
 For example, taking $\eta := \{2,3,5,8,10\}$, given the partitions 
\begin{align*}
  \rho = \big\{
 & \{(2,1),(3,1),(3,2),(3,3)\}, 
\{(2,2),(2,3),(2,4),(3,4)\},
\{(5,1)\},
\{(5,2),(8,2)\},
\\
& \{(5,3)\},
\{(5,4),(8,3)\},
\{(8,1),(10,1)\},
\{(8,4)\},
\{(10,2),(10,3),(10,4)\}\big\}
\end{align*} 
and
\begin{align*}
  \sigma = \big\{ & 
  \{(2,1),(3,1)\},
  \{(2,2)\},
  \{(2,3),(3,4)\},
  \{(2,4)\},
  \{(3,2),(5,2),(8,2)\},
  \\
  &
  \{(3,3),(5,4),(8,3),(10,2)\},
  \{(5,1)\},
  \{(5,3)\},
  \{(8,1),(10,1)\},
  \{(8,4)\},
  \{(10,3)\},
  \{(10,4)\}
  \big\}, 
\end{align*} 
of $\eta \times [4]$,
Figure~\ref{fig:diagram0}-$a)$ presents an example of a non-connected partition diagram  
$\Gamma ( \rho , \pi)$,
and Figure~\ref{fig:diagram0}-$b)$ presents an example of a connected partition diagram $\Gamma ( \sigma , \pi)$. 

% \vspace{-0.3cm}

\begin{figure}[H]
\captionsetup[subfigure]{font=footnotesize}
\centering
\subcaptionbox{Non-connected partition diagram $\Gamma(\rho,\pi)$.}[.5\textwidth]{%
\begin{tikzpicture}[scale=0.9] 
\draw[black, thick] (0,0) rectangle (5,6);

\node[anchor=east,font=\small] at (0.8,5) {2};
\node[anchor=east,font=\small] at (0.8,4) {3};
\node[anchor=east,font=\small] at (0.8,3) {5};
\node[anchor=east,font=\small] at (0.8,2) {8};
\node[anchor=east,font=\small] at (0.8,1) {10};

\node[anchor=south,font=\small] at (1,0) {1};
\node[anchor=south,font=\small] at (2,0) {2};
\node[anchor=south,font=\small] at (3,0) {3};
\node[anchor=south,font=\small] at (4,0) {4};

\filldraw [gray] (1,1) circle (2pt);
\filldraw [gray] (2,1) circle (2pt);
\filldraw [gray] (3,1) circle (2pt);
\filldraw [gray] (4,1) circle (2pt);
\filldraw [gray] (1,2) circle (2pt);
\filldraw [gray] (2,2) circle (2pt);
\filldraw [gray] (3,2) circle (2pt);
\filldraw [gray] (4,2) circle (2pt);
\filldraw [gray] (1,3) circle (2pt);
\filldraw [gray] (2,3) circle (2pt);
\filldraw [gray] (3,3) circle (2pt);
\filldraw [gray] (4,3) circle (2pt);
\filldraw [gray] (2,3) circle (2pt);
\filldraw [gray] (1,4) circle (2pt);
\filldraw [gray] (2,4) circle (2pt);
\filldraw [gray] (3,4) circle (2pt);
\filldraw [gray] (4,4) circle (2pt);
\filldraw [gray] (1,5) circle (2pt);
\filldraw [gray] (2,5) circle (2pt);
\filldraw [gray] (3,5) circle (2pt);
\filldraw [gray] (4,5) circle (2pt);

\draw[very thick] (1,5) -- (1,4) -- (2,4) -- (3,4);
\draw[very thick] (2,5) -- (3,5) -- (4,5) -- (4,4);

\draw[very thick] (1,2) -- (1,1);
\draw[very thick] (2,3) -- (2,2);
\draw[very thick] (2,1) -- (3,1) -- (4,1);
\draw[very thick] (3,2) -- (4,3);

\end{tikzpicture}}%
\subcaptionbox{Connected partition diagram $\Gamma(\sigma,\pi)$.}[.5\textwidth]{
\begin{tikzpicture}[scale=0.9] 
\draw[black, thick] (0,0) rectangle (5,6);

\node[anchor=east,font=\small] at (0.8,5) {2};
\node[anchor=east,font=\small] at (0.8,4) {3};
\node[anchor=east,font=\small] at (0.8,3) {5};
\node[anchor=east,font=\small] at (0.8,2) {8};
\node[anchor=east,font=\small] at (0.8,1) {10};

\node[anchor=south,font=\small] at (1,0) {1};
\node[anchor=south,font=\small] at (2,0) {2};
\node[anchor=south,font=\small] at (3,0) {3};
\node[anchor=south,font=\small] at (4,0) {4};

\filldraw [gray] (1,1) circle (2pt);
\filldraw [gray] (2,1) circle (2pt);
\filldraw [gray] (3,1) circle (2pt);
\filldraw [gray] (4,1) circle (2pt);
\filldraw [gray] (1,2) circle (2pt);
\filldraw [gray] (2,2) circle (2pt);
\filldraw [gray] (3,2) circle (2pt);
\filldraw [gray] (4,2) circle (2pt);
\filldraw [gray] (1,3) circle (2pt);
\filldraw [gray] (2,3) circle (2pt);
\filldraw [gray] (3,3) circle (2pt);
\filldraw [gray] (4,3) circle (2pt);
\filldraw [gray] (2,3) circle (2pt);
\filldraw [gray] (1,4) circle (2pt);
\filldraw [gray] (2,4) circle (2pt);
\filldraw [gray] (3,4) circle (2pt);
\filldraw [gray] (4,4) circle (2pt);
\filldraw [gray] (1,5) circle (2pt);
\filldraw [gray] (2,5) circle (2pt);
\filldraw [gray] (3,5) circle (2pt);
\filldraw [gray] (4,5) circle (2pt);

\draw[very thick] (1,5) -- (1,4); 
\draw[very thick] (3,5) -- (4,4);

\draw[very thick] (1,2) -- (1,1);
\draw[very thick] (2,2) -- (2,4);
\draw[very thick] (2,1) -- (3,2) -- (4,3) -- (3,4);

\end{tikzpicture}}%
\caption{Two examples of partition diagrams with $\eta = \{2,3,5,8,10\}$, $n=10$, $r=4$.}
\label{fig:diagram0}
\end{figure}

\vspace{-.4cm}

\noindent 
Note that the above notion of connected partition diagram is distinct from
that of irreducible partition, see, e.g., \cite{eabender}.
\begin{definition}
  Let $n\geq 1$,
  $G$ a connected graph with $|V(G)| = r$ vertices, $r \geq 1$,  
  and consider $G_1,\ldots , G_n$ copies of $G$ respectively built on
  $\pi_1,\ldots , \pi_n$.
  Let also $\rho \in\Pi( [n] \times[r])$
  be a partition of $ [n] \times[r]$. 
\begin{enumerate}%[\rm 1.]
\item
  We let $\widetilde{\rho}_G$ be the multigraph constructed
  on the blocks of $\rho$ 
  by adding an edge between two blocks $\rho_1,\rho_2$ of the
  partition $\rho$ whenever there exist $(k,l_1)\in \rho_1$
  and $(k,l_2)\in \rho_2$ such that $(l_1,l_2)$ is an edge in $G_k$.
  % , where $G_k$ denotes the $k$-th replica of graph $G$;
\item 
  We let $\rho_G$ be the graph constructed
  on the blocks of $\rho$
  by removing redundant edges
  in $\widetilde{\rho}_G$,
  so that at most one edge remains between any two blocks $\rho_1,\rho_2\in\rho$. 
 \end{enumerate}
\end{definition}

\begin{figure}[H]
\captionsetup[subfigure]{font=footnotesize}
\centering
\subcaptionbox{Diagram $\Gamma(\rho,\pi)$ and multigraph $\widetilde{\rho}_G$ in blue.}[.5\textwidth]{%
\begin{tikzpicture}[scale=0.9] 
\draw[black, thick] (0,0) rectangle (5,6);

\node[anchor=east,font=\small] at (0.8,5) {1};
\node[anchor=east,font=\small] at (0.8,4) {2};
\node[anchor=east,font=\small] at (0.8,3) {3};
\node[anchor=east,font=\small] at (0.8,2) {4};
\node[anchor=east,font=\small] at (0.8,1) {5};

\node[anchor=south,font=\small] at (1,0) {1};
\node[anchor=south,font=\small] at (2,0) {2};
\node[anchor=south,font=\small] at (3,0) {3};
\node[anchor=south,font=\small] at (4,0) {4};

\filldraw [gray] (1,1) circle (2pt);
\filldraw [gray] (2,1) circle (2pt);
\filldraw [gray] (3,1) circle (2pt);
\filldraw [gray] (4,1) circle (2pt);
\filldraw [gray] (1,2) circle (2pt);
\filldraw [gray] (2,2) circle (2pt);
\filldraw [gray] (3,2) circle (2pt);
\filldraw [gray] (4,2) circle (2pt);
\filldraw [gray] (1,3) circle (2pt);
\filldraw [gray] (2,3) circle (2pt);
\filldraw [gray] (3,3) circle (2pt);
\filldraw [gray] (4,3) circle (2pt);
\filldraw [gray] (2,3) circle (2pt);
\filldraw [gray] (1,4) circle (2pt);
\filldraw [gray] (2,4) circle (2pt);
\filldraw [gray] (3,4) circle (2pt);
\filldraw [gray] (4,4) circle (2pt);
\filldraw [gray] (1,5) circle (2pt);
\filldraw [gray] (2,5) circle (2pt);
\filldraw [gray] (3,5) circle (2pt);
\filldraw [gray] (4,5) circle (2pt);

\draw[very thick] (1,5) -- (1,4) -- (2,4) -- (3,4);
\draw[very thick] (2,5) -- (3,5) -- (4,5) -- (4,4);

\draw[very thick] (1,2) -- (1,1);
\draw[very thick] (2,3) -- (2,2);
\draw[very thick] (2,1) -- (3,1) -- (4,1);
\draw[very thick] (3,2) -- (4,3);

\draw[thick,dash dot,blue] (1,5) .. controls (1.5,5+.5) .. (2,5);

\draw[thick,dash dot,blue] (3,4) .. controls (3.5,4+.5) .. (4,4);
\draw[thick,dash dot,blue] (2,4) .. controls (3,4-.5) .. (4,4);
\foreach \i in {2,...,3}
         {
\draw[thick,dash dot,blue] (1,\i) .. controls (1.5,\i+.5) .. (2,\i);
\draw[thick,dash dot,blue] (3,\i) .. controls (3.5,\i+.5) .. (4,\i);
\draw[thick,dash dot,blue] (2,\i) .. controls (3,\i-.5) .. (4,\i);
} 

\draw[thick,dash dot,blue] (1,1) .. controls (1.5,1+.5) .. (2,1);

\end{tikzpicture}}%
\subcaptionbox{Diagram $\Gamma(\rho ,\pi)$ and graph $\rho_G$ in red.}[.5\textwidth]{
\begin{tikzpicture}[scale=0.9] 
\draw[black, thick] (0,0) rectangle (5,6);

\node[anchor=east,font=\small] at (0.8,5) {1};
\node[anchor=east,font=\small] at (0.8,4) {2};
\node[anchor=east,font=\small] at (0.8,3) {3};
\node[anchor=east,font=\small] at (0.8,2) {4};
\node[anchor=east,font=\small] at (0.8,1) {5};

\node[anchor=south,font=\small] at (1,0) {1};
\node[anchor=south,font=\small] at (2,0) {2};
\node[anchor=south,font=\small] at (3,0) {3};
\node[anchor=south,font=\small] at (4,0) {4};

\filldraw [gray] (1,1) circle (2pt);
\filldraw [gray] (2,1) circle (2pt);
\filldraw [gray] (3,1) circle (2pt);
\filldraw [gray] (4,1) circle (2pt);
\filldraw [gray] (1,2) circle (2pt);
\filldraw [gray] (2,2) circle (2pt);
\filldraw [gray] (3,2) circle (2pt);
\filldraw [gray] (4,2) circle (2pt);
\filldraw [gray] (1,3) circle (2pt);
\filldraw [gray] (2,3) circle (2pt);
\filldraw [gray] (3,3) circle (2pt);
\filldraw [gray] (4,3) circle (2pt);
\filldraw [gray] (2,3) circle (2pt);
\filldraw [gray] (1,4) circle (2pt);
\filldraw [gray] (2,4) circle (2pt);
\filldraw [gray] (3,4) circle (2pt);
\filldraw [gray] (4,4) circle (2pt);
\filldraw [gray] (1,5) circle (2pt);
\filldraw [gray] (2,5) circle (2pt);
\filldraw [gray] (3,5) circle (2pt);
\filldraw [gray] (4,5) circle (2pt);

\draw[very thick] (1,5) -- (1,4) -- (2,4) -- (3,4);
\draw[very thick] (2,5) -- (3,5) -- (4,5) -- (4,4);

\draw[very thick] (1,2) -- (1,1);
\draw[very thick] (2,3) -- (2,2);
\draw[very thick] (2,1) -- (3,1) -- (4,1);
\draw[very thick] (3,2) -- (4,3);

\draw[thick,dash dot,purple] (1,5) .. controls (1.5,5.5) .. (2,5);
%\draw[thick,dash dot,blue] (2,4) .. controls (3,3.5) .. (4,4);
%\draw[thick,dash dot,blue] (3,4) .. controls (3.5,4.5) .. (4,4);

\draw[thick,dash dot,purple] (1,3) .. controls (1.5,3.5) .. (2,3);
\draw[thick,dash dot,purple] (2,3) .. controls (3,2.5) .. (4,3);
\draw[thick,dash dot,purple] (3,3) .. controls (3.5,3.5) .. (4,3);

\draw[thick,dash dot,purple] (1,2) .. controls (1.5,2.5) .. (2,2);
\draw[thick,dash dot,purple] (2,2) .. controls (3,1.5) .. (4,2);
\draw[thick,dash dot,purple] (3,2) .. controls (3.5,2.5) .. (4,2);

\draw[thick,dash dot,purple] (1,1) .. controls (1.5,1.5) .. (2,1);

\end{tikzpicture}}%
\caption{Diagram and graphs $G$, $\rho_G$, $\widetilde{\rho}_G$ with $n=5$, $r=4$.}
\label{fig:diagram1}
\end{figure}

\noindent
 Figure~\ref{fig:diagram1}-b) presents an illustration of the
multigraph $\widetilde{\rho}_G$ and graph $\rho_G$ on the blocks of $\rho$ when
$G$ is the line graph
$\{(1,2),(2,4),(3,4)\}$ on $\{1,2,3,4\}$. % with vertices $(1,2)$, $(2,4)$, $(3,4)$.  

\begin{definition}
  Let $n , r \geq 1$,
  and let $\rho \in \Pi ([n]\times [r])$ be a partition of $[n]\times [r]$.
  \begin{enumerate}%[\rm 1.] 
  \item
    For $b \subset [n]$, we let $\rho_b \subset \rho$ be defined as 
$$
\rho_b := \{ c \in \rho \ : \ c \subset b\times[r] \}. 
$$
\item Given $\eta \subset [n]$ 
 we split any partition $\rho$ of $\eta\times[r]$ into the equivalence classes
 deduced from the connected components of the graph $\rho_G$, as 
% Since the graph $G$ is connected, we have 
\begin{equation}
\label{fjklds1} 
\rho = \bigcup_{\substack{b\subset\eta\\b\times [r] \in \rho \vee \pi_\eta}} \rho_b, 
\end{equation} 
\end{enumerate}
\end{definition} 
 As an example, in Figure~\ref{fig:diagram1-3}-$a)$, when $b = \{1,2\}$ we have
$$
\rho_{\{1,2\}} = \big\{\{(1,1),(2,1),(2,2),(2,3)\},
\{(1,2),(1,3),(1,4),(2,4)\}\big\}, 
$$
% \medskip
and the partition \eqref{fjklds1} is
illustrated in Figure~\ref{fig:diagram1-3}-$b)$ with
 $b_1 = \{ 1,2\}$ and $b_2 = \{3,4,5\}$. 
% such that $b\times [r] \in \rho \vee \pi$. 

% \medskip

\begin{figure}[H]
\captionsetup[subfigure]{font=footnotesize}
\centering
\subcaptionbox{Diagram $\Gamma(\rho,\pi)$ and block $\rho_{\{1,2\}}$.}[.5\textwidth]{%
\begin{tikzpicture}[scale=0.9,hide labels]
\tikzstyle{VertexStyle}=[shape = circle, fill = blue!20, minimum size = 0pt, scale=0., text = white, hide labels]

\draw[black, thick] (0,0) rectangle (5,6);

\node[anchor=east,font=\small] at (0.8,5) {1};
\node[anchor=east,font=\small] at (0.8,4) {2};
\node[anchor=east,font=\small] at (0.8,3) {3};
\node[anchor=east,font=\small] at (0.8,2) {4};
\node[anchor=east,font=\small] at (0.8,1) {5};

\node[anchor=south,font=\small] at (1,0) {1};
\node[anchor=south,font=\small] at (2,0) {2};
\node[anchor=south,font=\small] at (3,0) {3};
\node[anchor=south,font=\small] at (4,0) {4};

\filldraw [gray] (1,1) circle (2pt);
\filldraw [gray] (2,1) circle (2pt);
\filldraw [gray] (3,1) circle (2pt);
\filldraw [gray] (4,1) circle (2pt);
\filldraw [gray] (1,2) circle (2pt);
\filldraw [gray] (2,2) circle (2pt);
\filldraw [gray] (3,2) circle (2pt);
\filldraw [gray] (4,2) circle (2pt);
\filldraw [gray] (1,3) circle (2pt);
\filldraw [gray] (2,3) circle (2pt);
\filldraw [gray] (3,3) circle (2pt);
\filldraw [gray] (4,3) circle (2pt);
\filldraw [gray] (2,3) circle (2pt);
\filldraw [gray] (1,4) circle (2pt);
\filldraw [gray] (2,4) circle (2pt);
\filldraw [gray] (3,4) circle (2pt);
\filldraw [gray] (4,4) circle (2pt);
\filldraw [gray] (1,5) circle (2pt);
\filldraw [gray] (2,5) circle (2pt);
\filldraw [gray] (3,5) circle (2pt);
\filldraw [gray] (4,5) circle (2pt);

\draw[very thick] (1,5) -- (1,4) -- (2,4) -- (3,4);
\draw[very thick] (2,5) -- (3,5) -- (4,5) -- (4,4);

\draw[very thick] (1,2) -- (1,1);
\draw[very thick] (2,3) -- (2,2);
\draw[very thick] (2,1) -- (3,1) -- (4,1);
\draw[very thick] (3,2) -- (4,3);

\node (1) [label=above:{}] at (1,5) {};
\node (2) [label=above:{}] at (2,5) {};
\node (3) [label=above:{}] at (3,5) {};
\node (4) [label=above:{}] at (4,5) {};

\node (5) [label=above:{}] at (1,4) {};
\node (6) [label=above:{}] at (2,4) {};
\node (7) [label=above:{}] at (3,4) {};
\node (8) [label=above:{}] at (4,4) {};

% \Vertex[x=1, y=5]{1}
% \Vertex[x=2, y=5]{2}
% \Vertex[x=3, y=5]{3}
% \Vertex[x=4, y=5]{4}

% \Vertex[x=1, y=4]{5}
% \Vertex[x=2, y=4]{6}
% \Vertex[x=3, y=4]{7}
% \Vertex[x=4, y=4]{8}

\draw[very thick,blue] \convexpath{1,4,8,5}{.2cm};

\draw[blue,line width=1mm, ->] node[font=\fontsize{12}{0}\selectfont, right=of 2, right=0.5cm, below=-0.9cm] {$~~~~~~\rho_{\{1,2\}}$};

% \foreach \i in {1,...,5}
%          {
% \draw[thick,dash dot,blue] (1,\i) .. controls (1.5,\i+.5) .. (2,\i);
% \draw[thick,dash dot,blue] (3,\i) .. controls (3.5,\i+.5) .. (4,\i);
% \draw[thick,dash dot,blue] (2,\i) .. controls (3,\i-.5) .. (4,\i);
% } 
\end{tikzpicture}}%
\subcaptionbox{Splitting $\{\rho_{b_1},\rho_{b_2}\}$
  of $\rho$ according to $\rho_G$.}[.5\textwidth]{
\begin{tikzpicture}[scale=0.9] 
\draw[black, thick] (0,0) rectangle (5,6);

\node[anchor=east,font=\small] at (0.8,5) {1};
\node[anchor=east,font=\small] at (0.8,4) {2};
\node[anchor=east,font=\small] at (0.8,3) {3};
\node[anchor=east,font=\small] at (0.8,2) {4};
\node[anchor=east,font=\small] at (0.8,1) {5};

\node[anchor=south,font=\small] at (1,0) {1};
\node[anchor=south,font=\small] at (2,0) {2};
\node[anchor=south,font=\small] at (3,0) {3};
\node[anchor=south,font=\small] at (4,0) {4};

\filldraw [gray] (1,1) circle (2pt);
\filldraw [gray] (2,1) circle (2pt);
\filldraw [gray] (3,1) circle (2pt);
\filldraw [gray] (4,1) circle (2pt);
\filldraw [gray] (1,2) circle (2pt);
\filldraw [gray] (2,2) circle (2pt);
\filldraw [gray] (3,2) circle (2pt);
\filldraw [gray] (4,2) circle (2pt);
\filldraw [gray] (1,3) circle (2pt);
\filldraw [gray] (2,3) circle (2pt);
\filldraw [gray] (3,3) circle (2pt);
\filldraw [gray] (4,3) circle (2pt);
\filldraw [gray] (2,3) circle (2pt);
\filldraw [gray] (1,4) circle (2pt);
\filldraw [gray] (2,4) circle (2pt);
\filldraw [gray] (3,4) circle (2pt);
\filldraw [gray] (4,4) circle (2pt);
\filldraw [gray] (1,5) circle (2pt);
\filldraw [gray] (2,5) circle (2pt);
\filldraw [gray] (3,5) circle (2pt);
\filldraw [gray] (4,5) circle (2pt);

\draw[very thick] (1,5) -- (1,4) -- (2,4) -- (3,4);
\draw[very thick] (2,5) -- (3,5) -- (4,5) -- (4,4);

\draw[very thick] (1,2) -- (1,1);
\draw[very thick] (2,3) -- (2,2);
\draw[very thick] (2,1) -- (3,1) -- (4,1);
\draw[very thick] (3,2) -- (4,3);

\draw[thick,dash dot,purple] (1,5) .. controls (1.5,4.5) .. (2,5);
%\draw[thick,dash dot,blue] (2,4) .. controls (3,3.5) .. (4,4);
%\draw[thick,dash dot,blue] (3,4) .. controls (3.5,4.5) .. (4,4);

\draw[thick,dash dot,purple] (1,3) .. controls (1.5,2.5) .. (2,3);
\draw[thick,dash dot,purple] (2,3) .. controls (3,2.5) .. (4,3);
\draw[thick,dash dot,purple] (3,3) .. controls (3.5,3.1) .. (4,3);

\draw[thick,dash dot,purple] (1,2) .. controls (1.5,2.5) .. (2,2);
\draw[thick,dash dot,purple] (2,2) .. controls (3,1.5) .. (4,2);
\draw[thick,dash dot,purple] (3,2) .. controls (3.5,2.5) .. (4,2);

\draw[thick,dash dot,purple] (1,1) .. controls (1.5,1.5) .. (2,1);

\node (1) [label=above:{}] at (1,5) {};
\node (2) [label=above:{}] at (2,5) {};
\node (3) [label=above:{}] at (3,5) {};
\node (4) [label=above:{}] at (4,5) {};

\node (5) [label=above:{}] at (1,4) {};
\node (6) [label=above:{}] at (2,4) {};
\node (7) [label=above:{}] at (3,4) {};
\node (8) [label=above:{}] at (4,4) {};

\draw[very thick,blue] \convexpath{1,4,8,5}{.2cm};

\draw[blue,line width=1mm, ->] node[font=\fontsize{12}{0}\selectfont, right=of 4, left=-.5cm, below=0.2cm] {$~~~~~~~\rho_{b_1}$};

\node (9) [label=above:{}] at (1,3) {};
\node (10) [label=above:{}] at (2,3) {};
\node (11) [label=above:{}] at (3,3) {};
\node (12) [label=above:{}] at (4,3) {};

\node (13) [label=above:{}] at (1,2) {};
\node (14) [label=above:{}] at (2,2) {};
\node (15) [label=above:{}] at (3,2) {};
\node (16) [label=above:{}] at (4,2) {};

\node (17) [label=above:{}] at (1,1) {};
\node (18) [label=above:{}] at (2,1) {};
\node (19) [label=above:{}] at (3,1) {};
\node (20) [label=above:{}] at (4,1) {};

\draw[very thick,blue] \convexpath{9,12,20,17}{.2cm};

\draw[blue,line width=1mm, ->] node[font=\fontsize{12}{0}\selectfont, right=of 12, left=-.5cm, below=0.7cm] {$~~~~~~~\rho_{b_2}$};

\end{tikzpicture}}%
\caption{Splitting of the partition $\rho$ with $\rho \vee \pi = \{\pi_1\cup\pi_2, \pi_3\cup\pi_4\cup\pi_5\}$ and $n=5$, $r=4$.}
\label{fig:diagram1-3}
\end{figure}

\vspace{-0.4cm}
  
% and 
% $\rho_b := \{ c\in \rho \ : \ c \subset b\times[r] \}$ % ( b \times [r] )$
% is connected in the diagram $\rho_G$. 
\begin{definition}
 Let $n , r \geq 1$. 
 Given $\sigma \in \Pi ([n] )$ a partition of $[n]$, 
 we let $\Pi_\sigma ( [n] \times [r])$ denote the set of
 partitions $\rho$ of $[n] \times [r]$ such that
 $$
 \rho\vee\pi= \{ b \times [r] \ : \ b \in \sigma \}, 
 $$
 and % for any $\eta \subset [n]$
 we partition $\Pi ([n] \times [r])$ as
\begin{equation}
\label{partition} 
 \Pi ([n] \times [r])
 = \bigcup_{\sigma \in \Pi ([n] )} \Pi_\sigma ([n] \times [r]). 
\end{equation} 
\end{definition}
 We note that given $\eta \subset [n]$,
the set $\Pi_{\widehat{1}} ( \eta \times [r])$ consists of the partitions
$\rho$ of $\eta \times [r]$ for which the graph $\rho_G$
is connected, as in Figure~\ref{fig:diagram2}. 

% \medskip

\begin{figure}[H]
\captionsetup[subfigure]{font=footnotesize}
\centering
\subcaptionbox{Diagram $\Gamma(\rho,\pi)$ and multigraph $\widetilde{\rho}_G$ in blue.}[.5\textwidth]{%
\begin{tikzpicture}[scale=0.9] 
\draw[black, thick] (0,0) rectangle (5,6);

\node[anchor=east,font=\small] at (0.8,5) {1};
\node[anchor=east,font=\small] at (0.8,4) {2};
\node[anchor=east,font=\small] at (0.8,3) {3};
\node[anchor=east,font=\small] at (0.8,2) {4};
\node[anchor=east,font=\small] at (0.8,1) {5};

\node[anchor=south,font=\small] at (1,0) {1};
\node[anchor=south,font=\small] at (2,0) {2};
\node[anchor=south,font=\small] at (3,0) {3};
\node[anchor=south,font=\small] at (4,0) {4};

\filldraw [gray] (1,1) circle (2pt);
\filldraw [gray] (2,1) circle (2pt);
\filldraw [gray] (3,1) circle (2pt);
\filldraw [gray] (4,1) circle (2pt);
\filldraw [gray] (1,2) circle (2pt);
\filldraw [gray] (2,2) circle (2pt);
\filldraw [gray] (3,2) circle (2pt);
\filldraw [gray] (4,2) circle (2pt);
\filldraw [gray] (1,3) circle (2pt);
\filldraw [gray] (2,3) circle (2pt);
\filldraw [gray] (3,3) circle (2pt);
\filldraw [gray] (4,3) circle (2pt);
\filldraw [gray] (2,3) circle (2pt);
\filldraw [gray] (1,4) circle (2pt);
\filldraw [gray] (2,4) circle (2pt);
\filldraw [gray] (3,4) circle (2pt);
\filldraw [gray] (4,4) circle (2pt);
\filldraw [gray] (1,5) circle (2pt);
\filldraw [gray] (2,5) circle (2pt);
\filldraw [gray] (3,5) circle (2pt);
\filldraw [gray] (4,5) circle (2pt);

\draw[very thick] (1,5) -- (1,4); 
\draw[very thick] (3,5) -- (4,4);

\draw[very thick] (1,2) -- (1,1);
\draw[very thick] (2,2) -- (2,4);
\draw[very thick] (2,1) -- (3,2) -- (4,3) -- (3,4);

\foreach \i in {1,...,5}
         {
           \draw[thick,dash dot,blue] (1,\i) .. controls (1.5,\i+.5) .. (2,\i);
           \draw[thick,dash dot,blue] (2,\i) .. controls (2.5,\i+.5) .. (3,\i);
           \draw[thick,dash dot,blue] (3,\i) .. controls (3.5,\i+.5) .. (4,\i);
           \draw[thick,dash dot,blue] (4,\i) .. controls (2.5,\i-.5) .. (1,\i);
         }
         
\end{tikzpicture}}%
\subcaptionbox{Diagram $\Gamma(\rho ,\pi)$ and graph $\rho_G$ in red.}[.5\textwidth]{
\begin{tikzpicture}[scale=0.9] 
\draw[black, thick] (0,0) rectangle (5,6);

\node[anchor=east,font=\small] at (0.8,5) {1};
\node[anchor=east,font=\small] at (0.8,4) {2};
\node[anchor=east,font=\small] at (0.8,3) {3};
\node[anchor=east,font=\small] at (0.8,2) {4};
\node[anchor=east,font=\small] at (0.8,1) {5};

\node[anchor=south,font=\small] at (1,0) {1};
\node[anchor=south,font=\small] at (2,0) {2};
\node[anchor=south,font=\small] at (3,0) {3};
\node[anchor=south,font=\small] at (4,0) {4};

\filldraw [gray] (1,1) circle (2pt);
\filldraw [gray] (2,1) circle (2pt);
\filldraw [gray] (3,1) circle (2pt);
\filldraw [gray] (4,1) circle (2pt);
\filldraw [gray] (1,2) circle (2pt);
\filldraw [gray] (2,2) circle (2pt);
\filldraw [gray] (3,2) circle (2pt);
\filldraw [gray] (4,2) circle (2pt);
\filldraw [gray] (1,3) circle (2pt);
\filldraw [gray] (2,3) circle (2pt);
\filldraw [gray] (3,3) circle (2pt);
\filldraw [gray] (4,3) circle (2pt);
\filldraw [gray] (2,3) circle (2pt);
\filldraw [gray] (1,4) circle (2pt);
\filldraw [gray] (2,4) circle (2pt);
\filldraw [gray] (3,4) circle (2pt);
\filldraw [gray] (4,4) circle (2pt);
\filldraw [gray] (1,5) circle (2pt);
\filldraw [gray] (2,5) circle (2pt);
\filldraw [gray] (3,5) circle (2pt);
\filldraw [gray] (4,5) circle (2pt);

\draw[very thick] (1,5) -- (1,4); 
\draw[very thick] (3,5) -- (4,4);

\draw[very thick] (1,2) -- (1,1);
\draw[very thick] (2,2) -- (2,4);
\draw[very thick] (2,1) -- (3,2) -- (4,3) -- (3,4);

\foreach \i in {1,...,3}
         {
           \draw[thick,dash dot,purple] (1,\i) .. controls (1.5,\i+.5) .. (2,\i);
           \draw[thick,dash dot,purple] (2,\i) .. controls (2.5,\i+.5) .. (3,\i);
           \draw[thick,dash dot,purple] (3,\i) .. controls (3.5,\i+.5) .. (4,\i);
           \draw[thick,dash dot,purple] (4,\i) .. controls (2.5,\i-.5) .. (1,\i);
         }

           \draw[thick,dash dot,purple] (1,4) .. controls (1.5,4+.5) .. (2,4);
%           \draw[thick,dash dot,blue] (2,2) .. controls (2.5,2+.5) .. (3,2);
           \draw[thick,dash dot,purple] (3,4) .. controls (3.5,4+.5) .. (4,4);
           \draw[thick,dash dot,purple] (4,4) .. controls (2.5,4-.5) .. (1,4);

\foreach \i in {5,...,5}
         {
           \draw[thick,dash dot,purple] (1,\i) .. controls (1.5,\i+.5) .. (2,\i);
           \draw[thick,dash dot,purple] (2,\i) .. controls (2.5,\i+.5) .. (3,\i);
           \draw[thick,dash dot,purple] (3,\i) .. controls (3.5,\i+.5) .. (4,\i);
           \draw[thick,dash dot,purple] (4,\i) .. controls (2.5,\i-.5) .. (1,\i);
         }
\end{tikzpicture}}%
\caption{Connected non-flat partition diagram with $G$ a cycle graph and $n=5$, $r=4$.}
\label{fig:diagram2}
\end{figure}

\vskip-0.3cm

\noindent
The following lemma will be useful when applying induction
arguments on connected set partitions in $\Pi_{\widehat{1}}([n]\times [r])$.
% \textcolor{cyan}
\begin{lemma}
  \label{restrict-partition}
  Let $n\geq 2$.
  For any connected partition 
  $\rho\in \Pi_{\widehat{1}}( [n]\times [r])$
  there exists $i\in \{1,\dots,n \}$
  such that the set partition 
  %    \label{recur-1} \rho^{(i)}:=
  $\{b\backslash\pi_i:b\in\rho\}$ 
  of $\{1,\dots,i-1,i+1,\dots,n \}\times [r]$
  is connected.
\end{lemma}
\begin{proof}
 Let $\rho\in \Pi_{\widehat{1}}( [n]\times [r])$. 
% be a partition such that the diagram $\Gamma(\rho,\pi )$ is connected.
 We consider the connected undirected graph
 $\eufrak{g}$ on $[n]$ in which two vertices $i,j\in [n]$
 are connected if and only
 if there exists a block $b\in \rho$ such that $\pi_i \cap b \not= \emptyset$
 and $\pi_j \cap b \not= \emptyset$,
 see Figure~\ref{fig:diagram0-11}-$a)$ for an example with $n=5$.  
 \vspace{-0.4cm}
\begin{figure}[H]
\captionsetup[subfigure]{font=footnotesize}
\centering
\scalebox{0.75}{
\subcaptionbox{Diagram $\Gamma(\rho,\pi)$ and graph $\eufrak{g}$.}[.48\textwidth]{%
\begin{tikzpicture}[scale=0.9] 
\draw[black, thick] (0,0) rectangle (5,6);

\node[anchor=east,draw, circle, inner sep=0pt, minimum size=11pt,font=\small] at (0.7,5) {1};
\node[anchor=east,draw, circle, inner sep=0pt, minimum size=11pt,font=\small] at (0.7,4) {2};
\node[anchor=east,draw, circle, inner sep=0pt, minimum size=11pt,font=\small] at (0.7,3) {3};
\node[anchor=east,draw, circle, inner sep=0pt, minimum size=11pt,font=\small] at (0.7,2) {4};
\node[anchor=east,draw, circle, inner sep=0pt, minimum size=11pt,font=\small] at (0.7,1) {5};

\node[anchor=south,font=\small] at (1,0) {1};
\node[anchor=south,font=\small] at (2,0) {2};
\node[anchor=south,font=\small] at (3,0) {3};
\node[anchor=south,font=\small] at (4,0) {4};

\filldraw [gray] (1,1) circle (2pt);
\filldraw [gray] (2,1) circle (2pt);
\filldraw [gray] (3,1) circle (2pt);
\filldraw [gray] (4,1) circle (2pt);
\filldraw [gray] (1,2) circle (2pt);
\filldraw [gray] (2,2) circle (2pt);
\filldraw [gray] (3,2) circle (2pt);
\filldraw [gray] (4,2) circle (2pt);
\filldraw [gray] (1,3) circle (2pt);
\filldraw [gray] (2,3) circle (2pt);
\filldraw [gray] (3,3) circle (2pt);
\filldraw [gray] (4,3) circle (2pt);
\filldraw [gray] (2,3) circle (2pt);
\filldraw [gray] (1,4) circle (2pt);
\filldraw [gray] (2,4) circle (2pt);
\filldraw [gray] (3,4) circle (2pt);
\filldraw [gray] (4,4) circle (2pt);
\filldraw [gray] (1,5) circle (2pt);
\filldraw [gray] (2,5) circle (2pt);
\filldraw [gray] (3,5) circle (2pt);
\filldraw [gray] (4,5) circle (2pt);

\draw[very thick] (1,5) -- (1,4); 
\draw[very thick] (3,5) -- (4,4);

\draw[very thick] (1,2) -- (1,1);
\draw[very thick] (2,2) -- (2,4);
\draw[very thick] (2,1) -- (3,2) -- (4,3) -- (3,4);

\draw[very thick,purple] (0.3,5) .. controls (0.2,4.5) .. (0.3,4);
\draw[very thick,purple] (0.3,4) .. controls (0.2,3.5) .. (0.3,3);
\draw[very thick,purple] (0.3,3) .. controls (0.2,2.5) .. (0.3,2);
\draw[very thick,purple] (0.3,2) .. controls (0.2,1.5) .. (0.3,1);
\draw[very thick,purple] (0.7,4) .. controls (0.85,3) .. (0.7,2);
\draw[very thick,purple] (0.7,4) .. controls (0.95,3) .. (0.7,1);

\end{tikzpicture}%
}
}
\scalebox{0.75}{
\subcaptionbox{Diagram $\Gamma(\rho,\pi)$ and spanning tree $\widebar{\eufrak{g}}$.}[.5\textwidth]{%
\begin{tikzpicture}[scale=0.9] 
\draw[black, thick] (0,0) rectangle (5,6);

\node[anchor=east,draw, circle, inner sep=0pt, minimum size=11pt, font=\small] at (0.7,5) {1};
\node[anchor=east,draw, circle, inner sep=0pt, minimum size=11pt,font=\small] at (0.7,4) {2};
\node[anchor=east,draw, circle, inner sep=0pt, minimum size=11pt,font=\small] at (0.7,3) {3};
\node[anchor=east,draw, circle, inner sep=0pt, minimum size=11pt,font=\small] at (0.7,2) {4};
\node[anchor=east,draw, circle, inner sep=0pt, minimum size=11pt,font=\small] at (0.7,1) {5};

\node[anchor=south,font=\small] at (1,0) {1};
\node[anchor=south,font=\small] at (2,0) {2};
\node[anchor=south,font=\small] at (3,0) {3};
\node[anchor=south,font=\small] at (4,0) {4};

\filldraw [gray] (1,1) circle (2pt);
\filldraw [gray] (2,1) circle (2pt);
\filldraw [gray] (3,1) circle (2pt);
% \draw [very thick] (3,1) circle (4.5pt);
\filldraw [gray] (4,1) circle (2pt);
% \draw [very thick] (4,1) circle (4.5pt);
\filldraw [gray] (1,2) circle (2pt);
\filldraw [gray] (2,2) circle (2pt);
\filldraw [gray] (3,2) circle (2pt);
\filldraw [gray] (4,2) circle (2pt);
% \draw [very thick] (4,2) circle (4.5pt);
\filldraw [gray] (1,3) circle (2pt);
% \draw [very thick] (1,3) circle (4.5pt);
\filldraw [gray] (2,3) circle (2pt);
\filldraw [gray] (3,3) circle (2pt);
% \draw [very thick] (3,3) circle (4.5pt);
\filldraw [gray] (4,3) circle (2pt);
\filldraw [gray] (2,3) circle (2pt);
\filldraw [gray] (1,4) circle (2pt);
\filldraw [gray] (2,4) circle (2pt);
\filldraw [gray] (3,4) circle (2pt);
\filldraw [gray] (4,4) circle (2pt);
\filldraw [gray] (1,5) circle (2pt);
\filldraw [gray] (2,5) circle (2pt);
% \draw [very thick] (2,5) circle (4.5pt);
\filldraw [gray] (3,5) circle (2pt);
\filldraw [gray] (4,5) circle (2pt);
% \draw [very thick] (4,5) circle (4.5pt);

\draw[very thick] (1,5) -- (1,4); 
\draw[very thick] (3,5) -- (4,4);

\draw[very thick] (1,2) -- (1,1);
\draw[very thick] (2,2) -- (2,4);
\draw[very thick] (2,1) -- (3,2) -- (4,3) -- (3,4);

\draw[very thick,blue] (0.3,5) .. controls (0.2,4.5) .. (0.3,4);
% \draw[thick,purple] (0.3,4) .. controls (0.2,3.5) .. (0.3,3);
\draw[very thick,blue] (0.3,3) .. controls (0.2,2.5) .. (0.3,2);
\draw[very thick,blue] (0.3,2) .. controls (0.2,1.5) .. (0.3,1);
\draw[very thick,blue] (0.7,4) .. controls (0.85,3) .. (0.7,2);
% \draw[thick,purple] (0.7,4) .. controls (0.95,3) .. (0.7,1);

\end{tikzpicture}%
}
}
\caption{Example of graph $\eufrak{g}$ and its spanning tree subgraph.}
\label{fig:diagram0-11}
\end{figure}

\vspace{-.4cm}
  
\noindent
By e.g. Theorem~4.2.3 in \cite{balakrishnan},
 $\eufrak{g}$ contains a spanning
 tree $\widebar{\eufrak{g}}$, 
 as shown in Figure~\ref{fig:diagram0-11}-$b)$. 
 Let $i\in [n]$ be a leaf in the tree $\widebar{\eufrak{g}}$. 
%  (e.g. the leaf with lowest or highest index)
 If the partition 
 ${\rho}^{(i)}:=\{ b \setminus \pi_i \ \!  : \ \!  b\in \rho \}$
 of $([n] \setminus \{i\} ) \times [r]$ 
 had more than one connected component, then, for $\rho$ to be connected,
 $\pi_i$ would have to connect to all such components, 
 hence the vertex $i$ would be adjacent to
 more than one vertex in $\widebar{\eufrak{g}}$,
 which is not the case.
\end{proof}
 In what follows, we say that a partition $\rho$ of $[n]\times [r]$ 
 is {\em non-flat} if 
 its diagram $\Gamma(\rho,\pi )$ is non-flat, i.e. 
 if $\rho \wedge \pi = \widehat{0}$,
 see Chapter~4 of \cite{peccatitaqqu} and Figure~\ref{fig:diagram2}. 

\begin{definition} 
  \noindent
 Given $n,r\geq 1$, we denote by 
 $$
 \mathcal{C} (n,r) :=\{\rho\in\Pi_{\widehat{1}} ([n]\times[r]) \ : \ \rho\wedge \pi=\widehat{0} \}
  $$
 the set of connected non-flat partitions of $[n]\times[r]$. 
\end{definition}
\noindent 
We will also consider the following set of 
 connected non-flat partitions which have a maximal
 number of blocks. 
 \begin{definition} 
  \noindent
 Given $n\geq 1$ and $r\geq 2$,
we denote by
$$
 \mathcal{M}(n,r):=\{\rho\in \mathcal{C} (n,r)  \ : \ |\rho|= 1 + (r-1)n \}
  $$
 the set of maximal connected non-flat partitions of $[n]\times[r]$.
\end{definition}

\noindent
 The bound in part $(a)$ of the next lemma is consistent with
 (6.2) in Proposition~6.1 of \cite{schulte-thaele},
 which shows that the power $r$ of $n!$ cannot be improved in \eqref{coeff-0}. 
\begin{lemma}
  \label{fjkldsf-l}
  \noindent
  $a)$ The cardinality of the set \ $\mathcal{C} (n,r)$
 of connected non-flat partitions of $[n]\times[r]$ satisfies 
 \begin{equation}
   \label{coeff-0}
  |\mathcal{C} (n,r) | \leq n!^r r!^{n-1}, 
  \qquad n,r \geq 1. 
\end{equation}
\noindent
$b)$ 
 The cardinality of the set 
$ \mathcal{M}(n,r)$
 of maximal connected non-flat partitions of $[n]\times[r]$ satisfies 
\begin{equation}\label{coeff-10}
  |\mathcal{M}(n,r)|=r^{n-1}\prod_{i=1}^{n-1}(1+(r-1)i),
  \qquad n,r\geq 1, 
\end{equation}
 with the bounds 
\begin{equation}\label{coeff-1}
    ( (r-1)r )^{n-1}(n-1)!\le
    |\mathcal{M}(n,r)|
     \leq ( (r-1)r )^{n-1}n!, \quad n\geq 1, \ r\geq 2. 
\end{equation}
\end{lemma}
\begin{proof}
\noindent
 $a)$ We clearly have $|\mathcal{C} (1,r) | =1$ for all $r\geq 1$. 
 As a consequence of Lemma~\ref{restrict-partition}, 
 any connected non-flat partition $\rho \in \mathcal{C}(n+1,r)$ can be
 obtained from a connected non-flat partition in $\mathcal{C}(n,r)$ in at most
 $(n+1)^r r!$ ways, by connecting each of the $r$ new points
 in at most $n+1$ possible ways (including non-connection),
 and multiplying by the number $r!$ of possible permutations of $\pi_i$. 
 This yields the induction inequality 
 $$
 |\mathcal{C} (n+1,r) | \leq (n+1)^r r! |\mathcal{C} (n,r) |,
 $$
 from which we conclude that \eqref{coeff-0} holds. 
 
 \medskip 

 \noindent
 $b)$ 
 We clearly have $|\mathcal{M}(1,r)|=1$ for all $r\geq 1$. 
 Next, each maximal connected non-flat partition 
 $\rho\in\mathcal{M}(n+1,r)$ can be
   obtained by choosing 
   one of $r$ elements of $\{(n+1,1),\dots,(n+1,r)\}$, 
   and connecting them in $1+(r-1)n$ ways 
   to any partition in $\mathcal{M}(n,r)$, $n\geq 1$.
   This implies the recursion formula 
   $$|\mathcal{M}(n+1,r)|=r\times (1+(r-1)n)|\mathcal{M}(n,r)|,
   $$
 which yields \eqref{coeff-10}.
\end{proof} 

\section{Virtual cumulants} 
\label{s3}
\noindent
%\begin{definition}Let $r\ge1$ be fixed. A functional $F:\cup_{n\ge1}P(\Delta_{n,r})\to\R$ is said to own the {\it splitting property} if for every $n\ge1$, $\forall~\rho\in P(\Delta_{n,r})$, there exists a group of partitions (not necessarily unique)
%$$\eta^i\in P(\Delta_{n_i,r}),~~~~i=1,\dots,\ell,$$with $n_1+\dots+n_{\ell}=n$, such that%\begin{equation}\label{splitting1}
%F(\rho)=\prod_{i=1}^\ell F(\eta^i).
%\end{equation}
%And we say that $F$ allows the connectedness factorisation if (\Ref{splitting1}) holds for $\rho^i,~i=1,\dots,\ell$ where $\Gamma(\rho^1,\pi^1),\dots,\Gamma(\rho^\ell,\pi^\ell)$ are all connected sub-diagrams in $\Gamma(\rho,\pi)$. 
%\end{definition}
The following definition
uses the concept of independence of a virtual field
with respect to graph connectedness,
see Relation~(17) in \cite[p.~34]{MalyshevMinlos91}. 
\begin{definition}
  Let $n,r\geq 1$.
  We say that a mapping $F$ defined on partitions of
  $[n]\times [r]$
  % $\rho \in \bigcup_{\eta \subset [n]} \Pi ( \eta \times [r])$ 
  admits the {\it connectedness factorization} property
  if it decomposes according to the partition \eqref{fjklds1} as 
  \begin{equation}
    \label{dia-factoriz}
  F ( \rho ) = \prod_{b \times [r] \in \rho \vee \pi } F ( \rho_b ),  
  \qquad
   \rho \in \Pi ([n]\times [r]). 
\end{equation}
% for every partition $\rho$ of $[n]\times [r]$. 
% where $g_i=\Gamma(\pi^i,\rho^i)$ is given by (\Ref{dimreduce}), (\Ref{decom2}) and (\Ref{decom3}).
\end{definition} 
In what follows,
given $F$ a mapping defined on the partitions of
  $[n]\times [r]$, 
% $\bigcup_{\eta \subset [n]} \Pi ( \eta \times [r])$, 
  we will use the M\"obius transform
  $\widehat{F}$ of $F$, 
  defined as
  % $f:2^{[n]}\to \R$ be a set function, given by
\begin{equation}
\nonumber % \label{genfun-1}
  \widehat{F}( \eta ):=\sum_{\rho \in \Pi ( \eta \times [r])} F(\rho ),
  \qquad
  \eta \subset [n],
\end{equation}
with $\widehat{F}(\emptyset):=0$,
see \cite{rota1964} % Proposition~2 of \cite{rota1964} and 
% Chapter~5 of \cite{rotabk},
 and \S~2.5 of \cite{peccatitaqqu}. 
 % , where $2^{[n]}$ stands for the power set of $[n]$.
 We refer to \cite[p.~33]{MalyshevMinlos91} for the following definition.
\begin{definition}
  Let $n,r\geq 1$.
  The virtual cumulant $G$ of 
 a mapping $F$ on $\bigcup_{\eta \subset [n]} \Pi ( \eta \times [r])$ 
 is defined by letting  
 $C_F ( \eta ):=\widehat{F}( \eta )$ when $| \eta |=1$, and
 then recursively by 
\begin{equation}
  \label{fjkldsf}
  C_F ( \eta ):=\widehat{F}( \eta )-\sum_{
    { \sigma \in \Pi (\eta )
    \atop 
    |\sigma|\geq 2
    }
  }
  \prod_{b \in \sigma } C_F ( b ), \qquad
  \eta \subset [n], \quad |\eta |\geq 2. 
\end{equation}
\end{definition} 
% In the particular case $r=1$, we note that when $(X_1,\ldots , X_n)$ is a sequence of random variables, letting $$ F(\rho) := \E\left[ \prod_{b \in \rho} \prod_{i \in b} X_i \right] = \E\left[ \prod_{i=1}^n X_i \right], $$ 
Relation~\eqref{fjkldsf} also implies the relation 
 \begin{equation}
   \label{jlkdf3} 
 C_F ( \eta )=
 \sum_{
   \sigma \in \Pi ( \eta )
 }
 (-1)^{|\sigma |-1} (|\sigma |-1)! \prod_{b\in \sigma} \widehat{F} ( b ), 
% = \sum_{   \sigma \in \Pi ( \eta ) } (-1)^{|\sigma |-1} (|\sigma |-1)!  \prod_{b \in \sigma} \E\left[ \prod_{i \in b} X_i \right],
\end{equation}
 see Relation~(16') page~33 of \cite{MalyshevMinlos91},
 which is also the classical cumulant-moment relationship,
 see e.g. Corollary~3.2.2 in \cite{peccatitaqqu}. 
 % coincides with the actual joint cumulant of $(X_i)_{i\in \eta}$, $\eta \subset [n]$. \medskip
\noindent  
The following proposition is an extension of
the classical Lemma~2 in \cite[p.~34]{MalyshevMinlos91},
see also Lemma~3.1 in \cite{khorunzhiy}.
% Recall that given $\eta \subset [n]$, $\Pi_{\widehat{1}} ( \eta \times [r])$ consists of the partitions $\rho$ of $\eta \times [r]$ for which the diagram $\rho_G$ is connected. 
\begin{prop}
\label{mainthm-1}
Let $n,r\geq 1$. 
Let $F$ be a mapping defined on $\bigcup_{\eta \subset [n]}
 \Pi ( \eta \times [r])$ and admitting the {connectedness factorization} property
 \eqref{dia-factoriz}.
 Then, for $\eta \subset [n]$ with $\eta \ne\emptyset$, 
 the virtual cumulant of $F$ is given by the sum 
\begin{equation}
\label{keylma-2}
C_F (\eta ) = \sum_{\sigma \in \Pi_{\widehat{1}} ( \eta \times [r] )
  \atop {\rm (connected)}}
F(\sigma )
\end{equation}
over connected partitions on $\eta \times [r]$. 
\end{prop} 
\begin{proof}
The claim is true when $|\eta |=1$. %Let $[n]$ be the set of vertices of some graph $G$. 
Assume that it is true for all $\eta \subset [n]$ for some $n \geq 1$, and 
% , by induction principle, it remains to show that the claim keeps true for any $A$ with $|A|=m+1$.
 let $\eta$ be such that $|\eta |=n+1$.
 By \eqref{partition} and \eqref{dia-factoriz}, we have 
\begin{eqnarray*}
  \widehat{F}(\eta )
  &=&
  \sum_{\rho \in \Pi (\eta \times [r])} F(\rho )
% \qquad \qquad \mathrm{(by~~(\Ref{genfun-1}))}
  \\
  &=&
  \sum_{\sigma \in \Pi (\eta )}
  \sum_{\rho \in \Pi_\sigma (\eta \times [r])} F(\rho )
% \qquad \qquad \mathrm{(by~~(\Ref{genfun-1}))}
  \\
  &=&
  \sum_{\sigma \in \Pi (\eta )}
  \sum_{\rho \in \Pi_\sigma (\eta \times [r])}
  \prod_{b \in \sigma} 
  F(\rho_b ) 
% \qquad \qquad \mathrm{(by~~(\Ref{genfun-1}))}
 \\
  &=&
  \sum_{\sigma \in \Pi (\eta )}
  \prod_{b \in \sigma } 
  \sum_{\rho \in \Pi_{\widehat{1}} (b \times [r])
    \atop {\rm (connected)}
  } 
  F(\rho )
% \qquad \qquad \mathrm{(by~~(\Ref{genfun-1}))}
% \\ &=& \sum_{\rho \in \Pi_{\widehat{1}} ( \eta \times [r]) } F(\rho ) + \sum_{\sigma \in \Pi (\eta ) \atop |\sigma |\geq 2} \prod_{\mu \in \sigma } \sum_{\rho \in \Pi_{\widehat{1}} ( \mu \times [r] ) } F(\rho )
% \qquad \qquad \mathrm{(by~~(\Ref{genfun-1}))}
 \\
  &=&
 \sum_{\rho \in \Pi_{\widehat{1}} ( \eta \times [r] )
   \atop {\rm (connected)}
 }
  F(\rho )
  +
  \sum_{\sigma \in \Pi (\eta ) \atop |\sigma |\geq 2}
  \prod_{b \in \sigma } 
  C_F (b ), 
% \qquad \qquad \mathrm{(by~~(\Ref{genfun-1}))}, 
\end{eqnarray*}
where the last equality follows from the induction hypothesis \eqref{keylma-2} 
when $| \eta |\leq n$.
 The proof is completed by subtracting the last term on both sides. 
\end{proof}
%\textcolor{red}{
%\begin{remark}\label{rem-cumulant1}
%The assertion can be regarded as an extension of the Lemma~2 in \cite[p.~34]{MalyshevMinlos91}, which is presented in a language of dependency graph approach.
%\end{remark}
%}
% In the particular case $r=1$, we note that when $(X_1,\ldots , X_n)$ is a sequence of independent random variables, the functional
% $$F(\rho) := \E\left[ \prod_{b \in \rho} \prod_{i \in b} X_i \right] = \prod_{i \in [n]} \E [ X_i ] $$ 
% satisfies the {connectedness factorization} property \eqref{dia-factoriz}, and Proposition~\ref{mainthm-1}  recovers the vanishing of the joint cumulants  of $(X_i)_{i\in \eta}$ when $|\eta |\geq 2$, as  the set $\Pi_{\widehat{1}} ( \eta \times [1] )$  of connected partition diagrams on $\eta \times [1]$ is empty in this case. 
\section{Cumulants of multiparameter stochastic integrals} 
\label{s4}
\noindent 
% We write the set of $r$ vertices of the connected graph $G$ as $V_G = [r]$, and we denote by $E_G \subset V_G\times V_G$ the set of its edges. 
 Consider a Poisson point process $\Xi$ on $\R^d$, $d \geq 1$, with intensity
measure $\Lambda$ on $\real^d$, % =\lambda\mu(\cdot)$, where $\mu$ is a probability measure on $\R^d$ having bounded density function with respect to the Lebesgue measure, and $\lambda>0$ is a constant. 
% Since $\Lambda$ is $\sigma$-finite and $\R^d$ is a Borel space,
 constructed on the space $$
 \Omega = \big\{
 \omega = \{ x_i \}_{i\in I} \subset \R^d \ : \
 \#( A \cap \omega ) < \infty 
 \mbox{ for all compact } A\in {\cal B} (\R^d) 
 \big\}
 $$
 of locally finite configurations on $\R^d$, whose elements 
 $\omega \in \Omega$ are identified with the Radon point measures 
 $\displaystyle \omega = \sum_{x\in \omega} \epsilon_x$, 
 where $\epsilon_x$ denotes the Dirac measure at $x\in \R^d$. 
 \textcolor{red}{As in} \cite[Corollary~6.5]{LastPenrose17}, almost every
 element $ \omega$ of $\Omega$ 
 can be represented as $\omega =\{V_i\}_{1\leq i\leq N}$,
 where $(V_i)_{i\geq 1}$ is a random sequence 
 in $\R^d$ and a $\N\cup\{\infty\}$-valued random variable $N$.
 % such that $\Xi$ can be almost surely represented as 

 \medskip

 In this section, using sums over partitions
 we express the moments of the multiparameter stochastic integral 
\begin{equation}
\label{e1}
 \sum_{V_1,\dots,V_r \in\Xi} \ u_G (V_1,\ldots , V_r ) 
 = \int_{(\real^d)^r} u_G (x_1,\ldots , x_r ) \omega (dx_1)\cdots \omega (dx_r), 
\end{equation} 
 where $u_G (x_1,\ldots , x_r )$ is a \textcolor{red}{bounded} measurable process of the form 
$$
 u_G (x_1,\ldots , x_r ) := \prod_{\{i,j\} \in E(G)} v_{i,j}(x_i,x_j), 
$$
% where $E(G) \subset [r]\times [r]$ is the set of edges of $G$, 
 and $v_{i,j}(x,y)$, $\{i,j\}\in E(G)$,
 are random processes $v(x,y)$
 independent of the underlying Poisson point process~$\Xi$. 
 The next proposition is a consequence of Proposition~2 in \cite{prkhp}, 
which relies on Proposition~3.1 of \cite{momentpoi} % Theorem~3.1 of \cite{flint}
and Lemma~2.1 of \cite{bogdan}. 
\begin{prop}
\label{p01-1-0} 
Let $n \geq 1$ and $r\geq 2$. 
The $n$-$th$ moment of the multiparameter stochastic integral
 \eqref{e1} is given by the summation 
\begin{equation} 
\label{nthexpectation0-0} 
\sum_{\rho \in \Pi ( [n] \times [r] )}
\int_{(\R^d )^{|\rho |}}
\E \left[
  \prod_{k=1}^n
\prod_{% (\eta_{k,i},\eta_{k,j}) \in E(\rho_G) \atop
\{i,j\}\in E(G_k)}
v\big(x^\rho_{k,i},x^\rho_{k,j}\big) % ^{m({\eta_{k,i},\eta_{k,j}})}
\right] 
   \ \! \prod_{\eta \in V(\rho_G)}
   \Lambda(dx_\eta ),
\end{equation}
 where we let $x_{k,l}^\rho:=x_\eta$ whenever $(k,l)\in \eta$,
 for $\rho \in \Pi([n]\times [r])$ and $\eta\in\rho$.
% where $m ({\eta_1,\eta_2})$ represents the multiplicity of the edge $(\eta_1,\eta_2)$ in the multigraph $\widetilde{\rho}_G$. 
% and $V(\rho_G)$ denotes the set of vertices of the graph $\rho_G$. 
\end{prop}
 The next proposition rewrites the product in \eqref{nthexpectation0-0} 
 as a product on the edges of the graph $\rho_G$ 
 similarly to Proposition~4 of \cite{prkhp} when $v(x,y)$ vanishes on
 the diagonal, and it generalizes
 Proposition~2.4 of \cite{jansen}
 from two-parameter Poisson stochastic
 integrals to multiparameter integrals of higher orders. 
\begin{prop}
\label{p01-1-4} 
Let $n \geq 1$, $r\geq 2$, and assume that the process $v(x,y)$ vanishes
on diagonals, i.e. $v(x,x) = 0$, $x\in \real^d$. 
Then, the $n$-$th$ moment of the multiparameter stochastic integral
 \eqref{e1} is given by the summation 
\begin{equation} 
\nonumber % \label{nthexpectation0-1} 
\sum_{\rho \in \Pi ( [n] \times [r] )
\atop {
    \rho \wedge \pi = \widehat{0}
    \atop
        {\rm (non-flat)}
}
}
\int_{(\R^d )^{|\rho |}}
  \prod_{\{\eta_1,\eta_2\} \in E(\rho_G)}
 \E \big[ v(x_{\eta_1},x_{\eta_2})^{m({\eta_1,\eta_2})} \big] 
   \ \! \prod_{\eta \in V(\rho_G)}
\Lambda(dx_\eta ), 
\end{equation} 
over connected non-flat partitions,
where $m ({\eta_1,\eta_2})$ represents the multiplicity of the edge
$(\eta_1,\eta_2)$ in the multigraph $\widetilde{\rho}_G$. 
% and $V(\rho_G)$ denotes the set of vertices of the graph $\rho_G$. 
\end{prop}
The next proposition is a consequence of Propositions~\ref{mainthm-1}
and \ref{p01-1-4},
 and it also extends Proposition~2.5 of \cite{jansen}
 from the two-parameter case to the multiparameter
 case. 
 Note that in our setting, the two-parameter case
 only applies to the edge counting. 
\begin{prop}
\label{p01-1} 
Let $n \geq 1$, $r \geq 2$, and assume that the process $v(x,y)$ vanishes
on diagonals, i.e. $v(x,x) = 0$, $x\in \real^d$. 
Then, the $n$-$th$ cumulant of the multiparameter stochastic integral
 \eqref{e1} is given by the summation 
\begin{equation}
  \label{nthexpectation0-2}
  % \kappa_n(N_G)=
  \sum_{\rho \in \Pi_{\widehat{1}} ( [n] \times [r])
    \atop
    {\rho \wedge \pi = \widehat{0}
    \atop {\rm (non-flat \ \! connected)}
    }
    }
    \int_{(\R^d )^{|\rho |}}
      \prod_{\{\eta_1,\eta_2\} \in E(\rho_G)}
  \E \big[ v (x_{\eta_1},x_{\eta_2})^{m({\eta_1,\eta_2})} \big] 
 \ \! \prod_{\eta \in V(\rho_G)}
\Lambda(dx_\eta )
\end{equation}
over connected non-flat partitions.
\end{prop}
\begin{proof}
 The functional 
\begin{equation}
\nonumber %    \label{pluggingin-1}
F(\rho ):=
% \sum_{\rho \in \Pi ( [n] \times [r] ) \atop { \rho \wedge \pi = \widehat{0} \atop {\rm (non-flat)} } }
\int_{(\R^d )^{|\rho |}}
  \prod_{\{\eta_1,\eta_2\} \in E(\rho_G)}
 \E \big[ v(x_{\eta_1},x_{\eta_2})^{m({\eta_1,\eta_2})} \big] 
   \ \! \prod_{\eta \in V(\rho_G)}
\Lambda(dx_\eta )
\end{equation}
 satisfies the connectedness factorization property
 \eqref{dia-factoriz}, as for
 $\sigma = b \times [r] \in \rho \vee \pi$ and
 $\sigma' = b' \times [r] \in \rho \vee \pi$
 with $b \not= b'$, the variables $(x_\eta )_{\eta \in \rho_b}$
 are distinct from the variables $(x_\eta )_{\eta \in \rho_{b'}}$
 in the above integration. 
 Hence, Relation~\eqref{nthexpectation0-2} follows from Proposition~\ref{mainthm-1}
 and the classical cumulant-moment relationship \eqref{jlkdf3},
 since by Proposition~\ref{p01-1-0},
 $\widehat{F}([n])$ is the $n$-$th$ moment of the multiparameter stochastic integral \eqref{e1}. 
\end{proof}
\section{Cumulants of subgraph counts} %  in the random-connection model}
\label{s5}
\noindent
 Let 
% $\Xi$ a point process with $\sigma$-finite intensity measure $\Lambda$ associated with
 $H:\R^d\times\R^d\to[0,1]$ 
 denote a % symmetric 
 measurable connection function such that
 \begin{equation}
 \nonumber %  \label{integ-connecting}
0<\int_{\real^d}H(x,y) \Lambda(dx)< \infty, 
\end{equation}
 for all $y \in\R$. \textcolor{red}{Recall that $\Xi$ is a Poisson point process on $\R^d$, $d \geq 1$, with intensity measure $\Lambda$.}
 Given $\omega \in \Omega$, for any $x,y\in\omega$ with $x\not= y$,
 an edge connecting $x$ and $y$ is added with probability $H(x,y)$,
 independently of the other pairs, and in this case we write $x \leftrightarrow y$.
 The resulting random graph, together with the point process $\Xi$,
 is called the random-connection model and denoted by $G_H(\Xi)$.
 % , and we write $V_i\leftrightarrow V_j$ if and only if $V_i,V_j\in\Xi$ are connected by an edge in $G_H(\Xi)$, $V_i\not= V_j$.

 \medskip

 In the case where the connection function $H$ is given by $H(x,y) := \bone_{\{\|x-y\|\leq R\}}$ for some $R>0$, the resulting graph is completely determined by the \textcolor{red}{geometry} of the underlying point process $\Xi$, and is called a random geometric graph, \textcolor{red}{see \cite{penrosebk} for more details,}
 which is included as a special case in this paper. The main case of interest in this section is general RCM. We shall have further discussion about the random geometric graph in Section~\ref{rgg}, as we believe it is of independent interest. 
% $G_r(\Xi)$.
% Note for a set $A$, we write $|A|$ for its cardinality, and for $x\in\R^d$, denote by $\|x\|$ its Euclidean norm. In the case when $\Xi$ a binomial point process with $n$ points and $H(x,y)\equiv p$ for some $0<p<1$, edges are added independently between $n$ vertices despite their locations. The resulting graph obtained is so-called Erd\"os-R\'enyi graph $G_n(p)$. So one can regard the RCM as a combination of two classical models in random graph theory, Erd\"os-R\'enyi graph and random geometric graph, see \cite{LNS21,CanTrinh22} for some recent developments. 
% For the RCM $G_H(\Xi)$, let $E$ denote the set of all edges of $G_H(\Xi)$.

\medskip

%  and we call the edge $(v_i,v_j)$ a {\it hop}. By {\it $k$-hop}, we mean a path with $k$ distinct edges. More precisely, given $\Xi$, $v_0,\dots,v_k\in\Xi$ (not necessarily to be distinct), if $v_{i-1}\leftrightarrow v_i$ for $i=1,\dots,k$, the path between $v_0$ and $v_k$ is called a $k$-hop, see \cite{kartun-giles} for more details. There are two type of $k$-hop count being of interest in this note:
% Note the assumption that $H(x,x)\equiv0$ in (\Ref{assmconn1}) eliminates the possibility of loops on the RCM, while circles are still permitted in $k$-hop.
% For the RCM $G_H(\Xi)$, let $E$ denote the set of all edges of $G_H(\Xi)$. When $\Xi$ is given, if $v_i,v_j\in\Xi$ are connected by an edge, we write $v_i\leftrightarrow v_j$ and we call the edge $(v_i,v_j)$ a {\it hop}. By {\it $k$-hop}, we mean a path with $k$ distinct edges. More precisely, given $\Xi$, $v_0,\dots,v_k\in\Xi$ (not necessarily to be distinct), if $v_{i-1}\leftrightarrow v_i$ for $i=1,\dots,k$, the path between $v_0$ and $v_k$ is called a $k$-hop, see \cite{kartun-giles} for more details. There are two type of $k$-hop count being of interest in this note:
% \begin{enumerate}[I.]
% \item $r+1$-hop connecting $x,y\in\R^d$: for $x,y$ fixed, $\tilde{\Xi}:=\Xi\cup\{x,y\}$. Let $N_{r+1}^{x,y}$ denote the total number of $r+1$-hop in the RCM $G_H(\tilde{\Xi})$ with endpoints $x$ and $y$, i.e. 
% \begin{equation}
% N_{r+1}^{x,y}:=\sum_{V_{i_1},\dots,V_{i_r}\in\Xi}\bone_{\{x\leftrightarrow V_{i_1}\}}\bone_{\{V_{i_1}\leftrightarrow V_{i_2}\}}\cdots\bone_{\{V_{i_{r-1}}\leftrightarrow V_{i_r}\}}\bone_{\{V_{i_r}\leftrightarrow y\}}.
% \end{equation}
% \item $k$-hop counts in general: When $\Lambda$ is a finite measure, denote by $N$ the total number of $k$-hop observed in the space, i.e. 
% \begin{equation}
% N_k:=\sum_{V_{i_0},\dots,V_{i_k}\in\Xi}\bone_{\{V_{i_0}\leftrightarrow V_{i_1}\}}\cdots\bone_{\{V_{i_{k-1}}\leftrightarrow V_{i_k}\}}.
% \end{equation}
% \end{enumerate}
% Note the assumption that $H(x,x)\equiv0$ in (\Ref{assmconn1}) eliminates the possibility of loops on the RCM, while circles are still permitted in $k$-hop.
 
 Given $G$ a connected graph with $|V(G)| = r$ vertices,
 we denote $N_G$ the count of subgraphs isomorphic to $G$ in the
 random-connection model $G_H( \Xi )$,
 which can be represented as the multiparameter stochastic integral 
\begin{equation}
\nonumber % \label{subgraphcount}
  N_G:=\sum_{V_1,\dots,V_r \in\Xi} \ 
  \prod_{\{i,j\} \in E(G)} 
  \bone_{\{V_i\leftrightarrow V_j\}}
  = \int_{(\real^d)^r}
  \prod_{\{i,j\} \in E(G)} 
  \bone_{\{x_i\leftrightarrow x_j\}}
  \ \! \Xi(\mathrm{d}x_1)\cdots \Xi(\mathrm{d}x_r), 
\end{equation}
up to division by the number of automorphisms of $G$.
 Here, we have $\bone_{\{V_i\leftrightarrow V_j\}}=1$ or $0$ depending
whether $V_i$ and $V_j$ are connected or not by an edge
in $G_H(\Xi)$, with 
\begin{equation}
\label{jflds} 
\bone_{\{x\leftrightarrow x\}}=0, \qquad x\in \real^d. 
\end{equation}
% since the same vertex cannot be used twice which ensures that subgraphs are properly counted without gluing of vertices.
\noindent
 The first moment of $N_G$ can be computed as 
\begin{equation}
\label{fm} 
\E [ N_G ] =\int_{(\R^d)^r}\left(\prod_{\{i,j\}\in E(G)}H(x_i,x_j)\right)\prod_{i=1}^r\Lambda(\mathrm{d}x_i).
\end{equation}
 Higher order moments can be computed from the following result which
is a direct consequence of Proposition~\ref{p01-1}
by taking $v(x,y) : = \bone_{\{x \leftrightarrow y\}}$
in \eqref{nthexpectation0-2} and by using
{\em non-flat} partition diagrams $\Gamma(\rho,\pi )$
 such that $\rho \wedge \pi = \widehat{0}$, 
% see Chapter~4 of \cite{peccatitaqqu} and Figure~\ref{fig:diagram2},
 to take into account condition~\eqref{jflds}.  
\begin{prop}
 \label{lma-diagram1}
% \label{djklssa} 
  Let $n \geq 1$ and $r\geq 2$.
  The moments and cumulants of $N_G$ are given by the summation 
\begin{equation} 
\label{nthexpectation-2} 
\E [(N_G)^n] =
\sum_{\rho \in \Pi ( [n] \times [r] )
  \atop {
    \rho \wedge \pi = \widehat{0}
    \atop
        {\rm (non-flat)}
  }
}
\int_{(\R^d )^{|\rho |}}
\Bigg(
\prod_{\{\eta_1,\eta_2\} \in E(\rho_G)}
H(x_{\eta_1},x_{\eta_2})
\Bigg)
\prod_{\eta \in V(\rho_G)}
\Lambda (\mathrm{d}x_\eta ), 
\end{equation} 
 over non-flat partitions, and by the summation   
\begin{equation}
  \label{cumulant-diagram1}
  \kappa_n(N_G)=\sum_{\rho \in \Pi_{\widehat{1}} ( [n] \times [r])
    \atop
    {\rho \wedge \pi = \widehat{0}
    \atop {\rm (non-flat \ \! connected)}}
  }
  \int_{(\R^d )^{|\rho |}}
  \Bigg(
\prod_{\{\eta_1,\eta_2\} \in E(\rho_G)}
H(x_{\eta_1},x_{\eta_2})
\Bigg)
\prod_{\eta \in V(\rho_G)}
\Lambda (\mathrm{d}x_\eta ), 
\end{equation} 
 over connected non-flat partitions. 
\end{prop}
\begin{proof}
 Relations~\eqref{nthexpectation-2}-\eqref{cumulant-diagram1}
 are consequence of Proposition~\ref{p01-1}, after taking
  $v_{i,j}(x_i,x_j):=\bone_{\{x_i\leftrightarrow x_j\}}$,
  $\{i,j\}\in E(G)$.
  The summations are restricted to {\em non-flat} partitions
  due to condition~\eqref{jflds} as in Section~2 of \cite{prkhp}. 
\end{proof}
\section{Asymptotic growth of subgraph count cumulants}
\label{s6}
\noindent 
In this section we consider the following assumption,
where $(\Lambda_\lambda)_{\lambda >0}$ is a family 
of intensity measures on $\real^d$ and $H(x,y)$ \textcolor{red}{is
 a connection function.}
\begin{assumption}
\label{a61} 
 Let $r\geq 2$ and $n \geq 1$.
 There exist constants $c_H, \textcolor{red}{C_H} > 0$
% and a function $\lambda \mapsto f_\lambda \in (0,\infty )$
 such that 
 for any connected non-flat partition 
 $\rho \in \Pi_{\widehat{1}} ( [n] \times [r])$, we have 
\begin{equation}
\label{integ-connecting3}
  c_H^{|E(\rho_G)|} ( \lambda \textcolor{red}{C_H} )^{|V(\rho_G)|}
\leq     \int_{\R^d}\cdots\int_{\R^d}
\Bigg(
\prod_{\{i,j\}\in E(\rho_G) }H(x_i,x_j)
\Bigg)
\
\prod_{k\in V(\rho_G) } \Lambda_\lambda (dx_k),
\quad \lambda >0. 
\end{equation}
\end{assumption}
 We consider two settings satisfying Assumption~\ref{a61}.
% \smallskip

% \label{page1}
\noindent
\begin{example}
  \label{examplea}
  Increasing intensity. 
  When the intensity measure $\Lambda_\lambda$ takes the form 
$$\Lambda_\lambda (\mathrm{d}x)=\lambda\mu(\mathrm{d}x),\qquad\lambda>0,$$
 for $\mu$ a finite diffuse measure on $\R^d$, 
 Assumption~\ref{a61} is satisfied by any translation-invariant
 continuous kernel function $H : \real^d\times \real^d \to [0,1]$ non vanishing at $(0,0)$. 
 Indeed, in this case there exist $c_H>0$ and a Borel set $B\subset \real^d$
 such that $\mu ( B)>0$ and
$$
H(x,y) = H(x-y,0) \geq c_H \bone_B(x)\bone_B(y), \qquad x,y \in \real^d,
$$
hence 
\begin{eqnarray*} 
  c_H^{|E(\rho_G)|} ( \mu ( B ))^{|V(\rho_G)|} & = & 
  c_H^{|E(\rho_G)|}
  \int_B \cdots \int_B 
  \prod_{k\in V(\rho_G) } \mu (dx_k)
  \\
   & \leq &  
  \int_{\R^d}\cdots\int_{\R^d}
  \Bigg(
  \prod_{\{i,j\}\in E(\rho_G)} H(x_i,x_j)
  \Bigg)
  \prod_{k\in V(\rho_G) } \mu (dx_k), 
\end{eqnarray*}
so that we can take $\textcolor{red}{C_H} := \mu (B)$. 
\end{example}
The setting of \textcolor{red}{the above increasing intensity example, Example~\ref{examplea},
 includes} the following long and short range dependence examples: 
\begin{enumerate}%[i)]
\item
 the power-law fading kernel 
 $H(x,y) = 1 \wedge \Vert x - y \Vert^{- \beta}$, $x,y\in \real^d$, for some $\beta > 0$,  
\item
 the Rayleigh fading kernel $H(x,y) = e^{ - \beta \Vert x - y\Vert^2}$, $x,y\in \real^d$, for some $\beta > 0$, 
\item
 the Boolean kernel $H(x,y)=\bone_{\{\|x-y\|\leq R \}}$, $x,y\in \real^d$,
 for some $R >0$, which yields the random geometric graph, with
% $B$ the centered ball of radius $r/2$ in $\real^d$, and
 $\textcolor{red}{C_H} = v_d (R /2)^d$,
 where $v_d$ denotes the volume of the unit ball in $\real^d$. 
\end{enumerate}
% \smallskip
% \label{page2}
\begin{example}
  \label{exampleb}
  Growing observation window. 
  When the intensity measure $\Lambda_\lambda$ takes the form 
$$\Lambda_\lambda (\mathrm{d}x)= {\bf 1}_{A_\lambda} (x) \mu(\mathrm{d}x),\qquad\lambda>0,$$
  where $(A_\lambda )_{\lambda >0}$ is a non-decreasing sequence
  of Borel subsets of $\real^d$, 
  $\mu$ a % finite
  diffuse \textcolor{red}{$\sigma$-finite measure} on $\R^d$, 
  and the kernel function $H : \real^d\times \real^d \to [0,1]$
  is lower bounded as $H(x,y) \geq c_H$, $x,y\in \real^d$ for some $c_H>0$,
  Assumption~\ref{a61} is satisfied with 
  $\mu ( A_\lambda ) = \textcolor{red}{C_H} \lambda$,
  e.g. when $A_\lambda$ is  the ball of radius $\lambda^{1/d} >0$
  and $\mu$ is the Lebesgue measure on $\real^d$,
  with $\textcolor{red}{C_H} = v_d$. 
\end{example} 

% \medskip
 
\noindent
% In addition, the measurable connection function $H: \R^d \times \R^d \to[0,1]$ is assumed to be symmetric and translation invariant, i.e. $H(x,y)=H(0,y-x)$, $x,y\in\R^d$.
 Next, we investigate the asymptotic behaviour of the cumulants $\kappa_n(N_G)$
as the intensity $\lambda$ tends to infinity, as a consequence of the partition diagram representation of cumulants. 
In what follows, given two positive functions $f$ and $g$ on $(1,\infty )$ we write
$f(\lambda ) \ll g(\lambda )$ if 
$\lim_{\lambda \to \infty} g(\lambda ) / f(\lambda ) = \infty$.
\begin{definition}
  Let $G$ be a connected graph with $|V(G)| = r$ vertices, $r\geq 2$.
  For every $\lambda >0$,
  let $G_{H_\lambda} (\Xi)$
  denote the random-connection model
  with connection function 
  $$
  H_\lambda(x,y):= c_\lambda H(x,y), \qquad x,y \in \real^d.
  $$ 
  We consider the following regimes.
\begin{itemize}
% \item Full random graph regime: $c_\lambda = 1 $, $\lambda > 0$. 
\item Dilute regime: for some constant $K>0$ we have 
\begin{equation}
    \label{fjnldsf}
    \lambda^{-1/\zeta} \ll c_\lambda \leq K,
    \qquad
    \lambda\to \infty,
\end{equation}
 where
 \begin{equation}
   \label{fjkldf}
   \zeta:=\max\left\{\frac{|E(H)|}{|V(H)|-1} \ \! : \ \!
   H\subseteq G, \ \! |V(H)| \geq 2 \right\}.
   \end{equation} 
\item Sparse regime: for some constants $K>0$ and $\alpha \geq 1$ we have
  \begin{equation}
    \label{fjnldsf-2}
    c_\lambda \leq \frac{K}{\lambda^\alpha},
        \qquad
    \lambda\to \infty. 
    \end{equation} % for some $C>0$. 
\end{itemize} 
\end{definition}
In case $c_\lambda = K$ for all $\lambda > 0$
we also say that we are in the full random graph regime,
and in the sequel we take $K=1$ for simplicity.
We note that in general we have $\zeta > 1$ except when $G$ is a tree,
in which case $\zeta = 1$. 
% since $\prod_{1\leq i,j\leq k}H(x_i,x_j)\le\prod_{(i,j)\in E(G) }H(x_i,x_j)\le1$
% we have the following estimate by taking $R=\beta^{-1/2}$,
\noindent
% \subsubsection*{Dilute regime}
% \noindent
% In what follows, we write $u(\lambda)\asymp v(\lambda)$ if the ratio $u(\lambda)/v(\lambda)$ is bounded away from $0$ and away from infinity by constants depending only on $H$ and $G$ as $\lambda$ tends to infinity. 
\begin{prop}[Dilute regime]
\label{t1}
  %Let the above notations and assumptions prevail.
Let $G$ be a connected graph with $|V(G)| = r$ vertices, $r\geq 2$, 
satisfying
Assumption~\ref{a61} for all $n\geq 1$
in the dilute regime \eqref{fjnldsf}. 
  We have the cumulant bounds 
\begin{equation}
\label{equiv-1}
 (n-1)! c_\lambda^{n |E(G)| } ( K_1 \lambda )^{1+(r-1)n}
 \leq 
 \kappa_n(N_G)
 \leq 
 n!^r c_\lambda^{n |E(G) |} ( K_2 \lambda )^{1+(r-1)n},
 \quad \lambda \geq 1, 
\end{equation}
for some constants $K_1$, $K_2>0$ independent of $\lambda$ and $n\geq 1$.
% where $\gamma\ge0$ and $\Delta>0$ are some constant. 
% for some constant $K>0$ independent of $\lambda, n\geq 1$. 
\end{prop} 
\begin{proof}
  % Under \eqref{integ-connecting},
  % in Proposition~\ref{lma-diagram1}
  % when $H_\lambda(x,y):= H(x,y)$, 
 % For our analysis of cumulant behavior
 We identify the leading terms in the sum \eqref{cumulant-diagram1}
 over connected non-flat partitions, in which 
 every summand involves 
 a factor $c_\lambda^{|E(\rho_G)|} \lambda^{|V(\rho_G)|}$,
 since every
 vertex in $\rho_G$ contributes a factor $\lambda$, and that 
 every edge contributes a factor $c_\lambda$.
 Therefore, it is sufficient to show that for any $\rho_G$ 
 with $\rho \in \mathcal{C} (n,r)$
 a connected non-flat partition of $[n]\times[r]$, 
 % with $v(\rho_G)<(r-1)n+1$,
  we have
  \begin{equation}\label{negli}
    c_\lambda^{|E(\rho_G)|} \lambda^{|V(\rho_G)|}
    = O \big(\lambda^{(r-1)n+1}c_\lambda^{n |E(G)|}\big). 
\end{equation}
 This claim follows from \eqref{fm} when $n=1$.
Suppose that \eqref{negli} holds up to the rank $n\ge1$, 
% with $|\rho|<(r-1)(n+1)+1$.
 and let $\rho \in \Pi_{\widehat{1}}( [n+1]\times [r])$ 
 be a connected non-flat partition. 
By Lemma~\ref{restrict-partition}, there exists $i\in[n+1]$ such that
  the subgraph $\widebar{\rho}_G$
  induced by $\rho_G$ on the vertex set
$$
V(\widebar{\rho}_G):=
\big\{
b\in \rho \ \!  : \ \!  b \cap (\cup_{j\ne i}\pi_j) \not= \emptyset
\big\} 
$$
is connected. 
 Let $\widehat{\rho}_G $ 
 denote the subgraph induced by $\rho_G$ on the vertex set
$$
V(\widehat{\rho}_G):=
\big\{
b\in \rho \ \!  : \ \!  b \cap \pi_{i} \not= \emptyset
\big\}, 
$$
with
 $\widehat{\rho}_G\simeq G$
because $\rho$ is non-flat, and let $H:=\widebar{\rho}_G\cap\widehat{\rho}_G$. 
Since $H\subseteq\widehat{\rho}_G$ 
we have  
$$
\lambda^{|V(H)|-1} c_{\lambda}^{|E(H)|} \geq 
\big( \lambda c_{\lambda}^{\zeta }\big)^{|E(H)|/\zeta },
\qquad
\lambda \geq 1. 
$$ 
 Hence from 
$
  \lim_{\lambda \to \infty} 
  \lambda c_\lambda^{\zeta} = \infty$ 
  we get 
  $$
  \liminf_{\lambda \to \infty}
  \lambda^{|V(H)|-1} c_{\lambda}^{|E(H)|} 
  >0. 
  $$ 
 On the other hand, by the induction hypothesis we have 
\begin{equation}
  \frac{\lambda^{|V(\widebar{\rho}_G)|}c_\lambda^{|E(\widebar{\rho}_G)|}}{\lambda^{(r-1)n+1}c_\lambda^{n|E(G)|}} = O(1), 
\end{equation}
hence, since
$|V(\widehat{\rho}_G)| = |V(G)|$
and
$|E(\widehat{\rho}_G)| = |E(G)|$, 
\begin{eqnarray}
  \nonumber
  \frac{\lambda^{|\rho|}c_\lambda^{|E(\rho_G)|}}{\lambda^{(r-1)(n+1)+1}c_\lambda^{(n+1)|E(G)|}}
  &=&\frac{\lambda^{|V(\widebar{\rho}_G)|+|V(\widehat{\rho}_G)|-|V(H)|}c_\lambda^{|E(\widebar{\rho}_G)|+|E(\widehat{\rho}_G)|-|E(H)|}}{\lambda^{(r-1)(n+1)+1}c_\lambda^{(n+1)|E(G)|}}
  \\
  \nonumber
  &=&\frac{\lambda^{|V(\widebar{\rho}_G)|}c_\lambda^{|E(\widebar{\rho}_G)|}}{\lambda^{(r-1)n+1}c_\lambda^{n|E(G)|}}\cdot 
  \frac{\lambda^{|V(\widehat{\rho}_G)|}c_\lambda^{|E(\widehat{\rho}_G)|}}{\lambda^{r}c_\lambda^{|E(G)|}}\cdot
  \frac{\lambda^{-|V(H)|}c_\lambda^{-|E(H)|}}{\lambda^{-1}}
  \\
  \nonumber
  &=&\frac{\lambda^{|V(\widebar{\rho}_G)|}c_\lambda^{|E(\widebar{\rho}_G)|}}{\lambda^{(r-1)n+1}c_\lambda^{n|E(G)|}}\cdot\big(
  \lambda^{|V(H)|-1} c_{\lambda}^{|E(H)|}\big)^{-1}  \\
  \nonumber
  &=& O(1), 
\end{eqnarray}
% If $|V(H)|=1$ we have $\lambda^{|V(H)|-1} c_{\lambda}^{|E(H)|}=1$, $\lambda >0$. On the other hand,
% because and satisfies $v(\widebar{\rho}_G) \leq (r-1)n+1$.
 therefore \eqref{negli} holds at the rank $n+1$. 
 As a consequence, the leading terms in \eqref{cumulant-diagram1} are those associated
  with the connected partition diagrams $\Gamma(\rho ,\pi )$ having the highest
  block count, i.e. which have $1+(r-1)n$ blocks, see Figure~\ref{fig:diagram3}
  for a sample of such a partition diagram.

% \medskip

\begin{figure}[H]
\captionsetup[subfigure]{font=footnotesize}
\centering
\subcaptionbox{Diagram $\Gamma(\rho,\pi)$ and graph $\widetilde{\rho}_G$ in blue.}[.5\textwidth]{%
\begin{tikzpicture}[scale=0.9] 
\draw[black, thick] (0,0) rectangle (5,6);

\node[anchor=east,font=\small] at (0.8,5) {1};
\node[anchor=east,font=\small] at (0.8,4) {2};
\node[anchor=east,font=\small] at (0.8,3) {3};
\node[anchor=east,font=\small] at (0.8,2) {4};
\node[anchor=east,font=\small] at (0.8,1) {5};

\node[anchor=south,font=\small] at (1,0) {1};
\node[anchor=south,font=\small] at (2,0) {2};
\node[anchor=south,font=\small] at (3,0) {3};
\node[anchor=south,font=\small] at (4,0) {4};

\filldraw [gray] (1,1) circle (2pt);
\filldraw [gray] (2,1) circle (2pt);
\filldraw [gray] (3,1) circle (2pt);
\filldraw [gray] (4,1) circle (2pt);
\filldraw [gray] (1,2) circle (2pt);
\filldraw [gray] (2,2) circle (2pt);
\filldraw [gray] (3,2) circle (2pt);
\filldraw [gray] (4,2) circle (2pt);
\filldraw [gray] (1,3) circle (2pt);
\filldraw [gray] (2,3) circle (2pt);
\filldraw [gray] (3,3) circle (2pt);
\filldraw [gray] (4,3) circle (2pt);
\filldraw [gray] (2,3) circle (2pt);
\filldraw [gray] (1,4) circle (2pt);
\filldraw [gray] (2,4) circle (2pt);
\filldraw [gray] (3,4) circle (2pt);
\filldraw [gray] (4,4) circle (2pt);
\filldraw [gray] (1,5) circle (2pt);
\filldraw [gray] (2,5) circle (2pt);
\filldraw [gray] (3,5) circle (2pt);
\filldraw [gray] (4,5) circle (2pt);

\draw[very thick] (1,5) -- (1,1); 

\foreach \i in {1,...,5}
         {
           \draw[thick,dash dot,blue] (1,\i) .. controls (1.5,\i+.5) .. (2,\i);
           \draw[thick,dash dot,blue] (2,\i) .. controls (2.5,\i+.5) .. (3,\i);
           \draw[thick,dash dot,blue] (2,\i) .. controls (3,\i-.5) .. (4,\i);
           \draw[thick,dash dot,blue] (3,\i) .. controls (3.5,\i+.5) .. (4,\i);
         }
         
\end{tikzpicture}}%
\subcaptionbox{Diagram $\Gamma(\rho ,\pi)$ and graph $\rho_G$ in red.}[.5\textwidth]{
\begin{tikzpicture}[scale=0.9] 
\draw[black, thick] (0,0) rectangle (5,6);

\node[anchor=east,font=\small] at (0.8,5) {1};
\node[anchor=east,font=\small] at (0.8,4) {2};
\node[anchor=east,font=\small] at (0.8,3) {3};
\node[anchor=east,font=\small] at (0.8,2) {4};
\node[anchor=east,font=\small] at (0.8,1) {5};

\node[anchor=south,font=\small] at (1,0) {1};
\node[anchor=south,font=\small] at (2,0) {2};
\node[anchor=south,font=\small] at (3,0) {3};
\node[anchor=south,font=\small] at (4,0) {4};

\filldraw [gray] (1,1) circle (2pt);
\filldraw [gray] (2,1) circle (2pt);
\filldraw [gray] (3,1) circle (2pt);
\filldraw [gray] (4,1) circle (2pt);
\filldraw [gray] (1,2) circle (2pt);
\filldraw [gray] (2,2) circle (2pt);
\filldraw [gray] (3,2) circle (2pt);
\filldraw [gray] (4,2) circle (2pt);
\filldraw [gray] (1,3) circle (2pt);
\filldraw [gray] (2,3) circle (2pt);
\filldraw [gray] (3,3) circle (2pt);
\filldraw [gray] (4,3) circle (2pt);
\filldraw [gray] (2,3) circle (2pt);
\filldraw [gray] (1,4) circle (2pt);
\filldraw [gray] (2,4) circle (2pt);
\filldraw [gray] (3,4) circle (2pt);
\filldraw [gray] (4,4) circle (2pt);
\filldraw [gray] (1,5) circle (2pt);
\filldraw [gray] (2,5) circle (2pt);
\filldraw [gray] (3,5) circle (2pt);
\filldraw [gray] (4,5) circle (2pt);

\draw[very thick] (1,5) -- (1,1); 

\foreach \i in {1,...,5}
         {
           \draw[thick,dash dot,purple] (1,\i) .. controls (1.5,\i+.5) .. (2,\i);
           \draw[thick,dash dot,purple] (2,\i) .. controls (2.5,\i+.5) .. (3,\i);
           \draw[thick,dash dot,purple] (2,\i) .. controls (3,\i-.5) .. (4,\i);
           \draw[thick,dash dot,purple] (3,\i) .. controls (3.5,\i+.5) .. (4,\i);
         }
\end{tikzpicture}}%
\caption{Example of maximal connected partition diagram with $n=5$ and $r=4$.}
\label{fig:diagram3}
\end{figure}

\vspace{-0.5cm} 

\noindent
 Finally, we observe that any maximal partition $\rho$ satisfies
% Also, for any $\rho\in\mathcal{M}(n,r)$, 
$|E(\rho_G)|=n\times|E(G)|$, 
as can be checked in Figure~\ref{fig:diagram3}.
% That is because, essentially, $\rho_G$ is obtained by merging vertices between $n$ replica of $G$ and making them connected without ``gluing'' edges. 
 Therefore, by % virtue of the
 \eqref{coeff-0}-\eqref{coeff-1},
 \eqref{cumulant-diagram1} and \eqref{integ-connecting3}, we obtain 
\begin{eqnarray*} 
  \lefteqn{
    c^{n|E(G)|}
    C^{1+(r-1)n}
    c_\lambda^{n|E(G)|}
    ( (r-1)r )^{n-1}(n-1)!
    \lambda^{1+(r-1)n}
  }
  \\
   & \leq &   
    \lambda^{1+(r-1)n}
% c_\lambda^{n|E(G)|}
  \sum_{\rho\in\mathcal{M}(n,r)}\int_{(\R^d)^{1+(r-1)n}}
  \Bigg(
  \prod_{\{\eta_1,\eta_2\}\in E(\rho_G)}H_\lambda(x_{\eta_1},x_{\eta_2})
  \Bigg)
  \prod_{\eta\in V(\rho_G)}\mu(dx_{\eta}),
  \\
   & \leq &     \kappa_n(N_G)
% \\  & \leq &   K  B_{nr}  \lambda^{1+(r-1)n}  (\mu ( \real^d))^{1+(r-1)n}  c_\lambda^{n|E(G)|}
% \\  & \leq &   K  (nr)!   \lambda^{1+(r-1)n}  (\mu ( \real^d))^{1+(r-1)n}  c_\lambda^{n|E(G)|}
      \\
  & \leq & 
  n!^r r!^{n-1} 
  ( 1 + \mu ( \real^d))^{1+(r-1)n}
  c_\lambda^{n|E(G)|}
 \lambda^{1+(r-1)n}, 
\end{eqnarray*}
%  & \asymp &
%  a_n\lambda^{1+(r-1)n}
%  \int_{(\R^d )^{1+(r-1)n}} 
%  \prod_{i=1}^{1+(r-1)n}
%       H(x_1,x_i ) 
%  \ \! \mu(dx_1)\cdots \mu (dx_{1+(r-1)n}) 
%  \\
which yields \eqref{equiv-1}.
\end{proof}
 In what follows, we consider the centered and normalized subgraph count cumulants defined as 
 $$
 \widetilde{N}_G
 := \frac{N_G - \kappa_1 (N_G)}{\sqrt{\kappa_2(N_G)}}, \qquad n \geq 1. 
$$
 The following result shows that for $n\geq 3$ the normalized cumulant
 $\kappa_n(\widetilde{N}_G)$ tends to zero in \eqref{Statuleviciuscond},
 hence $\widetilde{N}_G$ converges in distribution to the normal
 distribution by Theorem~1 in \cite{Janson1988}.   
\begin{corollary}[Dilute regime]
\label{t1-c}
  %Let the above notations and assumptions prevail.
Let $G$ be a connected graph with $|V(G)| = r$ vertices, $r\geq 2$, 
satisfying Assumption~\ref{a61} for $n=2$ in the dilute regime \eqref{fjnldsf}. 
  We have the normalized cumulant bounds 
 \begin{equation}
    \label{Statuleviciuscond}
  \big|\kappa_n \big(\widetilde{N}_G\big)\big|
  \leq n!^r ( K \lambda )^{-(n/2-1)},
%  \le\frac{n!^{1+\gamma}}{\Delta^{n-2}},
  \qquad \lambda \geq 1, \quad n\geq 2,
\end{equation}
where $K>0$ is a constant independent of $\lambda >0$ and $n\geq 1$.
% where $\gamma\ge0$ and $\Delta>0$ are some constant. 
% for some constant $K>0$ independent of $\lambda, n\geq 1$. 
\end{corollary}
\begin{proof}
  We note that the upper bound in \eqref{equiv-1}
  does not require Assumption~\ref{a61},
  hence we have, for $n\geq 2$,
$$ 
   \big|\kappa_n(\widetilde{N}_G)\big|
  \leq 
   \frac{n!^r % r!^n
     c_\lambda^{n|E(G)|} (K_2 \lambda )^{1+(r-1)n}}{\big((2-1)! c_\lambda^{2|E(G)|}
   (K_1 \lambda )^{1+2(r-1)}\big)^{n/2}}
 = 
   K_2
   \left(
   \frac{ % r!
      (K_2/K_1)^{r-1}}{\sqrt{K_1}}
   \right)^n
   n!^r \lambda^{-(n/2-1)}. 
$$ 
\end{proof}
The following result yields a positive cumulant growth
of order $\alpha     -(\alpha - 1)r>0$
in \eqref{cumulant-rhop2-0} for trees in the sparse regime
with $\alpha \in [1, r/(r-1) )$,
while in the case of non-tree graphs such as
cycle graphs the growth rate \textcolor{red}{exponent}
$r - \alpha |E(G)|
\leq ( 1 - \alpha ) r \leq 0$
is negative or zero in \eqref{cumulant-rhoph1} and \eqref{cumulant-rhoph2}. 
In addition, the normalized cumulant 
$\kappa_n(\widetilde{N}_G)$ tends to zero for $n\geq 3$ in \eqref{jfkla} only
when $G$ is a tree, in which case 
$\widetilde{N}_G$ converges in distribution to the normal
distribution by Theorem~1 in \cite{Janson1988}.   
We note that when $\alpha = 1$, \eqref{jfkla} is consistent with
 \eqref{Statuleviciuscond}. 
\begin{prop}[Sparse regime] 
\label{th6.4}
  % Let $G$ be a tree graph with $r$ edges. 
  Let $G$ be a connected graph with $|V(G)|=r$ vertices, $r\geq 2$,
  satisfying Assumption~\ref{a61} for all $n\geq 1$
  in the sparse regime \eqref{fjnldsf-2}. 
  \begin{enumerate}%[a)] 
  \item 
 If $G$ is a tree, i.e. $|E(G)| = r-1$, we have the cumulant bounds % equivalence 
\begin{equation} 
\label{cumulant-rhop2-0}
  (K_1)^{(r-1)n}
 \lambda^{\alpha  -(\alpha - 1)r }
     \leq 
% \lambda  \sum_{k=r}^{1+(r-1)n}          (K_1)^k {\cal N}^{(n,r)}_k      \leq 
  \kappa_n(N_G)
  %  \leq  \lambda  \sum_{k=r}^{1+(r-1)n} (K_2)^k {\cal N}^{(n,r)}_k
  \leq 
  n!^r
  % r!^n
  (K_2)^{(r-1)n} \lambda^{ \alpha     -(\alpha - 1)r       } ,
  \quad \lambda \geq 1, 
\end{equation} 
 for some constants $K_1>0$, $K_2>1$ independent of $\lambda, n\geq 1$.
% where ${\cal N}^{(n,r)}_k$ denotes the number of connected partitions $\rho\in\Pi_{\widehat{1}} ([n]\times[r])$ for which $\rho_G$ is a tree with $k$ vertices,
\item
 If $G$ is not a tree, i.e. $|E(G)|\geq r$,
 we have the cumulant bounds % equivalence 
 \begin{equation} 
 \label{cumulant-rhoph1}
  (K_1)^r 
 \lambda^{r-\alpha |E(G)|}
 \leq 
  \kappa_n(N_G)
  \leq
   n!^r
  % r!^n
   (K_2)^{(r-1)n}
   \lambda^{r-\alpha |E(G)|}, 
   \quad \lambda \geq 1,
 \end{equation} 
 for some constants $K_1>0$, $K_2>1$ independent of $\lambda, n\geq 1$. 
\item
  If $G$ is a cycle, i.e. $|E(G)| = r$,
  we have the cumulant bounds % equivalence 
 \begin{equation} 
\label{cumulant-rhoph2}
  (K_1)^r 
 \lambda^{- (\alpha - 1 )r}
\leq 
  \kappa_n(N_G)
  \leq
  n!^r
 % r!^n
  (K_2)^{(r-1)n}
  \lambda^{ - (\alpha -1)r},
    \quad \lambda \geq 1, 
\end{equation} 
 for some constants $K_1>0$, $K_2>1$ independent of $\lambda, n\geq 1$. 
\end{enumerate}
\end{prop}
\begin{proof}
  In the sparse regime \eqref{fjnldsf-2}, 
  every edge in the graph $\rho_G$ contributes a power $\lambda^{-\alpha}$
  and every vertex contributes a power $\lambda$,
  hence every term in \eqref{cumulant-diagram1} contributes a power
  \begin{equation}
    \label{fjkld34} 
  \lambda^{|V(\rho_G)| - \alpha |E(\rho_G)|}
  =
  \lambda^{    \alpha    - (\alpha - 1 ) |V(\rho_G)|    + ( |V (\rho_G)| - |E(\rho_G)| - 1 ) \alpha  }
 \leq
 \lambda^{   \alpha   - (\alpha - 1 ) |V(\rho_G)|     }
\end{equation} 
  since $|V (\rho_G)| - |E(\rho_G)| - 1 \leq 0$. 
  In addition, for any connected partition 
 $\rho\in\Pi_{\widehat{1}} ([n]\times[r])$,
  we have 
  $$r\le|V(\rho_G)|\leq 1+(r-1)n. 
  $$ 
    
% \medskip
  
\noindent
 $a)$
 When $G$ is a tree and the graph $\rho_G$ is also a tree, 
 i.e. $|V (\rho_G)| - |E(\rho_G)| - 1 = 0$, and 
 the maximal order 
 $\lambda^{
   \alpha
   - (\alpha - 1 ) |V(\rho_G)|
     }
 $ is attained in \eqref{fjkld34}, 
  see Figure~\ref{fig:diagram2-1-4} for an example.

\smallskip

\begin{figure}[H]
\captionsetup[subfigure]{font=footnotesize}
\centering
\subcaptionbox{Diagram $\Gamma(\rho,\pi)$ and multigraph $\widetilde{\rho}_G$ in blue.}[.5\textwidth]{%
\begin{tikzpicture}[scale=0.9] 
\draw[black, thick] (0,0) rectangle (5,6);

\node[anchor=east,font=\small] at (0.8,5) {1};
\node[anchor=east,font=\small] at (0.8,4) {2};
\node[anchor=east,font=\small] at (0.8,3) {3};
\node[anchor=east,font=\small] at (0.8,2) {4};
\node[anchor=east,font=\small] at (0.8,1) {5};

\node[anchor=south,font=\small] at (1,0) {1};
\node[anchor=south,font=\small] at (2,0) {2};
\node[anchor=south,font=\small] at (3,0) {3};
\node[anchor=south,font=\small] at (4,0) {4};

\filldraw [gray] (1,1) circle (2pt);
\filldraw [gray] (2,1) circle (2pt);
\filldraw [gray] (3,1) circle (2pt);
\filldraw [gray] (4,1) circle (2pt);
\filldraw [gray] (1,2) circle (2pt);
\filldraw [gray] (2,2) circle (2pt);
\filldraw [gray] (3,2) circle (2pt);
\filldraw [gray] (4,2) circle (2pt);
\filldraw [gray] (1,3) circle (2pt);
\filldraw [gray] (2,3) circle (2pt);
\filldraw [gray] (3,3) circle (2pt);
\filldraw [gray] (4,3) circle (2pt);
\filldraw [gray] (2,3) circle (2pt);
\filldraw [gray] (1,4) circle (2pt);
\filldraw [gray] (2,4) circle (2pt);
\filldraw [gray] (3,4) circle (2pt);
\filldraw [gray] (4,4) circle (2pt);
\filldraw [gray] (1,5) circle (2pt);
\filldraw [gray] (2,5) circle (2pt);
\filldraw [gray] (3,5) circle (2pt);
\filldraw [gray] (4,5) circle (2pt);

\draw[very thick] (1,5) -- (1,4); 
\draw[very thick] (3,5) -- (4,4);

\draw[very thick] (1,2) -- (1,1);
\draw[very thick] (2,2) -- (2,4);
\draw[very thick] (2,1) -- (3,2) -- (4,3) -- (3,4);

\foreach \i in {1,...,5}
         {
           \draw[thick,dash dot,blue] (1,\i) .. controls (1.5,\i+.5) .. (2,\i);
           \draw[thick,dash dot,blue] (2,\i) .. controls (2.5,\i+.5) .. (3,\i);
%           \draw[thick,dash dot,blue] (3,\i) .. controls (3.5,\i+.5) .. (4,\i);
           \draw[thick,dash dot,blue] (2,\i) .. controls (3,\i-.5) .. (4,\i);
         }
         
\end{tikzpicture}}%
\subcaptionbox{Diagram $\Gamma(\sigma,\pi)$ and graph $\rho_G$ in red.}[.5\textwidth]{
\begin{tikzpicture}[scale=0.9] 
\draw[black, thick] (0,0) rectangle (5,6);

\node[anchor=east,font=\small] at (0.8,5) {1};
\node[anchor=east,font=\small] at (0.8,4) {2};
\node[anchor=east,font=\small] at (0.8,3) {3};
\node[anchor=east,font=\small] at (0.8,2) {4};
\node[anchor=east,font=\small] at (0.8,1) {5};

\node[anchor=south,font=\small] at (1,0) {1};
\node[anchor=south,font=\small] at (2,0) {2};
\node[anchor=south,font=\small] at (3,0) {3};
\node[anchor=south,font=\small] at (4,0) {4};

\filldraw [gray] (1,1) circle (2pt);
\filldraw [gray] (2,1) circle (2pt);
\filldraw [gray] (3,1) circle (2pt);
\filldraw [gray] (4,1) circle (2pt);
\filldraw [gray] (1,2) circle (2pt);
\filldraw [gray] (2,2) circle (2pt);
\filldraw [gray] (3,2) circle (2pt);
\filldraw [gray] (4,2) circle (2pt);
\filldraw [gray] (1,3) circle (2pt);
\filldraw [gray] (2,3) circle (2pt);
\filldraw [gray] (3,3) circle (2pt);
\filldraw [gray] (4,3) circle (2pt);
\filldraw [gray] (2,3) circle (2pt);
\filldraw [gray] (1,4) circle (2pt);
\filldraw [gray] (2,4) circle (2pt);
\filldraw [gray] (3,4) circle (2pt);
\filldraw [gray] (4,4) circle (2pt);
\filldraw [gray] (1,5) circle (2pt);
\filldraw [gray] (2,5) circle (2pt);
\filldraw [gray] (3,5) circle (2pt);
\filldraw [gray] (4,5) circle (2pt);

\draw[very thick] (1,5) -- (1,4); 
\draw[very thick] (3,5) -- (4,4);

\draw[very thick] (1,2) -- (1,1);
\draw[very thick] (2,2) -- (2,4);
\draw[very thick] (2,1) -- (3,2) -- (4,3) -- (3,4);

\foreach \i in {1,...,3}
         {
           \draw[thick,dash dot,purple] (1,\i) .. controls (1.5,\i+.5) .. (2,\i);
           \draw[thick,dash dot,purple] (2,\i) .. controls (2.5,\i+.5) .. (3,\i);
%           \draw[thick,dash dot,blue] (3,\i) .. controls (3.5,\i+.5) .. (4,\i);
           \draw[thick,dash dot,purple] (2,\i) .. controls (3,\i-.5) .. (4,\i);
         }

           \draw[thick,dash dot,purple] (1,4) .. controls (1.5,4+.5) .. (2,4);
%           \draw[thick,dash dot,blue] (2,2) .. controls (2.5,2+.5) .. (3,2);
 %          \draw[thick,dash dot,blue] (3,4) .. controls (3.5,4+.5) .. (4,4);
           \draw[thick,dash dot,purple] (2,4) .. controls (3,4-.5) .. (4,4);
\foreach \i in {5,...,5}
         {
           \draw[thick,dash dot,purple] (1,\i) .. controls (1.5,\i+.5) .. (2,\i);
           \draw[thick,dash dot,purple] (2,\i) .. controls (2.5,\i+.5) .. (3,\i);
           %\draw[thick,dash dot,blue] (3,\i) .. controls (3.5,\i+.5) .. (4,\i);
           \draw[thick,dash dot,purple] (2,\i) .. controls (3,\i-.5) .. (4,\i);
         }
\end{tikzpicture}}%
\caption{Example of connected partition diagram with $\rho_G$ a tree and $n=5$, $r=4$.}
% Connected non-flat partition diagram with $G$ a tree and $n=5$, $r=4$.}
\label{fig:diagram2-1-4}
\end{figure}

\vskip-0.3cm

\noindent
 In this case, the corresponding term in \eqref{cumulant-diagram1} contributes a power
$$
 \lambda^{|V(\rho_G)|-\alpha|E(\rho_G)|}
  =
  \lambda^{
    \alpha
    - (\alpha - 1 ) |V(\rho_G)|
  }, 
  \qquad \lambda \geq 1. 
  $$
% which reaches the maximal rate in \eqref{fjkld34}. 
 In this case, since $|V(\rho_G)|\geq r$ and $\alpha \geq 1$, the optimal rate
 $\lambda^{ \alpha - ( \alpha - 1 ) r }$
 is attained by the partition diagrams
  $\Gamma(\rho ,\pi )$ such that 
  $|V(\rho_G )|=r$, as illustrated in Figure~\ref{fig:diagram4-0}. 
  
\begin{figure}[H]
\captionsetup[subfigure]{font=footnotesize}
\centering
\subcaptionbox{Diagram $\Gamma(\rho,\pi)$ and multigraph $\widetilde{\rho}_G$ in blue.}[.5\textwidth]{%
\begin{tikzpicture}[scale=0.9] 

\draw[black, thick] (0,0) rectangle (5,6);

\node[anchor=east,font=\small] at (0.8,5) {1};
\node[anchor=east,font=\small] at (0.8,4) {2};
\node[anchor=east,font=\small] at (0.8,3) {3};
\node[anchor=east,font=\small] at (0.8,2) {4};
\node[anchor=east,font=\small] at (0.8,1) {5};

\node[anchor=south,font=\small] at (1,0) {1};
\node[anchor=south,font=\small] at (2,0) {2};
\node[anchor=south,font=\small] at (3,0) {3};
\node[anchor=south,font=\small] at (4,0) {4};

\filldraw [gray] (1,1) circle (2pt);
\filldraw [gray] (2,1) circle (2pt);
\filldraw [gray] (3,1) circle (2pt);
\filldraw [gray] (4,1) circle (2pt);
\filldraw [gray] (1,2) circle (2pt);
\filldraw [gray] (2,2) circle (2pt);
\filldraw [gray] (3,2) circle (2pt);
\filldraw [gray] (4,2) circle (2pt);
\filldraw [gray] (1,3) circle (2pt);
\filldraw [gray] (2,3) circle (2pt);
\filldraw [gray] (3,3) circle (2pt);
\filldraw [gray] (4,3) circle (2pt);
\filldraw [gray] (2,3) circle (2pt);
\filldraw [gray] (1,4) circle (2pt);
\filldraw [gray] (2,4) circle (2pt);
\filldraw [gray] (3,4) circle (2pt);
\filldraw [gray] (4,4) circle (2pt);
\filldraw [gray] (1,5) circle (2pt);
\filldraw [gray] (2,5) circle (2pt);
\filldraw [gray] (3,5) circle (2pt);
\filldraw [gray] (4,5) circle (2pt);

\foreach \i in {1,...,4}
{
\draw[very thick] (\i,5) -- (\i,1); 
}

  \foreach \i in {1,...,5}
         {
           \draw[thick,dash dot,blue] (1,\i) .. controls (1.5,\i+.5) .. (2,\i);
           \draw[thick,dash dot,blue] (2,\i) .. controls (2.5,\i+.5) .. (3,\i);
%           \draw[thick,dash dot,blue] (3,\i) .. controls (3.5,\i+.5) .. (4,\i);
           \draw[thick,dash dot,blue] (2,\i) .. controls (3,\i-.5) .. (4,\i);
         }
         
\end{tikzpicture}}%
\subcaptionbox{Diagram $\Gamma(\rho,\pi)$ and graph $\rho_G$ in red.}[.5\textwidth]{
\begin{tikzpicture}[scale=0.9] 
\draw[black, thick] (0,0) rectangle (5,6);

\node[anchor=east,font=\small] at (0.8,5) {1};
\node[anchor=east,font=\small] at (0.8,4) {2};
\node[anchor=east,font=\small] at (0.8,3) {3};
\node[anchor=east,font=\small] at (0.8,2) {4};
\node[anchor=east,font=\small] at (0.8,1) {5};

\node[anchor=south,font=\small] at (1,0) {1};
\node[anchor=south,font=\small] at (2,0) {2};
\node[anchor=south,font=\small] at (3,0) {3};
\node[anchor=south,font=\small] at (4,0) {4};

\filldraw [gray] (1,1) circle (2pt);
\filldraw [gray] (2,1) circle (2pt);
\filldraw [gray] (3,1) circle (2pt);
\filldraw [gray] (4,1) circle (2pt);
\filldraw [gray] (1,2) circle (2pt);
\filldraw [gray] (2,2) circle (2pt);
\filldraw [gray] (3,2) circle (2pt);
\filldraw [gray] (4,2) circle (2pt);
\filldraw [gray] (1,3) circle (2pt);
\filldraw [gray] (2,3) circle (2pt);
\filldraw [gray] (3,3) circle (2pt);
\filldraw [gray] (4,3) circle (2pt);
\filldraw [gray] (2,3) circle (2pt);
\filldraw [gray] (1,4) circle (2pt);
\filldraw [gray] (2,4) circle (2pt);
\filldraw [gray] (3,4) circle (2pt);
\filldraw [gray] (4,4) circle (2pt);
\filldraw [gray] (1,5) circle (2pt);
\filldraw [gray] (2,5) circle (2pt);
\filldraw [gray] (3,5) circle (2pt);
\filldraw [gray] (4,5) circle (2pt);

\foreach \i in {1,...,4}
{
\draw[very thick] (\i,5) -- (\i,1); 
}

\foreach \i in {5,...,5}
         {
           \draw[thick,dash dot,purple] (1,\i) .. controls (1.5,\i+.5) .. (2,\i);
           \draw[thick,dash dot,purple] (2,\i) .. controls (2.5,\i+.5) .. (3,\i);
           %\draw[thick,dash dot,blue] (3,\i) .. controls (3.5,\i+.5) .. (4,\i);
           \draw[thick,dash dot,purple] (2,\i) .. controls (3,\i-.5) .. (4,\i);
         }
         
\end{tikzpicture}}%
\caption{Tree diagram $\rho_G$ with $G$ a tree with $|V(\rho_G)|=r$ and $n=5$, $r=4$.}
\label{fig:diagram4-0}
\end{figure}

\vskip-0.3cm

\noindent
 We conclude to \eqref{cumulant-rhop2-0} 
 as in the proof of Proposition~\ref{t1}, 
 by upper bounding the count of connected partitions from \eqref{coeff-0}
 and by lower bounding it by $1$. 

 \noindent
   $b)$ 
   When $G$ is not a tree it contains at least one cycle, and
   for any partition $\rho\in\Pi_{\widehat{1}} ([n]\times[r])$  
   the same holds for the graph $\rho_G$. 
In this case, the highest order contribution in
\eqref{cumulant-diagram1} is attained by connected non-flat partition diagrams 
$\Gamma ( \rho , \pi )$, $\rho\in\Pi_{\widehat{1}} ([n]\times[r])$,
such that $\rho_G$ has $|V(\rho_G)| = r$ vertices, 
and their contribution is given by a power of order $\lambda^{r-\alpha |E(G)|}$. 
 An example of such partition $\rho$ is given in Figure~\ref{fig:diagram4}, 
 with $G$ a cycle. 

\smallskip

\begin{figure}[H]
\captionsetup[subfigure]{font=footnotesize}
\centering
\subcaptionbox{Diagram $\Gamma(\rho,\pi)$ and multigraph $\widetilde{\rho}_G$ in blue.}[.5\textwidth]{%
\begin{tikzpicture}[scale=0.9] 
\draw[black, thick] (0,0) rectangle (5,4);

% \node[anchor=east,font=\small] at (0.8,5) {1};
% \node[anchor=east,font=\small] at (0.8,4) {2};
\node[anchor=east,font=\small] at (0.8,3) {1};
\node[anchor=east,font=\small] at (0.8,2) {2};
\node[anchor=east,font=\small] at (0.8,1) {3};

\node[anchor=south,font=\small] at (1,0) {1};
\node[anchor=south,font=\small] at (2,0) {2};
\node[anchor=south,font=\small] at (3,0) {3};
\node[anchor=south,font=\small] at (4,0) {4};

\filldraw [gray] (1,1) circle (2pt);
\filldraw [gray] (2,1) circle (2pt);
\filldraw [gray] (3,1) circle (2pt);
\filldraw [gray] (4,1) circle (2pt);
\filldraw [gray] (1,2) circle (2pt);
\filldraw [gray] (2,2) circle (2pt);
\filldraw [gray] (3,2) circle (2pt);
\filldraw [gray] (4,2) circle (2pt);
\filldraw [gray] (1,3) circle (2pt);
\filldraw [gray] (2,3) circle (2pt);
\filldraw [gray] (3,3) circle (2pt);
\filldraw [gray] (4,3) circle (2pt);
\filldraw [gray] (2,3) circle (2pt);
% \filldraw [gray] (1,4) circle (2pt);
% \filldraw [gray] (2,4) circle (2pt);
% \filldraw [gray] (3,4) circle (2pt);
% \filldraw [gray] (4,4) circle (2pt);
% \filldraw [gray] (1,5) circle (2pt);
% \filldraw [gray] (2,5) circle (2pt);
% \filldraw [gray] (3,5) circle (2pt);
% \filldraw [gray] (4,5) circle (2pt);

\foreach \i in {1,...,4}
{
\draw[very thick] (\i,3) -- (\i,1); 
}

\foreach \i in {1,...,3}
         {
           \draw[thick,dash dot,blue] (1,\i) .. controls (1.5,\i+.5) .. (2,\i);
           \draw[thick,dash dot,blue] (2,\i) .. controls (2.5,\i+.5) .. (3,\i);
           \draw[thick,dash dot,blue] (3,\i) .. controls (3.5,\i+.5) .. (4,\i);
           \draw[thick,dash dot,blue] (4,\i) .. controls (2.5,\i-.5) .. (1,\i);
         }
         
\end{tikzpicture}}%
\subcaptionbox{Diagram $\Gamma(\sigma,\pi)$ and graph $\rho_G$ in red.}[.5\textwidth]{
\begin{tikzpicture}[scale=0.9] 
\draw[black, thick] (0,0) rectangle (5,4);

% \node[anchor=east,font=\small] at (0.8,5) {1};
% \node[anchor=east,font=\small] at (0.8,4) {2};
\node[anchor=east,font=\small] at (0.8,3) {1};
\node[anchor=east,font=\small] at (0.8,2) {2};
\node[anchor=east,font=\small] at (0.8,1) {3};

\node[anchor=south,font=\small] at (1,0) {1};
\node[anchor=south,font=\small] at (2,0) {2};
\node[anchor=south,font=\small] at (3,0) {3};
\node[anchor=south,font=\small] at (4,0) {4};

\filldraw [gray] (1,1) circle (2pt);
\filldraw [gray] (2,1) circle (2pt);
\filldraw [gray] (3,1) circle (2pt);
\filldraw [gray] (4,1) circle (2pt);
\filldraw [gray] (1,2) circle (2pt);
\filldraw [gray] (2,2) circle (2pt);
\filldraw [gray] (3,2) circle (2pt);
\filldraw [gray] (4,2) circle (2pt);
\filldraw [gray] (1,3) circle (2pt);
\filldraw [gray] (2,3) circle (2pt);
\filldraw [gray] (3,3) circle (2pt);
\filldraw [gray] (4,3) circle (2pt);
\filldraw [gray] (2,3) circle (2pt);
% \filldraw [gray] (1,4) circle (2pt);
% \filldraw [gray] (2,4) circle (2pt);
% \filldraw [gray] (3,4) circle (2pt);
% \filldraw [gray] (4,4) circle (2pt);
% \filldraw [gray] (1,5) circle (2pt);
% \filldraw [gray] (2,5) circle (2pt);
% \filldraw [gray] (3,5) circle (2pt);
% \filldraw [gray] (4,5) circle (2pt);

\foreach \i in {1,...,4}
{
\draw[very thick] (\i,3) -- (\i,1); 
}

           \draw[thick,dash dot,purple] (1,3) .. controls (1.5,3+.5) .. (2,3);
           \draw[thick,dash dot,purple] (2,3) .. controls (2.5,3+.5) .. (3,3);
           \draw[thick,dash dot,purple] (3,3) .. controls (3.5,3+.5) .. (4,3);
           \draw[thick,dash dot,purple] (4,3) .. controls (2.5,3-.5) .. (1,3);
         
\end{tikzpicture}}%
\caption{Cycle graph $\rho_G$ with $G$ a cycle graph and $n=5$, $r=4$.}
\label{fig:diagram4}
\end{figure}

\vskip-0.3cm

\noindent
 Indeed, in order to remain non-flat,
 the partition $\rho$ can only be modified
 into a partition $\sigma$ by splitting a block of $\rho_G$ in two, 
 which entails the addition of a number $q$ of edges, $q\geq 1$, 
 resulting into an additional factor $\lambda^{1- q \alpha} \leq 1$ 
 that may only lower the order of the contribution, see
 Figures~\ref{fig:diagram3-1-0}-\ref{fig:diagram3-1-3}
 for an example where $G$ is a graph with one cycle.
 
\begin{figure}[H]
\captionsetup[subfigure]{font=footnotesize}
\centering
\subcaptionbox{Diagram $\Gamma(\rho,\pi)$ with order $\lambda^{4-4\alpha}$.}[.47\textwidth]{%
\begin{tikzpicture}[scale=0.9] 
\draw[black, thick] (0,0) rectangle (5,4);

\node[anchor=east,font=\small] at (0.8,3) {1};
\node[anchor=east,font=\small] at (0.8,2) {2};
\node[anchor=east,font=\small] at (0.8,1) {3};
\node[anchor=south,font=\small] at (1,0) {1};
\node[anchor=south,font=\small] at (2,0) {2};
\node[anchor=south,font=\small] at (3,0) {3};
\node[anchor=south,font=\small] at (4,0) {4};

\filldraw [gray] (1,1) circle (2pt);
\filldraw [gray] (2,1) circle (2pt);
\filldraw [gray] (3,1) circle (2pt);
\filldraw [gray] (4,1) circle (2pt);
\filldraw [gray] (1,2) circle (2pt);
\filldraw [gray] (2,2) circle (2pt);
\filldraw [gray] (3,2) circle (2pt);
\filldraw [gray] (4,2) circle (2pt);
\filldraw [gray] (1,3) circle (2pt);
\filldraw [gray] (2,3) circle (2pt);
\filldraw [gray] (3,3) circle (2pt);
\filldraw [gray] (4,3) circle (2pt);
\filldraw [gray] (2,3) circle (2pt);

\draw[very thick] (1,1) -- (1,1) -- (1,3);
\draw[very thick] (2,1) -- (2,1) -- (2,3);
\draw[very thick] (3,1) -- (3,1) -- (3,3);
\draw[very thick] (4,1) -- (4,1) -- (4,3);
\foreach \i in {3}
         {
           \draw[thick,dash dot,purple] (1,\i) .. controls (1.5,\i+.5) .. (2,\i);
           \draw[thick,dash dot,purple] (2,\i) .. controls (2.5,\i+.5) .. (3,\i);
           \draw[thick,dash dot,purple] (3,\i) .. controls (3.5,\i+.5) .. (4,\i);
           \draw[thick,dash dot,purple] (2,\i) .. controls (3,\i-.5) .. (4,\i);
         }
         
\end{tikzpicture}}
\hfill
\subcaptionbox{Diagram $\Gamma(\sigma,\pi)$ with order $\lambda^{5-5\alpha}= \lambda^{4-4\alpha}\lambda^{- ( \alpha - 1)}$.}[.52\textwidth]{
\begin{tikzpicture}[scale=0.9] 
\draw[black, thick] (0,0) rectangle (5,4);

\node[anchor=east,font=\small] at (0.8,3) {1};
\node[anchor=east,font=\small] at (0.8,2) {2};
\node[anchor=east,font=\small] at (0.8,1) {3};
\node[anchor=south,font=\small] at (1,0) {1};
\node[anchor=south,font=\small] at (2,0) {2};
\node[anchor=south,font=\small] at (3,0) {3};
\node[anchor=south,font=\small] at (4,0) {4};

\filldraw [gray] (1,1) circle (2pt);
\filldraw [gray] (2,1) circle (2pt);
\filldraw [gray] (3,1) circle (2pt);
\filldraw [gray] (4,1) circle (2pt);
\filldraw [gray] (1,2) circle (2pt);
\filldraw [gray] (2,2) circle (2pt);
\filldraw [gray] (3,2) circle (2pt);
\filldraw [gray] (4,2) circle (2pt);
\filldraw [gray] (1,3) circle (2pt);
\filldraw [gray] (2,3) circle (2pt);
\filldraw [gray] (3,3) circle (2pt);
\filldraw [gray] (4,3) circle (2pt);
\filldraw [gray] (2,3) circle (2pt);

% \draw[very thick] (1,1) -- (1,1) -- (1,3);
\draw[very thick] (1,1) -- (1,2);
\draw[very thick] (2,1) -- (2,1) -- (2,3);
\draw[very thick] (3,1) -- (3,1) -- (3,3);
\draw[very thick] (4,1) -- (4,1) -- (4,3);

\draw[thick,dash dot,purple] (1,2) .. controls (1.5,2.5) .. (2,2);

\foreach \i in {3}
         {
           \draw[thick,dash dot,purple] (1,\i) .. controls (1.5,\i+.5) .. (2,\i);
           \draw[thick,dash dot,purple] (2,\i) .. controls (2.5,\i+.5) .. (3,\i);
           \draw[thick,dash dot,purple] (3,\i) .. controls (3.5,\i+.5) .. (4,\i);
           \draw[thick,dash dot,purple] (2,\i) .. controls (3,\i-.5) .. (4,\i);
         }
         
\end{tikzpicture}}
\caption{Splitting of a vertex with addition of one edge and $n=3$, $r=4$.}
\label{fig:diagram3-1-0}
\end{figure}

\vskip-0.3cm

\noindent

\smallskip

\begin{figure}[H]
\captionsetup[subfigure]{font=footnotesize}
\centering
\subcaptionbox{Diagram $\Gamma(\rho,\pi)$ with order $\lambda^{4-4\alpha}$.}[.49\textwidth]{%
\begin{tikzpicture}[scale=0.9] 
\draw[black, thick] (0,0) rectangle (5,4);

\node[anchor=east,font=\small] at (0.8,3) {1};
\node[anchor=east,font=\small] at (0.8,2) {2};
\node[anchor=east,font=\small] at (0.8,1) {3};
\node[anchor=south,font=\small] at (1,0) {1};
\node[anchor=south,font=\small] at (2,0) {2};
\node[anchor=south,font=\small] at (3,0) {3};
\node[anchor=south,font=\small] at (4,0) {4};

\filldraw [gray] (1,1) circle (2pt);
\filldraw [gray] (2,1) circle (2pt);
\filldraw [gray] (3,1) circle (2pt);
\filldraw [gray] (4,1) circle (2pt);
\filldraw [gray] (1,2) circle (2pt);
\filldraw [gray] (2,2) circle (2pt);
\filldraw [gray] (3,2) circle (2pt);
\filldraw [gray] (4,2) circle (2pt);
\filldraw [gray] (1,3) circle (2pt);
\filldraw [gray] (2,3) circle (2pt);
\filldraw [gray] (3,3) circle (2pt);
\filldraw [gray] (4,3) circle (2pt);
\filldraw [gray] (2,3) circle (2pt);

\draw[very thick] (1,1) -- (1,1) -- (1,3);
\draw[very thick] (2,1) -- (2,1) -- (2,3);
\draw[very thick] (3,1) -- (3,1) -- (3,3);
\draw[very thick] (4,1) -- (4,1) -- (4,3);
\foreach \i in {3}
         {
           \draw[thick,dash dot,purple] (1,\i) .. controls (1.5,\i+.5) .. (2,\i);
           \draw[thick,dash dot,purple] (2,\i) .. controls (2.5,\i+.5) .. (3,\i);
           \draw[thick,dash dot,purple] (3,\i) .. controls (3.5,\i+.5) .. (4,\i);
           \draw[thick,dash dot,purple] (2,\i) .. controls (3,\i-.5) .. (4,\i);
         }
         
\end{tikzpicture}}
\hfill
\subcaptionbox{Diagram $\Gamma(\sigma,\pi)$ with order $\lambda^{5-6\alpha}=\lambda^{4-4\alpha}\lambda^{1-2\alpha}$.}[.49\textwidth]{
\begin{tikzpicture}[scale=0.9] 
\draw[black, thick] (0,0) rectangle (5,4);

\node[anchor=east,font=\small] at (0.8,3) {1};
\node[anchor=east,font=\small] at (0.8,2) {2};
\node[anchor=east,font=\small] at (0.8,1) {3};
\node[anchor=south,font=\small] at (1,0) {1};
\node[anchor=south,font=\small] at (2,0) {2};
\node[anchor=south,font=\small] at (3,0) {3};
\node[anchor=south,font=\small] at (4,0) {4};

\filldraw [gray] (1,1) circle (2pt);
\filldraw [gray] (2,1) circle (2pt);
\filldraw [gray] (3,1) circle (2pt);
\filldraw [gray] (4,1) circle (2pt);
\filldraw [gray] (1,2) circle (2pt);
\filldraw [gray] (2,2) circle (2pt);
\filldraw [gray] (3,2) circle (2pt);
\filldraw [gray] (4,2) circle (2pt);
\filldraw [gray] (1,3) circle (2pt);
\filldraw [gray] (2,3) circle (2pt);
\filldraw [gray] (3,3) circle (2pt);
\filldraw [gray] (4,3) circle (2pt);
\filldraw [gray] (2,3) circle (2pt);

\draw[very thick] (1,1) -- (1,1) -- (1,3);
% \draw[very thick] (2,1) -- (2,1) -- (2,3);
\draw[very thick] (2,1) -- (2,2);
\draw[very thick] (3,1) -- (3,1) -- (3,3);
\draw[very thick] (4,1) -- (4,1) -- (4,3);

\draw[thick,dash dot,purple] (1,2) .. controls (1.5,2.5) .. (2,2);
\draw[thick,dash dot,purple] (2,2) .. controls (2.5,2.5) .. (3,2);
\draw[thick,dash dot,purple] (2,2) .. controls (3,1.5) .. (4,2);


\foreach \i in {3}
         {
           \draw[thick,dash dot,purple] (1,\i) .. controls (1.5,\i+.5) .. (2,\i);
           \draw[thick,dash dot,purple] (2,\i) .. controls (2.5,\i+.5) .. (3,\i);
           \draw[thick,dash dot,purple] (3,\i) .. controls (3.5,\i+.5) .. (4,\i);
           \draw[thick,dash dot,purple] (2,\i) .. controls (3,\i-.5) .. (4,\i);
         }
         
\end{tikzpicture}}
\caption{Splitting of a vertex with addition of three edges and $n=3$, $r=4$.}
% \label{fig:diagram3-1-1}
\end{figure}

\vskip-0.3cm

\noindent

\smallskip

\begin{figure}[H]
\captionsetup[subfigure]{font=footnotesize}
\centering
\subcaptionbox{Diagram $\Gamma(\rho,\pi)$ with order $\lambda^{4-4\alpha}$.}[.49\textwidth]{%
\begin{tikzpicture}[scale=0.9] 
\draw[black, thick] (0,0) rectangle (5,4);

\node[anchor=east,font=\small] at (0.8,3) {1};
\node[anchor=east,font=\small] at (0.8,2) {2};
\node[anchor=east,font=\small] at (0.8,1) {3};
\node[anchor=south,font=\small] at (1,0) {1};
\node[anchor=south,font=\small] at (2,0) {2};
\node[anchor=south,font=\small] at (3,0) {3};
\node[anchor=south,font=\small] at (4,0) {4};

\filldraw [gray] (1,1) circle (2pt);
\filldraw [gray] (2,1) circle (2pt);
\filldraw [gray] (3,1) circle (2pt);
\filldraw [gray] (4,1) circle (2pt);
\filldraw [gray] (1,2) circle (2pt);
\filldraw [gray] (2,2) circle (2pt);
\filldraw [gray] (3,2) circle (2pt);
\filldraw [gray] (4,2) circle (2pt);
\filldraw [gray] (1,3) circle (2pt);
\filldraw [gray] (2,3) circle (2pt);
\filldraw [gray] (3,3) circle (2pt);
\filldraw [gray] (4,3) circle (2pt);
\filldraw [gray] (2,3) circle (2pt);

\draw[very thick] (1,1) -- (1,1) -- (1,3);
\draw[very thick] (2,1) -- (2,1) -- (2,3);
\draw[very thick] (3,1) -- (3,1) -- (3,3);
\draw[very thick] (4,1) -- (4,1) -- (4,3);
\foreach \i in {3}
         {
           \draw[thick,dash dot,purple] (1,\i) .. controls (1.5,\i+.5) .. (2,\i);
           \draw[thick,dash dot,purple] (2,\i) .. controls (2.5,\i+.5) .. (3,\i);
           \draw[thick,dash dot,purple] (3,\i) .. controls (3.5,\i+.5) .. (4,\i);
           \draw[thick,dash dot,purple] (2,\i) .. controls (3,\i-.5) .. (4,\i);
         }
         
\end{tikzpicture}}
\hfill
\subcaptionbox{Diagram $\Gamma(\sigma,\pi)$ with order $\lambda^{5-6\alpha}=\lambda^{4-4\alpha}\lambda^{1-2\alpha}$.}[.49\textwidth]{
\begin{tikzpicture}[scale=0.9] 
\draw[black, thick] (0,0) rectangle (5,4);

\node[anchor=east,font=\small] at (0.8,3) {1};
\node[anchor=east,font=\small] at (0.8,2) {2};
\node[anchor=east,font=\small] at (0.8,1) {3};
\node[anchor=south,font=\small] at (1,0) {1};
\node[anchor=south,font=\small] at (2,0) {2};
\node[anchor=south,font=\small] at (3,0) {3};
\node[anchor=south,font=\small] at (4,0) {4};

\filldraw [gray] (1,1) circle (2pt);
\filldraw [gray] (2,1) circle (2pt);
\filldraw [gray] (3,1) circle (2pt);
\filldraw [gray] (4,1) circle (2pt);
\filldraw [gray] (1,2) circle (2pt);
\filldraw [gray] (2,2) circle (2pt);
\filldraw [gray] (3,2) circle (2pt);
\filldraw [gray] (4,2) circle (2pt);
\filldraw [gray] (1,3) circle (2pt);
\filldraw [gray] (2,3) circle (2pt);
\filldraw [gray] (3,3) circle (2pt);
\filldraw [gray] (4,3) circle (2pt);
\filldraw [gray] (2,3) circle (2pt);

\draw[very thick] (1,1) -- (1,1) -- (1,3);
\draw[very thick] (2,1) -- (2,1) -- (2,3);
% \draw[very thick] (3,1) -- (3,1) -- (3,3);
\draw[very thick] (3,1) -- (3,2);
\draw[very thick] (4,1) -- (4,1) -- (4,3);

\draw[thick,dash dot,purple] (2,2) .. controls (2.5,2.5) .. (3,2);
\draw[thick,dash dot,purple] (3,2) .. controls (3.5,2.5) .. (4,2);


\foreach \i in {3}
         {
           \draw[thick,dash dot,purple] (1,\i) .. controls (1.5,\i+.5) .. (2,\i);
           \draw[thick,dash dot,purple] (2,\i) .. controls (2.5,\i+.5) .. (3,\i);
           \draw[thick,dash dot,purple] (3,\i) .. controls (3.5,\i+.5) .. (4,\i);
           \draw[thick,dash dot,purple] (2,\i) .. controls (3,\i-.5) .. (4,\i);
         }
         
\end{tikzpicture}}
\caption{Splitting of a vertex with addition of two edges and $n=3$, $r=4$.}
% \label{fig:diagram3-1-2}
\end{figure}

\vskip-0.3cm

\noindent

\smallskip

\begin{figure}[H]
\captionsetup[subfigure]{font=footnotesize}
\centering
\subcaptionbox{Diagram $\Gamma(\rho,\pi)$ with order $\lambda^{4-4\alpha}$.}[.49\textwidth]{%
\begin{tikzpicture}[scale=0.9] 
\draw[black, thick] (0,0) rectangle (5,4);

\node[anchor=east,font=\small] at (0.8,3) {1};
\node[anchor=east,font=\small] at (0.8,2) {2};
\node[anchor=east,font=\small] at (0.8,1) {3};
\node[anchor=south,font=\small] at (1,0) {1};
\node[anchor=south,font=\small] at (2,0) {2};
\node[anchor=south,font=\small] at (3,0) {3};
\node[anchor=south,font=\small] at (4,0) {4};

\filldraw [gray] (1,1) circle (2pt);
\filldraw [gray] (2,1) circle (2pt);
\filldraw [gray] (3,1) circle (2pt);
\filldraw [gray] (4,1) circle (2pt);
\filldraw [gray] (1,2) circle (2pt);
\filldraw [gray] (2,2) circle (2pt);
\filldraw [gray] (3,2) circle (2pt);
\filldraw [gray] (4,2) circle (2pt);
\filldraw [gray] (1,3) circle (2pt);
\filldraw [gray] (2,3) circle (2pt);
\filldraw [gray] (3,3) circle (2pt);
\filldraw [gray] (4,3) circle (2pt);
\filldraw [gray] (2,3) circle (2pt);

\draw[very thick] (1,1) -- (1,1) -- (1,3);
\draw[very thick] (2,1) -- (2,1) -- (2,3);
\draw[very thick] (3,1) -- (3,1) -- (3,3);
\draw[very thick] (4,1) -- (4,1) -- (4,3);
\foreach \i in {3}
         {
           \draw[thick,dash dot,purple] (1,\i) .. controls (1.5,\i+.5) .. (2,\i);
           \draw[thick,dash dot,purple] (2,\i) .. controls (2.5,\i+.5) .. (3,\i);
           \draw[thick,dash dot,purple] (3,\i) .. controls (3.5,\i+.5) .. (4,\i);
           \draw[thick,dash dot,purple] (2,\i) .. controls (3,\i-.5) .. (4,\i);
         }
         
\end{tikzpicture}}
\hfill
\subcaptionbox{Diagram $\Gamma(\sigma,\pi)$ with order $\lambda^{5-6\alpha}=\lambda^{4-4\alpha}\lambda^{1-2\alpha}$.}[.49\textwidth]{
\begin{tikzpicture}[scale=0.9] 
\draw[black, thick] (0,0) rectangle (5,4);

\node[anchor=east,font=\small] at (0.8,3) {1};
\node[anchor=east,font=\small] at (0.8,2) {2};
\node[anchor=east,font=\small] at (0.8,1) {3};
\node[anchor=south,font=\small] at (1,0) {1};
\node[anchor=south,font=\small] at (2,0) {2};
\node[anchor=south,font=\small] at (3,0) {3};
\node[anchor=south,font=\small] at (4,0) {4};

\filldraw [gray] (1,1) circle (2pt);
\filldraw [gray] (2,1) circle (2pt);
\filldraw [gray] (3,1) circle (2pt);
\filldraw [gray] (4,1) circle (2pt);
\filldraw [gray] (1,2) circle (2pt);
\filldraw [gray] (2,2) circle (2pt);
\filldraw [gray] (3,2) circle (2pt);
\filldraw [gray] (4,2) circle (2pt);
\filldraw [gray] (1,3) circle (2pt);
\filldraw [gray] (2,3) circle (2pt);
\filldraw [gray] (3,3) circle (2pt);
\filldraw [gray] (4,3) circle (2pt);
\filldraw [gray] (2,3) circle (2pt);

\draw[very thick] (1,1) -- (1,1) -- (1,3);
\draw[very thick] (2,1) -- (2,1) -- (2,3);
\draw[very thick] (3,1) -- (3,1) -- (3,3);
% \draw[very thick] (4,1) -- (4,1) -- (4,3);
\draw[very thick] (4,1) -- (4,2);

\draw[thick,dash dot,purple] (3,2) .. controls (3.5,2.5) .. (4,2);
\draw[thick,dash dot,purple] (2,2) .. controls (3,1.5) .. (4,2);

\foreach \i in {3}
         {
           \draw[thick,dash dot,purple] (1,\i) .. controls (1.5,\i+.5) .. (2,\i);
           \draw[thick,dash dot,purple] (2,\i) .. controls (2.5,\i+.5) .. (3,\i);
           \draw[thick,dash dot,purple] (3,\i) .. controls (3.5,\i+.5) .. (4,\i);
           \draw[thick,dash dot,purple] (2,\i) .. controls (3,\i-.5) .. (4,\i);
         }
         
\end{tikzpicture}}
\caption{Splitting of a vertex with addition of two edges and $n=3$, $r=4$.}
\label{fig:diagram3-1-3}
\end{figure}

\vskip-0.3cm

\noindent
 When $G$ is a triangle with $n=2$ and $r=3$, 
 the above procedure can be reversed 
 by first merging a vertex and then gluing edges, see Figure~\ref{fig:diagram3-23},
 which results into ``overlapping'' all copies of the graph $G$. 
  
\begin{figure}[H]
\captionsetup[subfigure]{font=footnotesize}
\centering
\subcaptionbox{Merging one vertex.}[.3\textwidth]{%
\begin{tikzpicture}[scale=0.9] 
\draw[black, thick] (0.2,0.2) rectangle (3.8,4.3);

\filldraw [gray] (1,1) circle (2pt);
\filldraw [gray] (2,1) circle (2pt);
\filldraw [gray] (3,1) circle (2pt);
\filldraw [gray] (1,2) circle (2pt);
\filldraw [gray] (2,2) circle (2pt);
\filldraw [gray] (3,2) circle (2pt);

\draw[very thick] (1,1) -- (1,2);
\foreach \i in {1,2}
         {
           \draw[thick,dash dot,purple] (1,\i) .. controls (1.5,\i+.5) .. (2,\i);
           \draw[thick,dash dot,purple] (2,\i) .. controls (2.5,\i+.5) .. (3,\i);
           \draw[thick,dash dot,purple] (1,\i) .. controls (2,\i-.5) .. (3,\i);
         }
\filldraw [black] (1.1,3.8) circle (1.5pt);         
\filldraw [black] (1.1,2.8) circle (1.5pt);   
\filldraw [black] (1.97,3.3) circle (1.5pt); 
\filldraw [black] (2.84,3.8) circle (1.5pt);  
\filldraw [black] (2.84,2.8) circle (1.5pt);  
\draw[thick,blue] (1.1,3.8) -- (2.84,2.8);
\draw[thick,blue] (1.1,2.8) -- (1.1,3.8);
\draw[thick,blue] (1.1,2.8) -- (2.84,3.8);
\draw[thick,blue] (2.84,2.8) -- (2.84,3.8);
         
\end{tikzpicture}}%
\hfill
\subcaptionbox{Gluing one edge.}[.3\textwidth]{%
\begin{tikzpicture}[scale=0.9] 
\draw[black, thick] (0.2,0.2) rectangle (3.8,4.3);

\filldraw [gray] (1,1) circle (2pt);
\filldraw [gray] (2,1) circle (2pt);
\filldraw [gray] (3,1) circle (2pt);
\filldraw [gray] (1,2) circle (2pt);
\filldraw [gray] (2,2) circle (2pt);
\filldraw [gray] (3,2) circle (2pt);

\draw[very thick] (1,1) -- (1,2);
\draw[very thick] (3,1) -- (3,2);
\foreach \i in {1,2}
         {
           \draw[thick,dash dot,purple] (1,\i) .. controls (1.5,\i+.5) .. (2,\i);
           \draw[thick,dash dot,purple] (2,\i) .. controls (2.5,\i+.5) .. (3,\i);
         }
          \draw[thick,dash dot,purple] (1,2) .. controls (2,1.5) .. (3,2);
\filldraw [black] (2,3.8) circle (1.5pt);         
\filldraw [black] (2,2.8) circle (1.5pt);   
\filldraw [black] (1.13,3.3) circle (1.5pt);   
\filldraw [black] (2.87,3.3) circle (1.5pt);  
 
\draw[thick,blue] (2,3.8) -- (2,2.8);
\draw[thick,blue] (2,3.8) -- (1.13,3.3);
\draw[thick,blue] (2,2.8) -- (1.13,3.3);
\draw[thick,blue] (2,3.8) -- (2.87,3.3);
\draw[thick,blue] (2,2.8) -- (2.87,3.3);
                 
\end{tikzpicture}}
\hfill
\subcaptionbox{Gluing three edges.}[.3\textwidth]{%
\begin{tikzpicture}[scale=0.9] 
\draw[black, thick] (0.2,0.2) rectangle (3.8,4.3);

\filldraw [gray] (1,1) circle (2pt);
\filldraw [gray] (2,1) circle (2pt);
\filldraw [gray] (3,1) circle (2pt);
\filldraw [gray] (1,2) circle (2pt);
\filldraw [gray] (2,2) circle (2pt);
\filldraw [gray] (3,2) circle (2pt);
\draw[very thick] (1,1) -- (1,2);
\draw[very thick] (2,1) -- (2,2);
\draw[very thick] (3,1) -- (3,2);
\foreach \i in {2}
         {
           \draw[thick,dash dot,purple] (1,\i) .. controls (1.5,\i+.5) .. (2,\i);
           \draw[thick,dash dot,purple] (2,\i) .. controls (2.5,\i+.5) .. (3,\i);
           \draw[thick,dash dot,purple] (1,\i) .. controls (2,\i-.5) .. (3,\i);
         }

\filldraw [black] (2,3.8) circle (1.5pt);         
\filldraw [black] (1.5,2.93) circle (1.5pt);   
\filldraw [black] (2.5,2.93) circle (1.5pt);   

\draw[thick,blue] (2,3.8) -- (1.5,2.93);
\draw[thick,blue] (2,3.8) -- (2.5,2.93);
\draw[thick,blue] (2.5,2.93) -- (1.5,2.93);
                  
\end{tikzpicture}}
\caption{Diagram patterns with $G$ a triangle and $n=2$, $r=3$.}
\label{fig:diagram3-23}
\end{figure}

\vskip-0.3cm

\noindent
 As in part-$(b)$ above, we lower
 bound $\kappa_n(N_G)$ using a single partition,
 and we upper bound using the total count of
 connected non-flat partitions using Lemma~\ref{fjkldsf-l}-$b)$
 to obtain \eqref{cumulant-rhoph1}.
 
\noindent
   $c)$ is a direct consequence of part~$b)$ above.
\end{proof}

\begin{corollary}[Sparse regime] 
\label{th6.4-c}
  % Let $G$ be a tree graph with $r$ edges. 
  Let $G$ be a connected graph with $|V(G)|=r$ vertices, $r\geq 2$,
  satisfying Assumption~\ref{a61} for $n=2$ in the sparse regime \eqref{fjnldsf-2}. 
  \begin{enumerate}%[a)] 
  \item 
    If $G$ is a tree, i.e. $|E(G)| = r-1$, we have the normalize
    cumulant bounds % equivalence 
 \begin{equation}
   \label{jfkla} 
   \big|\kappa_n(\widetilde{N}_G)\big|
  \leq 
   (K_3)^n 
 n!^r
      \lambda^{
       - (
\alpha       -(\alpha - 1)r 
       ) ( n/2-1 ) },
   \qquad \lambda \geq 1,
   \quad n \geq 2, 
\end{equation} 
 where $K_3:=(K_2/K_1)^{r-1}$.
\item
 If $G$ is not a tree, i.e. $|E(G)|\geq r$,
 we have the normalized cumulant bounds % equivalence 
 \begin{equation}
   \label{fjklds34} 
    \big|\kappa_n(\widetilde{N}_G)\big|
  \leq 
    n!^r
  % r!^n
    (K_3)^n
        \lambda^{(\alpha |E(G)|-r)(n/2-1)}  , 
   \qquad \lambda \geq 1,
  \quad n \geq 2, 
\end{equation} 
 for some $K_3 > 0$. 
\item
  If $G$ is a cycle, i.e. $|E(G)| = r$,
  we have the normalized cumulant bounds % equivalence 
 \begin{equation}
   \label{fjklds34-2} 
    \big|\kappa_n(\widetilde{N}_G)\big|
  \leq 
  n!^r
 %  r!^n
  (K_3)^n
    \lambda^{ (\alpha -1 )(n/2-1)r}, 
     \qquad \lambda \geq 1,
 \quad n \geq 2, 
\end{equation} 
 for some $K_3>0$.
  \end{enumerate}
\end{corollary}
\begin{proof}
  We note that the upper bound in \eqref{equiv-1}
  does not require Assumption~\ref{a61}. 
  Regarding \eqref{jfkla}, we have  
\begin{eqnarray*} 
  \big|\kappa_n(\widetilde{N}_G)\big|
& \leq &  
  \frac{n!^r % r!^n
    K_2^{(r-1)n}}{
   (
    (K_1)^{2(r-1)}
        \lambda^{
\alpha      -(\alpha - 1)r 
        }
        )^{n/2}}
  \lambda^{
 \alpha          -(\alpha - 1)r 
  }
    \\
    & = & 
 \left(
   \frac{K_2}{K_1}
   \right)^{(r-1)n}
   n!^r
        \lambda^{-(
 \alpha      -(\alpha - 1)r 
       ) ( n/2-1)
   }, \qquad n \geq 2.
\end{eqnarray*} 
   \noindent
 Regarding \eqref{fjklds34}, we have  
$$ 
   \big|\kappa_n(\widetilde{N}_G)\big|
 \leq 
   \frac{
  n!^r
  % r!^n
  (K_2)^{(r-1)n}
  \lambda^{r-\alpha |E(G)|}
   }{
            (
     (K_1)^r
     \lambda^{r-\alpha |E(G)|}
     )^{n/2}}
=
   n!^r
   \frac{
  % r!^n
     (K_2)^{(r-1)n}}{(K_1)^{nr/2}}
      \lambda^{ - (r-\alpha |E(G)|)(n/2 - 1)}  , 
  \quad n \geq 2. 
$$ 
 Finally, the bound 
 \eqref{fjklds34-2} is a direct consequence of \eqref{fjklds34}. 
\end{proof}
\section{Asymptotic normality of subgraph counts}
\label{s6-1}
\noindent
% and the references therein for more details.
 In this section, we let $H(x,y)$ be a connection function  
  satisfying Assumption~\ref{a61},
  and consider the
  random-connection model $G_{H_\lambda} (\Xi)$
  where $H_\lambda(x,y):= c_\lambda H(x,y)$,
  $\lambda >0$.
% in which the Berry-Esseen rate is obtained when $r=2$.
% We believe the rate can be improved significantly if a more tighter upper could found in the inequality \eqref{equiv-1}. 
\begin{corollary}[Dilute regime]
  \label{c01}
  Let $G$ be a connected graph with $|V(G)|=r$ vertices, $r\geq 2$,
  satisfying Assumption~\ref{a61}. 
  In the dilute regime \eqref{fjnldsf},
  the normalized subgraph count $\widetilde{N}_G$ in 
  $G_{H_\lambda} (\Xi)$ satisfies
  the Kolmogorov distance bound % with the % Berry-Esseen
\begin{equation}
  \label{fjkld13}
  \sup_{x\in \real}
\big| \P \big( \widetilde{N}_G \leq x \big) - \Phi(x) \big| \leq
C \lambda^{ - 1/(4r - 2)},
\end{equation}
 with rate \textcolor{red}{exponent} $1/(4r -2)$ as $\lambda$ tends to infinity, where $C>0$ depends only on $H$ and $G$.
\end{corollary}
\begin{proof}
   In the dilute regime, the cumulant bound 
% \eqref{Statuleviciuscond},
 \eqref{Statuleviciuscond} 
% \eqref{jfkla},
% \eqref{fjklds34}, 
% \eqref{fjklds34-2}
 shows that
 the centered and normalized subgraph count
 $\widetilde{N}_G$ 
 satisfies the {Statulevi\v{c}ius condition}
 \eqref{Statuleviciuscond2} in \textcolor{red}{the} \hyperref[appn]{Appendix}, see \cite{rudzkis,doering},
 with $\gamma := r-1$. 
 We conclude by applying Corollary~\ref{t1-c}
 and Lemma~\ref{l1}-$i)$ with $\gamma :=r-1$
 and $\Delta_\lambda:=\sqrt{K \lambda}$. 
\end{proof}
In the sparse regime we have the following result, 
 in which \eqref{fjkldsa1} is consistent with
 \eqref{fjkld13} when $\alpha = 1$.
\begin{corollary}[Sparse regime]
  \label{c01-2}
  Let $G$ be a tree with $|V(G)| = r \geq 2$ vertices,
  satisfying Assumption~\ref{a61}. 
  In the sparse regime \eqref{fjnldsf-2}
  with $\alpha \in [1, r/(r-1) )$,
  the normalized subgraph count $\widetilde{N}_G$ in 
  $G_{H_\lambda} (\Xi)$ satisfies
  the Kolmogorov distance bound % with the % Berry-Esseen
\begin{equation}
\label{fjkldsa1} 
\sup_{x\in \real}
\big| \P \big( \widetilde{N}_G \leq x \big) - \Phi(x) \big| \leq
C \lambda^{ - (
 \alpha   -(\alpha - 1)r 
    ) / ( 4r - 2) }, 
\end{equation} 
 as $\lambda$ tends to infinity, where $C>0$ depends only on $H$ and $G$.
\end{corollary}
\begin{proof}
  This is a consequence of 
  Corollary~\ref{th6.4-c}-$a)$
  and Lemma~\ref{l1}-$i)$ in \textcolor{red}{the} \hyperref[appn]{Appendix},
  with $\gamma :=r-1$ and 
  $\Delta_\lambda:=
  (K \lambda )^{ - (
 \alpha   -(\alpha - 1)r 
    )/2}$
  and $\alpha \in [1, r/(r-1) )$. 
\end{proof} 
  We note that  up to division by $2r - 1$,
   the rate in \eqref{fjkldsa1} is consistent  
 with the rate \textcolor{red}{exponent} $(
 \alpha    -(\alpha - 1)r 
     ) / 2$ obtained for the counting of trees in the
 Erd{\H o}s-R\'enyi graph, cf. Corollary~4.10 of \cite{PS2}. 
 In addition, since 
$
(\alpha |E(G)|-r)(n/2-1)
\geq
(\alpha -1)(n/2-1)r \geq 0$,
no significant Kolmogorov bounds
are derived from \eqref{fjklds34}
and \eqref{fjklds34-2} 
for cycle and other non-tree graphs 
in the sparse regime,
which is consistent with
Corollaries~4.8-4.9 of \cite{PS2}. 

\medskip

 Taking $\Delta_\lambda = \sqrt{K \lambda}$, 
% in Examples~\ref{examplea}-\ref{exampleb} above, 
 by Lemma~\ref{l1}-$ii)$ in \textcolor{red}{the} \hyperref[appn]{Appendix}, see Theorem~1.1 of \cite{doring},
we have the following result. 
\begin{corollary} % [Moderate deviation]
  \label{c01-2-0}
  Let $G$ be a connected graph with $|V(G)| = r \geq 2$ vertices,
  satisfying Assumption~\ref{a61}. 
  The normalized subgraph count $\widetilde{N}_G$
    satisfies a moderate deviation principle
  in the dilute regime of Corollary~\ref{c01}, 
  with speed $a_\lambda^2 = o( \lambda^{1/(2r - 1)} )$ and rate function $x^2/2$.
\end{corollary}
In addition,
by Lemma~\ref{l1}-$iii)$ in \textcolor{red}{the} \hyperref[appn]{Appendix},
see the corollary of \cite[Lemma~2.4]{saulis},
for any $x\ge0$ and sufficiently large $\lambda$,
for some constant $K>0$ 
we have the concentration inequality 
\begin{equation}
\label{concentrationineq}
  \P \big( \big| \widetilde{N}_G \big|
  \geq x)\le2\exp\left(-
  \frac{1}{4} \textcolor{red}{\min\left\{\frac{x^2}{2^r},
   \left( x\sqrt{K \lambda}\right)^{1/r}\right\}
  }
   \right), 
\end{equation} 
in agreement with the rate in Theorem~1.1 of \cite{bachmann},
which is stated for subgraph counts in random geometric
graphs. 
\section{Subgraph counts in random geometric graphs}
\label{rgg}
\noindent
In this section, we consider subgraph counts in
the (Poisson) random geometric graph model.
Assume that $\mu$ is the Lebesgue measure,
and that the intensity measure $\Lambda$ takes the form 
$$\Lambda (\mathrm{d}x) := {\bf 1}_A (x) \mu(\mathrm{d}x),\qquad\lambda>0,$$
  where $A$ is a Borel subset of $\real^d$
  such that $\mu (A)<\infty$. 
\begin{definition}
  For every $\lambda >0$,
  let $G_{H_\lambda} (\Xi)$
  denote the random-connection model
  with connection function 
  $$
  H_\lambda (x,y):=\bone_{\{\|x-y\|\leq R_\lambda\}},
  \qquad x,y\in\R^d,
  $$
  for some function $R_\lambda>0$ of $\lambda$.
 We consider the following regimes. 
\begin{itemize}
% \item Full random graph regime: $c_\lambda = 1 $, $\lambda > 0$. 
\item Dense regime: we have 
$$ 
    \lim_{\lambda\to \infty} \lambda R_\lambda^d \in (0,\infty]. 
$$
\item Sparse regime: we have 
$$ 
 \lim_{\lambda\to \infty} \lambda R_\lambda^d = 0
\ \mathrm{and} \
\lim_{\lambda\to \infty}
\lambda \big( \lambda R_\lambda^d \big)^{r-1} = \infty.
$$
\end{itemize} 
\end{definition}
When $\lim_{\lambda\to \infty} \lambda R_\lambda^d = c \in (0,\infty )$, 
we also say that we are in the thermodynamic regime.
The following result also extends Proposition~3.2 of \cite{lachiezerey2}
from second order cumulants to cumulants of any order. 
\begin{prop}
    \label{thm8}
 Let $G$ be a connected graph with $|V(G)|=r$ vertices, $r\geq 2$.
 In the random geometric graph model we have the following cumulant bounds. 
 \begin{enumerate}%[a)]
 \item (Dense regime). 
   We have
   \begin{equation}
     \label{b1}
     K_1 (n-1)! \lambda^{1+(r-1)n} ( R_\lambda^d)^{(r-1)n}
     \leq \kappa_n(N_G) \leq K_2 n!^r r!^{n-1}
 \lambda^{1+(r-1)n} ( R_\lambda^d)^{(r-1)n}, 
 \quad \lambda \geq 1,
   \end{equation}
% and \begin{equation} \kappa_2(N_G)\geq c_d\lambda_n, \end{equation}
for some constants $K_1, K_2 > 0$ independent
of $\lambda , n\geq 1$. 
  \item (Thermodynamic regime). 
   We have
   \begin{equation}
\label{b2}
K_1 \lambda \leq \kappa_n(N_G)\leq K_2 n!^r r!^{n-1}\lambda,
  \quad \lambda \geq 1, 
\end{equation}
 % and \begin{equation} \kappa_2(N_G)\geq c_d\lambda_n, \end{equation}
for some constants $K_1, K_2 > 0$ independent
of $\lambda , n\geq 1$. 
  \item (Sparse regime). 
   We have
   \begin{equation}
 \label{b3}
         K_1 \lambda^r ( R_\lambda^d)^{r-1} 
     \leq \kappa_n(N_G)\leq K_2 n!^r r!^{n-1} \lambda^r (R_\lambda^d)^{r-1},
     \quad \lambda \geq 1,
   \end{equation}
 % and \begin{equation} \kappa_2(N_G)\geq c_d\lambda_n, \end{equation}
for some constants $K_1, K_2 > 0$ independent
of $\lambda , n\geq 1$. 
 \end{enumerate}
\end{prop}
\begin{proof}
 By Proposition~\ref{lma-diagram1},
 letting $\widebar{\rho}$ denote a spanning tree contained in $\rho$,
  we have 
\begin{eqnarray}
  \nonumber
  \kappa_n(N_G)&=&\sum_{\rho \in \Pi_{\widehat{1}} ( [n] \times [r])
    \atop
    {\rho \wedge \pi = \widehat{0}}
  }
  \lambda^{|\rho|}\int_{A^{|\rho |}}
  \left(
\prod_{\{i,j\} \in E(\rho_G)}
\bone_{\{\|x_i-x_j\|\leq R_\lambda\}}
\right)
\mathrm{d}x_1\cdots\mathrm{d}x_{|\rho|}
\\
\nonumber
&\leq &\sum_{\rho \in \Pi_{\widehat{1}} ( [n] \times [r])
    \atop
    {\rho \wedge \pi = \widehat{0}}
  }
  \lambda^{|\rho|}\int_{A^{|\rho |}}
  \left(
\prod_{\{i,j\} \in E(\widebar{\rho}_G)}
\bone_{\{\|x_i-x_j\|\leq R_\lambda\}}
\right)
\mathrm{d}x_1\cdots\mathrm{d}x_{|\rho|}
\\
\nonumber
&=&\sum_{\rho \in \Pi_{\widehat{1}} ( [n] \times [r])
    \atop
    {\rho \wedge \pi = \widehat{0}}
  }
  \lambda^{|\rho|} \mu (A) (v_d R_\lambda^d)^{|\rho|-1}. 
\end{eqnarray}
 
  \noindent
$a)$ 
  In the dense regime
  with $\lim_{\lambda\to \infty} \lambda R_\lambda^d = \infty$,
  the dominating asymptotic order 
  $\lambda(\lambda R_\lambda^d)^{(r-1)n}$ of $\kappa_n(N_G)$ is
  achieved when $|\rho|=1+(r-1)n$,
  which yields the upper bound in \eqref{b1}. 

\noindent
$b)$ 
    In the thermodynamic regime
    with $\lim_{\lambda\to \infty} \lambda R_\lambda^d = c>0$,
    the dominating asymptotic order of $\kappa_n(N_G)$ is $\lambda$,
    which yields the upper bound in \eqref{b2}. 

  \noindent
$c)$ 
    In the sparse regime
    with $\lim_{\lambda\to \infty} \lambda R_\lambda^d = 0$ and
    $\lim_{\lambda\to \infty} \lambda(\lambda R_\lambda^d)^{r-1} = \infty$,
    the dominating asymptotic order $\lambda(\lambda R_\lambda^d)^{r-1}$
    of $\kappa_n(N_G)$ is achieved when $|\rho|=r$, 
    which yields the upper bound in \eqref{b3}. 

    \medskip 

 In addition, the kernel 
 $H_\lambda (x,y)=\bone_{\{\|x-y\|\leq R_\lambda\}}$
 satisfies Assumption~\ref{a61} for all $n\geq 1$,
 with $C_{\mu , H_\lambda} = v_d (R_\lambda /2)^d$
 in the framework of Example~\ref{examplea},
 which similarly yields the lower bounds in 
 \eqref{b1}-\eqref{b3}.
\end{proof}
 The next result is a direct consequence of Proposition~\ref{thm8}. \textcolor{red}{We notice that the asymptotic behavior of the expectation and the variance of subgraph in random geometric graph was also obtained in \cite{bachmann}.}
\begin{corollary}
    \label{thm8-0}
 Let $G$ be a connected graph with $|V(G)|=r$ vertices, $r\geq 2$.
 In the random geometric graph model we have the following
 normalized cumulant bounds.  
 \begin{enumerate}%[a)]
 \item (Dense and thermodynamic regime). 
   We have
   \begin{equation}
     \kappa_n\big(\widetilde{N}_G\big) \leq n!^r ( K \lambda )^{-(n/2-1)}, 
     \quad \lambda \geq 1,
   \end{equation}
% and \begin{equation} \kappa_2(N_G)\geq c_d\lambda_n, \end{equation}
for some $K>0$ constant independent of $\lambda , n\geq 1$. 
  \item (Sparse regime). 
   We have
   \begin{equation}
     \kappa_n\big(\widetilde{N}_G\big)\leq K \big(
      \lambda^r R_\lambda^{(r-1)d} \big)^{-(n/2-1)},
      \quad \lambda \geq 1,
   \end{equation}
 % and \begin{equation} \kappa_2(N_G)\geq c_d\lambda_n, \end{equation}
for some constants $K_1, K_2 > 0$ independent of $\lambda , n\geq 1$. 
 \end{enumerate}
\end{corollary}
 The following result then follows from Corollary~\ref{thm8-0}. 
\begin{corollary}
  \label{jdkj10}
  Let $G$ be a connected graph with $|V(G)|=r$ vertices, $r\geq 2$,
  in the random geometric graph model.
  \begin{enumerate}%[i)]
    \item 
    Dense \!\! \slash \ thermodynamic regimes. 
  If $\lim_{\lambda\to\infty} (\lambda R_\lambda^d) \in (0,\infty]$, 
 we have
\begin{equation}
\nonumber
  \sup_{x\in \real}
\big| \P \big( \widetilde{N}_G \leq x \big) - \Phi(x) \big| \leq
C
\lambda^{-1/ ( 4r - 2) }.
\end{equation}
\item
  Sparse regime.
  If $\lim_{\lambda\to\infty} \lambda R_\lambda^d = 0$ and
  $\lim_{\lambda\to\infty} \lambda(\lambda R_\lambda^d)^{r-1} = \infty$,
  we have 
\begin{equation}
\sup_{x\in \real}
\big| \P \big( \widetilde{N}_G \leq x \big) - \Phi(x) \big| \leq
C
\big(\lambda^r R_\lambda^{(r-1)d}\big)^{-1/ ( 4r - 2) }.
\end{equation}
\end{enumerate}
\end{corollary}
\begin{proof}
  In both cases $(i)$ and $(ii)$ we apply Corollary~\ref{thm8-0} 
  and Lemma~\ref{l1}-$i)$ with $\gamma :=r-1$,
  by taking 
  $\Delta_\lambda:=
  \sqrt{\lambda}$
  in the dense and  thermodynamic regimes,
  and 
  $\Delta_\lambda:= \lambda^{r/2} R_\lambda^{(r-1)d/2}$
  in the sparse regime.
\end{proof} 
We note that Berry-Esseen convergence rates have been obtained
for certain random functionals in the random geometric graph
model, including total edge lengths in \cite[Corollary~4.3]{schulte},
clique counts using Poisson U-statistics in \cite[Theorem~4.1]{reitzner}, and using stabilizing functionals in \cite[Theorem~3.15]{lachiezerey4},
and $k$-hop counts in the one-dimensional unit disk model in
\cite[Proposition~8.1]{privaultkhops}. 
% \textcolor{purple}{ Rates of normal convergence with respect to the Kolmogorov distance have been obtained for subgraph counts on random geometric graphs in \cite{schulte} using Poisson U-statistics, see also \cite{lachiezerey4} using stabilizing functionals. } 

 % \begin{remark}The monograph \cite{penrose} contains very comprehensive studies around the random geometric graph. Some non-quantitative central limit theorems are obtained for subgraph and component counts in \cite[Chapter~3]{penrose}. Rates of normal convergence with respect to the Kolmogorov distance were deduced for subgraph counts of the random geometric graph in \cite{schulte} as a Poisson U-statistics, and also in \cite{lachiezerey4} as a stabilizing functional. \end{remark}

% \begin{remark}\label{rem-1} For subgraph counts in general RCM, a CLT were yielded in \cite{can2022} under a weakly stabilizing condition, c.f. \cite{penrose01}, plus some moment conditions. As mentioned in \cite[Remark~2.5]{can2022}, the stabilizing condition appeared in \cite{penrose05,lachiezerey4} does not apply for subgraph counts with general random connection function. \end{remark}

\subsubsection*{Moderate deviation and concentration inequalities} 
\noindent 
Letting $\gamma =r-1$, 
In the dense and thermodynamic regimes of 
Corollary~\ref{jdkj10} with $\Delta_\lambda=
\sqrt{\lambda}$, $\widetilde{N}_G/a_\lambda$
    satisfies a moderate deviation principle
with rate function $x^2/2$
and speed $a_\lambda^2 = o( \lambda^{1/(2r - 1)} )$
in the setting of Lemma~\ref{l1}-$ii)$ in \textcolor{red}{the} \hyperref[appn]{Appendix},
and the concentration inequality 
\eqref{concentrationineq} holds by Lemma~\ref{l1}-$iii)$. 

%%%%%%%%%%%%%%%%%%%%%%%%%%%%%%%%%%%%%%%%%%%%%%
%% Example with single Appendix:            %%
%%%%%%%%%%%%%%%%%%%%%%%%%%%%%%%%%%%%%%%%%%%%%%
\begin{appendix}
\section*{}\label{appn} %% if no title is needed, leave empty \section*{}.
\noindent 
The following results are summarized from the ``main lemmas'' in Chapter~2 of \cite{saulis} and \cite{doring}, and are tailored to our RCM applications.
% We should mention that the assumption \eqref{Statuleviciuscond2} is now known as the {Statulevi\v{c}ius condition}. 
\begin{lemma}
  \label{l1}
  Let $(X_\lambda)_{\lambda \geq 1}$ be a family of random variables with mean zero and unit variance for all $\lambda>0$. Suppose that for all $\lambda \geq 1$, all moments of the random variable $X_\lambda$ exist and that % and sufficiently large $\lambda$,
  the cumulants of $X_\lambda$ satisfy  
  \begin{equation}
    \label{Statuleviciuscond2}
    |\kappa_n (X_\lambda)|\le\frac{(n!)^{1+\gamma}}{(\Delta_\lambda)^{n-2}},
    \qquad
 n\ge3, 
\end{equation}
  where $\gamma\ge0$ is a constant not depending on $\lambda$, while $\Delta_\lambda\in(0,\infty)$ may depend on $\lambda$.
   Then, the following assertions hold.
\begin{enumerate}%[i)]
\item (Kolmogorov bound,
 \cite[Corollary~2.1]{saulis} and \cite[Theorem~2.4]{doering})
  One has
\begin{equation}
\sup_{x\in\R}|\IP(X_\lambda\leq x)-\Phi(x)|\leq \frac{C}{(\Delta_\lambda)^{1/(1+2\gamma)}},
\end{equation}
for some constant $C>0$ depending only on $\gamma$.
\item (Moderate deviation principle,
  \cite[Theorem~1.1]{doring} \textcolor{red}{and \cite[Theorem~3.1]{doering}}).
  Let $( a_\lambda )_{\lambda > 0}$ be a sequence of real numbers tending to infinity, and such that 
  $$
  \lim_{\lambda \to \infty}
  \frac{a_\lambda}{(\Delta_\lambda)^{1/(1+2\gamma)}}
  = 0.
  $$
  % as $\lambda\to\infty$.
  Then, $ (a_\lambda^{-1}X_\lambda)_{\lambda >0}$ satisfies a moderate deviation principle with speed $a_\lambda^2$ and rate function $x^2/2$. 
\item (Concentration inequality,
  corollary of \cite[Lemma~2.4]{saulis} \textcolor{red}{and \cite[Theorem~2.5]{doering}}).
  For any $x\ge0$ and sufficiently large $\lambda$, 
\begin{equation}
\IP(|X_\lambda|\geq x)\le2\exp\left(-\frac14\min\left( \frac{x^2}{2^{\textcolor{red}{1+\gamma}}},(x\Delta_\lambda)^{1/(1+\gamma)}\right) \right).
\end{equation} 
% \item (Normal approximation with Cram\'er corrections, \cite[Lemma~2.3]{saulis}). There exists a constant $c>0$ such that for all $\lambda \geq 1$ and $x\in(0,c(\Delta_\lambda)^{1/(1+2\gamma)})$ we have 
% \begin{eqnarray*} \frac{\IP(X_\lambda\geq x)}{1-\Phi(x)}&=& \left(1+O\left(\frac{x+1}{(\Delta_\lambda)^{1/(1+2\gamma)}}\right)\right) \exp \big( \tilde{L}(x) \big), \\ \frac{\IP(X_\lambda\leq -x)}{\Phi(-x)}&=& \left(1+O\left(\frac{x+1}{(\Delta_\lambda)^{1/(1+2\gamma)}}\right)\right) \exp \big( \tilde{L}(-x) \big), \end{eqnarray*} where $\tilde{L}(x)$ is related to the Cram\'er-Petrov series. 
\end{enumerate}
\end{lemma}
\end{appendix}
%%%%%%%%%%%%%%%%%%%%%%%%%%%%%%%%%%%%%%%%%%%%%%
% %% Example with multiple Appendixes:        %%
% %%%%%%%%%%%%%%%%%%%%%%%%%%%%%%%%%%%%%%%%%%%%%%
% \begin{appendix}
% \section{Title of the first appendix}\label{appA}
% If there are more than one appendix, then please refer to it
% as \ldots\ in Appendix \ref{appA}, Appendix \ref{appB}, etc.

% \section{Title of the second appendix}\label{appB}
% \subsection{First subsection of Appendix \protect\ref{appB}}

% Use the standard \LaTeX\ commands for headings in \verb|{appendix}|.
% Headings and other objects will be numbered automatically.
% \begin{equation}
% \mathcal{P}=(j_{k,1},j_{k,2},\dots,j_{k,m(k)}). \label{path}
% \end{equation}

% Sample of cross-reference to the formula (\ref{path}) in Appendix \ref{appB}.
% \end{appendix}

%%%%%%%%%%%%%%%%%%%%%%%%%%%%%%%%%%%%%%%%%%%%%%
%% Support information, if any,             %%
%% should be provided in the                %%
%% Acknowledgements section.                %%
%%%%%%%%%%%%%%%%%%%%%%%%%%%%%%%%%%%%%%%%%%%%%%
\begin{acks}[Acknowledgments]
  We thank M. Schulte and C. Th\"ale for a correction to an earlier
  version of Lemma~\ref{fjkldsf-l}.
\end{acks}

%%%%%%%%%%%%%%%%%%%%%%%%%%%%%%%%%%%%%%%%%%%%%%
%% Funding information, if any,             %%
%% should be provided in the                %%
%% funding section.                         %%
%%%%%%%%%%%%%%%%%%%%%%%%%%%%%%%%%%%%%%%%%%%%%%
\begin{funding}
The first author was supported by NSF Grant DMS-??-??????.

The second author was supported in part by NIH Grant ???????????.
\end{funding}

%%%%%%%%%%%%%%%%%%%%%%%%%%%%%%%%%%%%%%%%%%%%%%
%% Supplementary Material, including data   %%
%% sets and code, should be provided in     %%
%% {supplement} environment with title      %%
%% and short description. It cannot be      %%
%% available exclusively as external link.  %%
%% All Supplementary Material must be       %%
%% available to the reader on Project       %%
%% Euclid with the published article.       %%
%%%%%%%%%%%%%%%%%%%%%%%%%%%%%%%%%%%%%%%%%%%%%%
% \begin{supplement}
% \stitle{Title of Supplement A}
% \sdescription{Short description of Supplement A.}
% \end{supplement}
% \begin{supplement}
% \stitle{Title of Supplement B}
% \sdescription{Short description of Supplement B.}
% \end{supplement}

%%%%%%%%%%%%%%%%%%%%%%%%%%%%%%%%%%%%%%%%%%%%%%%%%%%%%%%%%%%%%
%%                  The Bibliography                       %%
%%                                                         %%
%%  imsart-???.bst  will be used to                        %%
%%  create a .BBL file for submission.                     %%
%%                                                         %%
%%  Note that the displayed Bibliography will not          %%
%%  necessarily be rendered by Latex exactly as specified  %%
%%  in the online Instructions for Authors.                %%
%%                                                         %%
%%  MR numbers will be added by VTeX.                      %%
%%                                                         %%
%%  Use \cite{...} to cite references in text.             %%
%%                                                         %%
%%%%%%%%%%%%%%%%%%%%%%%%%%%%%%%%%%%%%%%%%%%%%%%%%%%%%%%%%%%%%

%% if your bibliography is in bibtex format, uncomment commands:
%\bibliographystyle{imsart-nameyear} % Style BST file (imsart-number.bst or imsart-nameyear.bst)
%\bibliography{bibliography}       % Bibliography file (usually '*.bib')

%% or include bibliography directly:
\begin{thebibliography}{4}
  \bibitem[\protect\citeauthoryear{Barbour, Karo{\'n}ski and Ruci{\'n}ski}{1989}]{BKR}
  Barbour,~A. D., Karo{\'n}ski,~M. and Ruci{\'n}ski,~A.~(1989).
  A central limit theorem for decomposable random variables with
    applications to random graphs. \textit{J. Combin. Theory Ser. B}~\textbf{47} 125--145.
  
  \bibitem[\protect\citeauthoryear{Bender, Odlyzko and Richmond}{1985}]{eabender}
  Bender,~E. A., Odlyzko,~A. M. and Richmond,~L. B.~(1985). The asymptotic number of irreducible partitions. \textit{European J. Combin.}~\textbf{6} 1--6.


  \bibitem[\protect\citeauthoryear {Balakrishnan and Ranganathan}{2012}]{balakrishnan}
  Balakrishnan,~R. and Ranganathan,~K.~(2012).
  \textit{A Textbook of Graph Theory}.
  Universitext. Springer, New York.

  \bibitem[\protect\citeauthoryear {Bachmann and Reitzner}{2018}]{bachmann}
  Bachmann,~S. and Reitzner,~M.~(2018). 
  Concentration for Poisson $U$-statistics: Subgraph counts in random geometric graphs.
  \textit{Stochastic Process. Appl.}~\textbf{128} 3327--3352.

  \bibitem[\protect\citeauthoryear {Bogdan, Rosi\'{n}ski, Serafin and Wojciechowski}{2017}]{bogdan}
  Bogdan,~K., Rosi\'{n}ski,~J., Serafin,~G. and Wojciechowski,~L.~(2017).
  L\'{e}vy systems and moment formulas for mixed Poisson integrals.
  In \textit{Stochastic analysis and related topics}~\textbf{72} of \textit{Progr. Probab.}, 139--164. Birkh{\"{a}}user/Springer, Cham.

  \bibitem[\protect\citeauthoryear {Can and Trinh}{2022}]{can2022}
  Can,~V. H. and Trinh,~K. D.~(2022).
  Random connection models in the thermodynamic regime: central limit theorems for add-one cost stabilizing functionals.
  \textit{Electron. J. Probab.}~\textbf{27} 1--40.

  \bibitem[\protect\citeauthoryear {D{\"{o}}ring and Eichelsbacher}{2013}]{doring}
  D{\"{o}}ring,~H. and Eichelsbacher,~P.~(2013).
  Moderate deviations via cumulants.
  \textit{J. Theoret. Probab.}~\textbf{26} 360--385.

  \bibitem[\protect\citeauthoryear {D{\"o}ring, Jansen and Schubert}{2022}]{doering}
  D{\"o}ring,~H., Jansen,~S. and Schubert,~K.
  The method of cumulants for the normal approximation.
  \textit{Probab. Surv.}~\textbf{19} 185--270.

  \bibitem[\protect\citeauthoryear {Eichelsbacher and Redno{\ss}}{2023}]{rednos}
  Eichelsbacher,~P. and Redno{\ss},~B.~(2023).
  Kolmogorov bounds for decomposable random variables and subgraph counting by the Stein-Tikhomirov method. \textit{Bernoulli}~\textbf{29} 1821--1848. 

  \bibitem[\protect\citeauthoryear{Eichelsbacher and Th{\"a}le}{2014}]{eichelsbacher}
  Eichelsbacher,~P. and Th{\"a}le,~C.~(2014).
  New Berry-Esseen bounds for non-linear functionals of Poisson random measures.
  \textit{Electron. J. Probab.}~\textbf{19} 1--25.

  \bibitem[\protect\citeauthoryear {Erd{\Horig{o}}s and R\'enyi}{1959}]{ER}
Erd{\Horig{o}}s,~P. and R\'enyi,~A.~(1959).
  On random graphs, {I}.
  \textit{Publ. Math. Debrecen}~\textbf{6} 290--297.

  \textcolor{red}{
\bibitem[\protect\citeauthoryear{F\'erary, M\'eliot and Nikghbali}{2016}]{feray}
F\'erary,~V., M\'eliot,~P.-L. and Nikghbali,~A.~(2016). 
\textit{Mod-$\phi$ convergence.} 
Springer Briefs in Probability and Mathematical Statistics. Springer, Cham.
}

  \bibitem[\protect\citeauthoryear{Gilbert}{1959}]{G} Gilbert,~E.N.~(1959).
  Random graphs.
  \textit{Ann. Math. Statist}~\textbf{30} 1141--1144.

  \textcolor{red}{
\bibitem[\protect\citeauthoryear{G\"otze, Heinrich and Hipp}{1995}]{gotze95}
G\"otze,~F., Heinrich,~L. and Hipp,~C.~(1995). $m$-dependent random fields with analytic cumulant generating function. 
\textit{Scand. J. Statist.}~\textbf{22} 183--195. 
}

  \bibitem[\protect\citeauthoryear {Grote and Th{\"a}le}{2018a}]{grotethale18}
  Grote,~J. and Th{\"a}le,~C.~(2018a).
  Concentration and moderate deviations for Poisson polytopes and polyhedra.
  \textit{Bernoulli}~\textbf{24} 2811--2841.

  \bibitem[\protect\citeauthoryear {Grote and Th{\"a}le}{2018b}]{thale18}
  Grote,~J. and Th{\"a}le,~C.~(2018b).
  Gaussian polytopes: a cumulant-based approach.
  \textit{J. Complexity}~\textbf{47} 1--41.

  \textcolor{red}{
\bibitem[\protect\citeauthoryear{Gusakova and Th{\"a}le}{2021}]{thale21}
  Gusakova,~A. and Th{\"a}le,~C.~(2021). 
  The volume of simplices in high-dimensional Poisson-Delaunay tessellations.
  \textit{Annales Henri Lebesgue}~\textbf{4} 121--153.  
\bibitem[\protect\citeauthoryear{Heinrich}{2005}]{heinrich}Heinrich,~L.~(2005). 
Large deviations of the empirical volume fraction for stationary Poisson grain models. 
\textit{Ann. Appl. Probab.}~\textbf{15} 392--420.
\bibitem[\protect\citeauthoryear{Heinrich and Spiess}{2009}]{heinrich09}
Heinrich,~L. and Spiess,~M.~(2009). 
Berry-Esseen bounds and Cram\'er-type large deviations for the volume distribution of Poisson cylinder processes. 
\textit{Lith. Math. J.}~\textbf{49} 381--398.
}

  \bibitem[\protect\citeauthoryear {Janson}{1988}]{Janson1988}
  Janson,~S.~(1988).
  Normal convergence by higher semiinvariants with applications to sums of dependent random variables and random graphs.
  \textit{Ann. Probab.}~\textbf{16} 305--312.

  \bibitem[\protect\citeauthoryear {Jansen}{2019}]{jansen}
  Jansen,~S.~(2019).
  Cluster expansions for Gibbs point processes.
  \textit{Adv. in Appl. Probab.}~\textbf{51} 1129--1178.

  \bibitem[\protect\citeauthoryear {Khorunzhiy}{2008}]{khorunzhiy}
  Khorunzhiy,~O.~(2008).
  On connected diagrams and cumulants of {E}rd{\Horig{o}}s-{R}\'enyi matrix models.
  \textit{Comm. Math. Phys.}~\textbf{282} 209--238.

  \bibitem[\protect\citeauthoryear {Krokowski, Reichenbachs and Th{\"a}le}{2017}]{reichenbachsAoP}
  Krokowski,~K., Reichenbachs,~A. and Th{\"a}le,~C.~(2017).
  Discrete Malliavin-Stein method: Berry-Esseen bounds for
    random graphs and percolation.
  \textit{Ann. Probab.}~\textbf{45} 1071--1109.

  \bibitem[\protect\citeauthoryear{Lachi\`eze-Rey and Peccati}{2013}]{lachiezerey2}
  Lachi\`eze-Rey,~R. and Peccati,~G.~(2013).
  Fine {G}aussian fluctuations on the {P}oisson space {II}: rescaled kernels, marked processes and geometric {$U$}-statistics.
  \textit{Stochastic Process. Appl.}~\textbf{123} 4186--4218.

  \bibitem[\protect\citeauthoryear {Lachi\`eze-Rey and Reitzner}{2016}]{lachieze-rey}
  Lachi\`eze-Rey,~R. and Reitzner,~M.~(2016).
  $U$-statistics in stochastic geometry.
  {\em Stochastic Analysis for {P}oisson Point Processes: {M}alliavin Calculus, {W}iener-{I}t{\^o} Chaos Expansions and Stochastic Geometry.} Cham: Springer International Publishing, 229--253.

  \bibitem[\protect\citeauthoryear {Lachi\`eze-Rey, Schulte and Yukich}{2019}]{lachiezerey4}
  Lachi\`eze-Rey,~R., Schulte,~M. and Yukich,~J.E.~(2019).
  Normal approximation for stabilizing functionals.
  \textit{Ann. Appl. Probab.}~\textbf{29} 931--993.

  \bibitem[\protect\citeauthoryear {Last, Nestmann and Schulte}{2021}]{LNS21}
  Last,~G., Nestmann,~F. and Schulte,~M.~(2021).
  The random connection model and functions of edge-marked {P}oisson processes: second order properties and normal approximation.
  \textit{Ann. Appl. Probab.}~\textbf{31} 128--168.

  \bibitem[\protect\citeauthoryear {Last, Peccati and Schulte}{2016}]{lastpeccatipenrose}
  Last,~G., Peccati,~G. and Schulte,~M.~(2016).
  Normal approximation on Poisson spaces: Mehler's formula, second order Poincar\'e inequality and stabilization.
  \textit{Probab. Theory Related Fields}~\textbf{165} 667--723.

  \bibitem[\protect\citeauthoryear {Last and Penrose}{2017}]{LastPenrose17}
  Last,~G. and Penrose,~M.D.~(2017).
  \textit{Lectures on the Poisson process}~(Vol. 7).
  Cambridge University Press.

  \bibitem[\protect\citeauthoryear {Malyshev and Minlos}{1991}]{MalyshevMinlos91}
  Malyshev,~V.A. and Minlos,~R.A.~(1991).
  \textit{Gibbs random fields}, Vol.44, {\em Mathematics and its Applications (Soviet Series)}.
  \newblock Kluwer Academic Publishers Group, Dordrecht.

  \bibitem[\protect\citeauthoryear {Peccati and Taqqu}{2011}]{peccatitaqqu}
  Peccati,~G. and Taqqu,~M.~(2011).
  \textit{Wiener Chaos: Moments, Cumulants and Diagrams: A survey with Computer Implementation}.
  \newblock Bocconi \& Springer Series. Springer.

  \bibitem[\protect\citeauthoryear {Penrose}{2003}]{penrosebk}
  Penrose,~M.~(2003).
  \textit{Random geometric graphs}, Vol.5 {\em Oxford Studies in Probability}.
  \newblock Oxford University Press.

  


  \bibitem[\protect\citeauthoryear {Penrose and Yukich}{2001}]{penrose01}
  Penrose,~M. and Yukich,~J.E.~(2001).
  Central limit theorems for some graphs in computational geometry.
  \textit{Ann. Appl. Probab.}~\textbf{11} 1005--1041.

  \bibitem[\protect\citeauthoryear {Penrose and Yukich}{2005}]{penrose05}
  Penrose,~M. and Yukich,~J.E.~(2005).
  Normal approximation in geometric probability.
  {\em Stein's method and applications}, Vol.5 {\em Lect.
    Notes Ser. Inst. Math. Sci. Natl. Univ. Singap.}, 37--58. Singapore Univ. Press.

    \bibitem[\protect\citeauthoryear{Privault}{2012}]{momentpoi}
  Privault,~N.~(2012).
  Moments of {P}oisson stochastic integrals with random integrands.
  \textit{Probab. Math. Statist.}~\textbf{32} 227--239.

  \bibitem[\protect\citeauthoryear {Privault}{2019}]{prkhp}
  Privault,~N.~(2019).
  Moments of $k$-hop counts in the random-connection model.
  \textit{J. Appl. Probab.}~\textbf{56} 1106--1121.

  \bibitem[\protect\citeauthoryear{Privault}{2022}]{privaultkhops}
  Privault,~N.~(2022).
  Asymptotic analysis of $k$-hop connectivity in the 1{D} unit disk random graph model.
  Preprint arXiv:2203.14535.

  \bibitem[\protect\citeauthoryear {Privault and Serafin}{2020}]{PS2}
  Privault,~N. and Serafin,~G.~(2020).
  Normal approximation for sums of discrete $U$-statistics -
  application to {K}olmogorov bounds in random subgraph counting.
  \textit{Bernoulli}~\textbf{26} 587--615.

  \bibitem[\protect\citeauthoryear {Privault and Serafin}{2022}]{PS4}
  Privault,~N. and Serafin,~G.~(2022).
  Berry-{E}sseen bounds for functionals of independent random variables.
  \textit{Electron. J. Probab.}~\textbf{27} 1--37.

  \bibitem[\protect\citeauthoryear {Reitzner and Schulte}{2013}]{reitzner}
  Reitzner,~M. and Schulte,~M.~(2013).
  Central limit theorems for ${U}$-statistics of {P}oisson point processes.
  \textit{Ann. Probab.}~\textbf{41} 3879--3909.

  \bibitem[\protect\citeauthoryear {R{\"o}llin}{2022}]{roellin2}
  R{\"o}llin,~A.~(2022).
  Kolmogorov bounds for the normal approximation of the number of triangles in the {E}rd{\Horig{o}}s-{R}\'enyi random graph.
  \textit{Probab. Engrg. Inform. Sci.}~\textbf{36} 747--773.

  \bibitem[\protect\citeauthoryear {Rota}{1964}]{rota1964}
  Rota,G.-C.~(1964).
  On the foundations of combinatorial theory. {I}. Theory of {M}\"obius functions.
  {\em Z. Wahrscheinlichkeitstheorie und Verw. Gebiete}~\textbf{2} 340--368.

  \bibitem[\protect\citeauthoryear {Rudzkis, Saulis and Statulevi\v{c}ius}{1978}]{rudzkis}
  Rudzkis,~R., Saulis,~L. and Statulevi\v{c}ius,~V.A.~(1978).
  A general lemma on probabilities of large deviations.
  \textit{Litovsk. Mat. Sb.}~\textbf{18} 99--116.

  \bibitem[\protect\citeauthoryear {Ruci{\'n}ski}{1988}]{rucinski}
  Ruci{\'n}ski,~A.~(1988).
  When are small subgraphs of a random graph normally distributed?
  \textit{Probab. Theory Related Fields}~\textbf{78} 1--10.

  \bibitem[\protect\citeauthoryear {Saulis and Statulevi\v{c}ius}{1991}]{saulis}
  Saulis,~L. and Statulevi\v{c}ius,~V.A.~(1991).
  \textit{Limit theorems for large deviations}, Vol.73, {\em
    Mathematics and its Applications (Soviet Series)}.
  \newblock Kluwer Academic Publishers Group, Dordrecht.

  \bibitem[\protect\citeauthoryear {Schulte}{2016}]{schulte}
  Schulte,~M.~(2016).
  Normal approximation of {P}oisson functionals in {K}olmogorov distance.
  \textit{J. Theoret. Probab.}~\textbf{29} 96--117.

  \bibitem[\protect\citeauthoryear {Schulte and Th{\"a}le}{2023}]{schulte-thaele}
  Schulte,~M. and Th{\"a}le,~C.~(2023).
  Moderate deviations on {P}oisson chaos.
  arXiv:2304:00876v1.

  \bibitem[\protect\citeauthoryear {Zhang}{2022}]{zhangzs}
  Zhang,~Z.S.~(2022).
  Berry-{E}sseen bounds for generalized {$U$}-statistics.
  \textit{Electron. J. Probab.}~\textbf{27} 1--36.


\end{thebibliography}

\end{document}
