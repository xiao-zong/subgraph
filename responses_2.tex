\documentclass[11pt]{article}

\let\Horig\H

\newcommand{\ignore}[1]{}

\usepackage{amssymb,amsmath}

\usepackage{enumerate}

\usepackage{url}

\usepackage{xcite}

\externalcitedocument{m}

\usepackage{xr} % [cross-referencing] 

\externaldocument[Z-]{m} % [cross-referencing] 

% \usepackage{xc} % [cross-referencing] 
% \externalcitedocument[AA-]{responses} % [cross-referencing] 

\makeatletter
\long\def\XR@test#1#2#3#4\XR@{%
  \let\XR@next\@gobbletwo
  \ifx#1\newlabel
    \let\XR@next\@firstoftwo%
  \else\ifx#1\@input
     \let\XR@next\@secondoftwo
  \fi\fi
   \XR@next{\newlabel{\XR@prefix#2}{#3}}{\edef\XR@list{\XR@list#2\relax}}%
  \ifeof\@inputcheck\expandafter\XR@aux
  \else\expandafter\XR@read\fi}
\makeatother

\usepackage{color}

\usepackage{xcolor}
\definecolor{ForestGreen}{RGB}{34,139,34}
\definecolor{mauve}{rgb}{0.7,0,0.43}
\definecolor{dkgreen}{rgb}{0,0.6,0}
\definecolor{darkgreen}{rgb}{0,0.6,0}
\definecolor{darkorange}{rgb}{1.0, 0.55, 0.0}
\definecolor{lightblue}{rgb}{0,0.2,0.5}
\definecolor{blue1}{rgb}{0,0.1,0.9}

\definecolor{lightblue}{rgb}{0,0.2,0.5}

\usepackage{soul}
\newcommand{\mathcolorbox}[2]{\colorbox{#1}{$\displaystyle #2$}}
\allowdisplaybreaks

\usepackage[colorlinks=true, urlcolor=lightblue,linkcolor=lightblue, citecolor=lightblue]{hyperref}

\usepackage{accsupp}    

\newcommand{\refb}[1]{\ref*{#1}}

\newcommand{\eqrefb}[1]{(\ref*{#1})}

\newcommand{\noncopynumber}[1]{
    \BeginAccSupp{method=escape,ActualText={}}
    #1
    \EndAccSupp{}
}

\usepackage{listings}
\lstset{ % 
  xleftmargin=3.4pt, % 3.4pt,
  xrightmargin=3.4pt,
  language=C,                % the language of the code 
  basicstyle=\tiny\ttfamily, % \scriptsize\ttfamily, % \footnotesize,           % the size of the fonts that are used for the code 
%  basicstyle=\ttfamily,
  numbers=left, % none,                   % where to put the line-numbers
  numberstyle=\tiny\color{gray}\noncopynumber,  % the style that is used for the line-numbers 
 stepnumber=2,                   % the step between two line-numbers. If it's 1, each line 
                                  % will be numbered 
  numbersep=5pt,                  % how far the line-numbers are from the code 
  backgroundcolor=\color{white},      % choose the background color. You must add \usepackage{color} 
  showspaces=false,               % show spaces adding particular underscores 
  showstringspaces=false,         % underline spaces within strings 
  showtabs=false,                 % show tabs within strings adding particular underscores 
  frame=single,                   % adds a frame around the code 
  rulecolor=\color{black},        % if not set, the frame-color may be changed on line-breaks within not-black text (e.g. commens (green here)) 
  tabsize=2,                      % sets default tabsize to 2 spaces 
  captionpos=b,                   % sets the caption-position to bottom 
  breaklines=true,                % sets automatic line breaking 
  breakatwhitespace=true, % false,        % sets if automatic breaks should only happen at whitespace 
  title=\lstname,                   % show the filename of files included with \lstinputlisting; 
                                  % also try caption instead of title 
  keywordstyle=\color{blue1},          % keyword style 
  commentstyle=\color{dkgreen},       % comment style 
  stringstyle=\color{mauve},         % string literal style 
  escapeinside={\%*}{*)},            % if you want to add a comment within your code 
  morekeywords={*,...},               % if you want to add more keywords to the set 
  columns=fullflexible,
  upquote
}

\usepackage[normalem]{ulem}

% \usepackage[round,sort,comma,numbers]{natbib}
% \bibpunct{\textcolor{lightblue}{(}}{\textcolor{lightblue}{)}}{,}{a}{}{;}
% \let\oldcitet=\citet
% \let\oldcitep=\citep
% \renewcommand{\cite}[1]{\textcolor[rgb]{0,0,1}{\oldcitet{#1}}}
% \renewcommand{\citet}[1]{\textcolor[rgb]{0,0,1}{\oldcitet{#1}}}
% \renewcommand{\citep}[1]{\textcolor[rgb]{0,0,1}{\oldcitep{#1}}}

% \numberwithin{equation}{section}

\newenvironment{Proof}{\removelastskip\par\medskip
\noindent{\em Proof.} \rm}{\penalty-20\null\hfill$\square$\par\medbreak}

\def\div{{\mathord{{\rm div}}}}
\def\Ad{{\mathord{{\rm Ad}}}}
\def\ad{{\mathord{{\rm ad}}}}
\def\trace{{\mathord{{\rm trace ~}}}}
\def\ric{{\mathord{{\rm ric}}}}
\def\so{{\mathord{{\rm so}}}}
\def\realic{{\mathord{{\rm Ric}}}}
\def\Id{{\mathord{{\rm Id}}}}
\def\diff{{\mathord{{\rm Diff}}}}
\def\real{{\mathord{{\rm I\kern-2.8pt R}}}}        % Fake blackboard bold R.
\def\inte{{\mathord{{\rm I\kern-2.8pt N}}}}
\def\PP{{\mathord{{\rm I\kern-2.8pt P}}}}
\def\ii{{\mathord{{\rm {\Im}}}}}
\def\G{{\mathord{{\sl {\sf G}}}}}

\def\real{{\mathord{\mathbb R}}}
\def\bbr{{\mathord{\mathbb R}}}
\def\inte{{\mathord{\mathbb N}}}
\def\z{{\mathord{\mathbb Z}}}
\def\qu{{\mathord{\mathbb Z}}}
\def\H{{\mathord{\mathbb H}}}
\def\HB{{\mathord{\mathbb{\bf H}}}}
\def\calf{{\cal F}}
\newcommand{\dee}{\mbox{$I  \! \! \! \, D$}}
\def\Cov{{\mathrm{{\rm Cov}}}}
\def\Var{{\mathrm{{\rm Var}}}}
\def\R{\right}
\def\L{\left}
\def\realef#1{(\ref{#1})}

\newcommand{\simgeq}{\;
\raisebox{-0.4ex}{\tiny$\stackrel{{\textstyle>}}{\sim}$}\;}
\newcommand{\simleq}{\;
\raisebox{-0.4ex}{\tiny$\stackrel{{\textstyle<}}{\sim}$}\;}
\newcommand{\theor}{{\bf Theorem}}
\newcommand{\Prop}{{\bf Proposition}}
\newcommand{\etape}{{\bf Step}}
\newcommand{\rem}{{\bf Remark:}}
\newcommand{\rems}{{\bf Remarks:}}
\newcommand{\ex}{{\bf Example:}}
\newcommand{\demo}{{\bf Proof:}}
\newcommand{\CQFD}{\nolinebreak\hfill\rule{2mm}{2mm}\medbreak\par}
\newcommand{\lem}{{\bf Lemma }}
\newcommand{\ind}{\mathbf{1}}
\newcommand{\sgn}{\mathrm{sign }\:}
\newcommand{\disp}{\displaystyle}
\newcommand{\cvar}{\stackrel{var}{\longrightarrow}}
\newcommand{\lto}{\longrightarrow}
\newcommand{\e}{\varepsilon}
\newcommand{\s}{\sigma}
\newcommand{\om}{\omega}

\def\rit{\mathbb{R}}
\def\cit{\mathbb{C}}
\def\nit{\mathbb{N}}
\def\zit{\mathbb{Z}}
\def\qit{\mathbb{Q}}
\def\dit{\mathbb{D}}
\def\Eit{\mathbb{E}}
\def\pit{\mathbb{P}}\def\P{\mathbb{P}}
\def\E{\mathop{\hbox{\rm I\kern-0.20em E}}\nolimits}
\def\Var{\mathop{\hbox{\rm Var}}\nolimits}
\def\Cov{\mathop{\hbox{\rm Cov}}\nolimits}

\def\cqfd{\hbox{\rule{2.5mm}{2.5mm}}}

\def\court {\hskip 5pt}
\def\med{\hskip 10pt}
\def\lng{\hskip 20pt}


\newtheorem{prop}{Proposition}[section]
\newtheorem{lemma}[prop]{Lemma}
\newtheorem{definition}[prop]{Definition}
\newtheorem{corollary}[prop]{Corollary}
\newtheorem{theorem}[prop]{Theorem}
\newtheorem{remark}[prop]{Remark}
\def\Dom{\rm Dom \ \! }
\def\trace{{\mathrm{{\rm trace}}}}
\def\Ent{{\mathrm{{\rm Ent}}}}
\def\var{{\mathrm{{\rm var}}}}
\def\div{{\mathrm{{\rm div}}}}

\textwidth16cm
\textheight22.5cm
\oddsidemargin-0.1cm
\evensidemargin-0.1cm
\topmargin0cm
\headheight0cm
\headsep0cm
\baselineskip1in
\parindent0.2in

\begin{document}

\pagestyle{empty} 

\begin{center} 
\large{Responses to Reviewer 1} % \#1}
\\ 
\end{center}

\noindent 
% We thank the referee for the many insightful comments and recommendations that have helped us in producing this revision.

\noindent
\begin{description}
\item 
Q1: Is it possible/worthwhile to provide a moment formula before discussing
the combinatorics?
Truth be told, my personal feeling upon reading this paper is that I had spent
a lot of time parsing sections 2-3 (understanding the notation and definitions
and propositions; I will come back to this later) before finally arriving at a first formula for moments of graph count (Proposition 4.1). In fact, this formula, as
well as Proposition 4.2, can be presented in a way that is easily understood by
an average probabilist like myself, without using the heavy notations in sections
2-3. Indeed, non-flat partition corresponds to identifying vertices, and edges are
independent once vertices are fixed. Presenting Proposition 4.1 and 4.2 before
sections 2-3 would motivate the introduction of connected partitions and their
connection with virtual cumulants (no pun intended).
\item
  Q2: Is it possible to be more precise about the combinatorial arguments?
The essence of the paper seems to revolve around two combinatorial estimates in Lemma 2.8 along with the idea of connecting partitions and cumulants.
After spending a substantial amount of time, I fail to understand Equation (3).
Starting from a non-flat connected partition for an array of shape $n \times r$
and
moving towards building a non-flat connected partition for an array of shape
$(n + 1) \times r$, it seems that one has to decide which row index will be the added r vertices. Then, for each added vertex, one has to decide which block it belongs
to (or if it is isolated). Since there are O(nr) blocks (a trivial upper bound, but
it does not seem easy to find a better bound), the number of possible ways of
``expanding from $n$ to $n + 1$'' is at most $O(n(nr)r ) = O(nr+1 )$
(considering large
n asymptotics with $r$ fixed). Would it be possible to provide more details in
proving the claimed O(nr ) bound (second display on page 8)?

\end{description} 

\newpage

\begin{center} 
\large{Responses to Reviewer 2} % \#1}
\\ 
\end{center}

\noindent 
% We thank the referee for the many insightful comments and recommendations that have helped us in producing this revision.

% \medskip

\begin{itemize}
  \item 
 I recommend to include a few more references to papers where the method of cumulants has already successfully been applied in stochastic geometry. A good overview can be found in the survey article [DJS22].
\item  When discussing cumulant bounds for subgraph counts it might be worth mentioning that for the
 Erd{\Horig o}s-R\'enyi graph 
 random graph model optimal (!) bounds are available from the booklet of Féray, Méliot and Nikeghbali.
\item  Whenever the random geometric graph is mentioned, do you need to assume that in the connection function also
  $\Vert x - y \Vert> 0$,
  or is this automatically satisfied by some other assumptions?
      \item I wonder why the reference [ST23] is not included in the discussion at the bottom of page 3.
  \item  page 10: When Poisson processes are introduced I guess that some assumptions on the intensity measure are necessary (such as being $\sigma$-finite).
  \item  page 10: Given the description of the space $\Omega$ it is not clear to me why the authors mention Corollary 6.5 of [LP18] here, because the claim seems to be part of the definition used here.
  \item  page 10: Do you need further conditions for (10) to be valid, for example is finiteness of the integral sufficient?
  \item  page 11: Same question for (11).
  \item  page 11/12: If might be a good ideal to briefly recall $\Xi$.
  \item  page 12, line 4: Change geometric to geometry.
  \begin{itemize}
    \item Nailed. 
  \end{itemize}
  \item  page 12: Maybe the authors would like to give a reference to the textbook on random geometric graphs at this place.
  \begin{itemize}
    \item We have added a classic book for RGG.
  \end{itemize}
  \item  page 13, line 3: On the connection function?
  \begin{itemize}
    \item We have deleted "on". It should be ``$H(x,y)$ is a connection funtion.'' 
  \end{itemize}
  \item  page 13: What is the meaning of µ in (16)? Also, is this the same as in the example thereafter?
  \begin{itemize}
    \item We have changed the notation, using $C_H$ instead of $C_{\mu, H}$.
  \end{itemize}
  \item  page 13: You refer to Example 6, a number which does not exist.
  \begin{itemize}
    \item We have added the numbering for the example.
  \end{itemize}
  \item  page 13: In the example at the bottom of the page you assume µ to be Radon. Why and why is this assumption not used before?
  \item  page 13: In the same example, is it really possible to allow
    $A_\lambda$ be so general that not even $A_\lambda \uparrow \real^d$?
  \item  page 14: It should be $x, y \in \real^d$ in Definition 6.1.
  \item  page 19: Is it correct to speak of $1/(4r - 2)$ as a rate? To me, this is at most the exponent of the rate. Maybe this can be skipped entirely.
  \item  page 20 and below: It should be the appendix if you refer to the appendix.
  \item  page 20: First, you should say where $x$ in (30) comes from. Moreover, I guess that $x$ should appear on both sides of the equation.
  \item  page 22: When it comes to the discussion of Corollary 8.3 there is also the paper of Reitzner and coauthors on the Gilbert graph that could be mentioned here.
  \item  page 23: If possible, a reference to both sources [SS91] and [DJS22] should be given for each item.
\end{itemize}

\end{document}

